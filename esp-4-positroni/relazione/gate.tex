\subsection{Gate dell'ADC}

Variamo la durata del gate dell'ADC e calcoliamo il rapporto $\sigma\!/$\!media per il picco di annichilazione rivelato dal PMT1, modellato come al solito con una gaussiana.
Su alcune misure abbiamo introdotto dei ritardi per migliorare o peggiorare la sincronizzazione con il gate al fine di quantificare quanto la messa in tempo influisca sulla nostra risoluzione. Il risultato di questa misura è in \autoref{fig:gate}, i dati sono in \autoref{tab:gate}.

\begin{figure}[h]
\centering
\includegraphics[width=20 em]{immagini/gate}
\caption{Grafico della risoluzione in funzione della durata del gate. Le misure sovrapposte sono state ottenute ritardando il segnale del PMT1.}
\label{fig:gate}
\end{figure}

\begin{table}[h]
\centering
\begin{tabular}{c|c|c|c}
durata (ritardo) [ns] & media [digit] & $\sigma$ [digit] & $\sigma\!/$\!media\\
\hline
 100 (40) & 304.88\,$\pm$\,0.34 & 307.68\,$\pm$\,0.36 &  1.0092\,$\pm$\,0.0016 \\
 190 (16) & 492.82\,$\pm$\,0.50 & 344.25\,$\pm$\,0.25 & 0.69853\,$\pm$\,0.0008 \\
 190 & 513.05\,$\pm$\,0.80 & 524.01\,$\pm$\,0.42 &  1.0213\,$\pm$\,0.0018 \\
 300 & 307.68\,$\pm$\,0.36 & 482.54\,$\pm$\,0.85 &  1.5683\,$\pm$\,0.0033 \\
 400 & 344.25\,$\pm$\,0.25 & 539.43\,$\pm$\,0.61 &  1.5670\,$\pm$\,0.0021 \\
 500 & 524.01\,$\pm$\,0.42 & 553.88\,$\pm$\,0.71 &  1.0570\,$\pm$\,0.0016 \\
 700 & 482.54\,$\pm$\,0.85 & 542.68\,$\pm$\,0.41 &  1.1247\,$\pm$\,0.0022 \\
1000 (16) & 539.43\,$\pm$\,0.61 & 304.88\,$\pm$\,0.34 & 0.56519\,$\pm$\,0.0009 \\
1000 (0) & 553.88\,$\pm$\,0.71 & 492.82\,$\pm$\,0.50 &  0.8898\,$\pm$\,0.0014 \\
1000 (32) & 542.68\,$\pm$\,0.41 & 513.05\,$\pm$\,0.80 &  0.9454\,$\pm$\,0.0016 
\end{tabular}

\caption{Andamento della risoluzione in funzione della durata del gate. I numeri tra parentesi indicano di quanto è stato ritardato il PMT1.}
\label{tab:gate}
\end{table}

Dall'analisi dei dati abbiamo potuto vedere che la risoluzione migliore si raggiunge quando il gate dura meno di \SI{200}{ns}
\marginpar{Con le incertezze riportate, mi sembra che, fatta eccezione per 300 e 400, le misure siano tutte compatibili.}
ma questo valore non è un buon punto di lavoro perché cambiare la durata del gate cambia la scala e, in questo caso, non ci permetteva di vedere il picco del neon.
\marginpar{Il motivo non è il picco del neon, perché accorciare il gate è come attenuare e quindi a maggior ragione si vedono i picchi più a destra. Inoltre usando gli attenuatori si può sempre compensare la variazione del gate.}
L'alternativa migliore sembrerebbe essere una durata del gate di \SI{1}{\micro s}, ma il manuale dell'ADC specifica che a tale valore (o anche maggiore) diminuisce la linearità della digitalizzazione. Per questo abbiamo scelto di usare un gate di \SI{550}{ns} nell'apparato A.
\marginpar{Il manuale dice che aumenta l'errore, non che peggiora la linearità (quella era per segnali sopra \SI1V). Inoltre come limite dà \SI{200}{ns}.}
Questa scelta è poi diventata \SI{1}{\micro s} nell'apparato B per evitare problemi derivanti dal jitter dei segnali. Infine notiamo che cambiare i ritardi sui segnali in ingresso non influisce significativamente sulla risoluzione dell'apparato.
\marginpar{I dati nella tabella non mi tornano. Le sigma sono grosse quanto le medie e infatti i rapporti vengono tutti dell'ordine dell'unità, diversi da quelli riportati nel grafico.}
