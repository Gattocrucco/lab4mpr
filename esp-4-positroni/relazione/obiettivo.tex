\subsection{Obiettivo}
Gli obiettivi principali dell'esperienza sono:
\begin{itemize}
	\item la misura della massa dell'elettrone a partire dallo spettro energetico di fotoni emessi dall'annichilazione di positroni;
	\item la misura del rapporto tra i branching ratio del decadimento del  \na\; per cattura elettronica e per emissione di positroni;
	\item la misura del rapporto tra le sezioni d'urto dell'annichilazione del $\beta^+$ in $2\gamma$ e quella in $3\gamma$ .
\end{itemize}

\subsection{Apparato di misura}
Gli strumenti principali a disposizione per questa esperienza (fatta eccezione per i soliti moduli \texttt{NIM}) consistono in:
\begin{itemize}
	\item una sorgente radioattiva principale di \na\; e altre sorgenti meno attive per la calibrazione: \cs\;, \co\;;
	\item i PMT1, 2 e 3\footnote{I PMT1 e 2 sono dello stesso modello}: rivelatori basati su cristalli scintillatori di NaI(Tl) che useremo come spettrometri;
	\item un'ADC \texttt{CAMAC} (8 canali, 12 bit) a integrazione di carica.
\end{itemize}


\begin{comment}

\subsubsection{Sorgenti radioattive}
La sorgente radioattiva principale di \na\; decade al $90.6\%$ $\beta^+$ e al $9.6\%$ per cattura elettronica in uno stato eccitato di $^{22}$Ne che a sua volta decade $\gamma$ emettendo un fotone di $\SI{1275}{keV}$. Nello spettro si vedranno anche i fotoni prodotti dall'annichilazione del positrone nella materia.
I principali modi di decadimento delle sorgenti di calibrazione a disposizione sono schematizzati in \autoref{tab:sorgenti_cal}.

\begin{table}[h]
	\centering
	\begin{tabular}{cccc}
		\toprule
		sorgenti & \multicolumn{3}{c}{principali modi di decadimento} \\
		\midrule
		\co & $\beta^{-} (\SI{318}{keV})$ & $\gamma (\SI{1173}{keV})$ & $\gamma (\SI{1332}{keV})$  \\
		\cs & $\beta^{-} (\SI{512}{keV})$ & $\gamma (\SI{662}{keV})$ \\
		\na & $\beta^{+} (\SI{546}{keV})$ & $\gamma (\SI{1275}{keV})$ \\
		\bottomrule
	\end{tabular}
	\caption{\label{tab:sorgenti_cal} Principali modi di decadimento delle sorgenti a disposizione \cite{2}.}
\end{table}

\end{comment}