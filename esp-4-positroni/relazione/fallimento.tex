\subsection{Misure con TDC}

Collegando le uscite discriminate dei PMT 1 e 2 ai due ingressi dell'unico TDC funzionante che abbiamo trovato in laboratorio, possiamo misurare i ritardi tra le risposte di due PMT posti uno di fronte all'altro sfruttando i fotoni dell'annichilazione. Per eseguire la misura usiamo come trigger le coincidenze e ritardiamo i segnali dei due PMT, dato che il trigger è successivo alla rivelazione dei fotoni. \`E stato scelto un ritardo di \SI{60}{ns} in modo che la differenza tra i tempi di arrivo $\Delta t=t_1-t_2$ possa essere sia positiva che negativa.

Prima di eseguire la misura calibriamo il TDC con il generatore di funzioni usando come trigger un'onda quadra e come \emph{stop} lo stesso segnale ma ritardato in modo arbitrario.
Eseguiamo questa calibrazione per le due scale disponibili: \SI{102}{ns} e \SI{510}{ns}.