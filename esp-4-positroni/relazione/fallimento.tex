\subsection{Misure con TDC}

\paragraph{Calibrazione}

Prima di eseguire la misura calibriamo il TDC con il generatore di funzioni usando come \emph{start} un'onda quadra e come \emph{stop} lo stesso segnale ma ritardato in modo arbitrario.
Eseguiamo questa calibrazione per i due fondiscala disponibili: \SI{102}{ns} e \SI{510}{ns}.
Le calibrazioni hanno mostrato una buona linearità,
anche se i fondiscala sono significativamente diversi da quelli nominali,
ma a volte il TDC smette di funzionare e richiede di riavviare il crate.

\paragraph{Risoluzione temporale}

Collegando le uscite discriminate dei PMT 1 e 2 ai due ingressi dell'unico TDC funzionante
che abbiamo trovato in laboratorio possiamo misurare i ritardi tra le risposte di due PMT posti uno di fronte all'altro sfruttando i fotoni dell'annichilazione. Per eseguire la misura usiamo come \emph{start} le coincidenze e ritardiamo i segnali dei due PMT, dato che lo start è successivo alla rivelazione dei fotoni. Questa procedura ci permette di misurare la differenza tra i tempi di arrivo $\Delta t=t_1-t_2$. Il ritardo introdotto fa in modo che la grandezza $\Delta t$ possa essere sia positiva che negativa. Abbiamo scelto un ritardo di \SI{60}{ns}.

% La distribuzione dei ritardi acquisita con due fondiscala diversi risulta identica (vedi \autoref{confronto}).
% Se nei due grafici ci fosse la stessa distribuzione, vedremmo la figura allargata di un fattore dato dal rapporto dei due fondiscala. Inoltre l'offset delle scale del TDC vale circa \SI{1}{digit}, quindi lo spostamento di circa \SI{50}{digit} del picco presente nei grafici non può essere spiegato da nessun effetto fisico.

In \autoref{confronto} riportiamo la misura fatta con i due fondiscala.
Usare due fondiscala serve per controllare che la distribuzione scali nel modo corretto
e quindi che non sia un artefatto del TDC.

\begin{figure}[h]
\centering
\subfloat
{\includegraphics[width=18 em]{immagini/100}
}
\subfloat
{\includegraphics[width=18 em]{immagini/500}
}

\caption{Misura della differenza di tempo dei segnali in coincidenza tra PMT 1 e PMT 2,
con entrambi i fondiscala dell'ADC.}
\label{confronto}
\end{figure}
