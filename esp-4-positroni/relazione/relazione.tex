\documentclass[a4paper]{article}

\usepackage[utf8]{inputenc}
\usepackage[T1]{fontenc}
\usepackage[italian]{babel}

\usepackage[margin=4.2cm, top=1.5cm, bottom=2.5cm]{geometry}

\usepackage{siunitx}
\usepackage{amsmath}
\usepackage{amssymb}
\usepackage{xfrac}
\usepackage{esint}
\usepackage[hidelinks]{hyperref}
\usepackage{graphicx}
\usepackage[font={sf}]{caption}
\usepackage{pdflscape}
\usepackage{makecell}
\usepackage{float}
\usepackage{subfig}
\usepackage{wasysym}
\usepackage{booktabs}

\setlength{\marginparwidth}{95pt}
\let\oldmarginpar\marginpar
\renewcommand\marginpar[1]{\oldmarginpar{\scriptsize\sffamily #1}}
\newcommand*\de{\mathrm{d}}
\newcommand*\pdv[2]{\frac{\partial #1}{\partial #2}}
\newcommand*\dv[2]{\frac{\de #1}{\de #2}}
\DeclareMathOperator\Ei{Ei}
\newcommand*\is{\equiv}
\newcommand\cs{$^{\text{137}}\text{Cs}$}
\newcommand\co{\ensuremath{^{\text{60}}\text{Co}}}
\newcommand\na{\ensuremath{^{\text{22}}\text{Na}}}
\newcommand\am{$^{\text{241}}\text{Am}$}
\newcommand\sr{$^{\text{90}}\text{Sr}$}

\sisetup{%
separate-uncertainty=true,
multi-part-units=single,
exponent-product=\cdot}

\frenchspacing

\title{Relazione di laboratorio:\\
Esperienza 4: Annichilazione del positrone}
\author{Andrea Marasciulli
\and Giacomo Petrillo
\and Roberto Ribatti}
\date{24 aprile -- 25 maggio 2018}

\begin{document}

\maketitle

\begin{abstract}

Misuriamo la massa dell'elettrone con una precisione maggiore dell'$1\permil$ facendo felice Punzi e (forse) il rate di cattura elettronica. Se siamo fortunati vediamo anche il positronio decadere in 3 fotoni.

\end{abstract}

{\tableofcontents}

\newpage
\section{Introduzione}

L'obiettivo principale dell'esperienza è la misura della massa del positrone sfruttando il decadimento $\beta^+$ del \na{}. Poi misuriamo l'efficienza dei rivelatori per stimare il rate di cattura elettronica rispetto al decadimento $\beta^+$. Infine cerchiamo di provare l'esistenza del decadimento a 3 fotoni del positronio.
\marginpar{Non è il positronio, è l'annichilazione in 3 fotoni.}

\section{Teoria}

\subsection{Spettro delle sorgenti a disposizione}

La nostra sorgente di \na{} ha 3 modi di decadimento:
\begin{itemize}
\item decadimento $\beta^+$ con $E_e=\SI{545}{keV}$, BR=90.4\%;
\item cattura elettronica (BR=9.5\% ) con la conseguente emissione di un fotone di energia \SI{1274}{keV} da parte del $^{22}$Ne formatosi;
\item decadimento $\beta^-$ con BR=0.1\%.
\end{itemize}

Useremo i decadimenti $\gamma$ del \cs{} (\SI{662}{keV}) e del \co{} (\SI{1173}{keV}, \SI{1332}{keV}) per calibrare il nostro apparato.

\marginpar{Scrivere perché non usiamo l'americio?}

\subsection{Fenomenologia dei rivelatori}

\marginpar{Scrivere materiale e dimensioni nella sezione ``apparato''}

Nel range di energie dei fotoni in cui lavoriamo, i processi di diffusione possibili sono gli scattering Rayleigh e Compton e l'effetto fotoelettrico.
Lo scattering Rayleigh, in quanto completamente elastico, non rilascia energia nel calorimetro.
Nella diffusione Compton il fotone cede una parte dell'energia ad un elettrone del nostro calorimetro e possiamo misurare l'energia persa da quest'ultimo. L'energia persa dal fotone segue la distribuzione di Klein-Nishina.
\marginpar{Klein-Nishina è la distribuzione angolare, non dell'energia.}
Se il fotone esegue un effetto fotoelettrico, perde tutta la sua energia cedendola ad un elettrone ed è l'unico modo che abbiamo per conoscere l'energia del fotone incidente.

A questo quadro si possono aggiungere processi di ordine superiore, come un numero maggiore di rimbalzi Compton all'interno del cristallo oppure un effetto fotoelettrico eseguito dopo uno scattering Compton.
Vedremo in seguito che un fotone, dopo aver eseguito un Compton all'interno di un rivelatore, può uscirne ed essere rivelato anche da un altro.

Riportiamo in \autoref{sezioni} un grafico che rappresenta la sezione d'urto dell'effetto Compton confrontata con quella del fotoelettrico all'interno del NaI. Per energie minori di \SI{200}{keV} l'effetto fotoelettrico domina sul Compton e viceversa per energie maggiori.
\marginpar{Dal grafico mi sembra più \SI{300}{keV}.}
Trascuriamo la produzione di coppie perché, quando è sopra soglia ($E_{\gamma}>\SI{1}{MeV}$), è 5 volte più piccola della sezione d'urto del fotoelettrico che, in questo range, è molto soppressa rispetto al Compton.
\marginpar{La caption di \autoref{fig:cross} è poco chiara, $\lambda$ è quello del grafico ma non c'è scritto da nessuna parte.}

\begin{figure}[h]
\centering
\includegraphics[width=25 em]{immagini/cross}
\caption{\label{fig:cross}
Sezioni d'urto in funzione dell'energia all'interno del nostro cristallo di NaI ($\rho=\SI{3.67}{g\,cm^{-3}})$ espressa come $\sigma=\frac{1}{\rho\lambda}$. Queste informazioni sono tratte da \cite{cross}.}
\label{sezioni}
\end{figure}


\section{Apparato}

\subsection{Apparato A}

% Doveva farlo Bob.

\subsubsection{Descrizione del circuito}

Il circuito del primo apparato costruito è rappresentato in \autoref{circuitone}.

\begin{figure}[h]
\centering
\includegraphics[width=30 em]{immagini/circuitone}
\caption{Schema circuitale dell'apparato A.}
\label{circuitone}
\end{figure}

Le uscite dei PMT sono inviate ai discriminatori, le cui uscite attraversano un modulo di antiretrigger (tempo morto di \SI{1}{\micro s}) e vengono usate per costruire funzioni logiche. I segnali dei PMT vengono anche inviati all'ADC attraverso un attenuatore per evitare di danneggiare lo strumento.
Per quanto riguarda la parte logica, costruiamo e mettiamo in tempo tre tipi di trigger:
\begin{enumerate}
\item canale
\item coincidenza a 2
\item coincidenza a 3.
\end{enumerate}
Il trigger di canale è semplicemente il segnale discriminato del PMT corrispondente; quello di coincidenze è dato dall'uscita di lunga durata del modulo stesso. Mandiamo questi segnali all'ADC in modo che il trigger di canale arrivi in contemporanea con il segnale analogico del canale stesso. Se in quel momento c'è anche un trigger di coincidenza, sappiamo se l'evento ci è arrivato da una coincidenza anziché da un canale singolo. Abbiamo fatto tutto questo per acquisire tutti i canali contemporaneamente ed agevolare la lettura dei dati via software. Purtroppo il \emph{cross-talk}%
\footnote{Spiegheremo come ci siamo accorti di questo problema nella \autoref{ref}.} presente nell'ADC ha rovinato le misure fatte con questo apparato.

\marginpar{``Influenzato'' non mi piace tanto.}

Per dare precedenza agli eventi più rari abbiamo collegato questi trigger ad un modulo \emph{or} ritardando i trigger singoli rispetto a quelli di coincidenze a 2 e questi ultimi rispetto alle coincidenze a 3. L'uscita dell'\emph{or} era poi inviata ad un \emph{timer} che generava il gate dell'ADC. L'altro canale di questo \emph{timer} veniva usato per avviare o fermare le acquisizioni.

\subsubsection{Problemi riscontrati}

I problemi descritti in seguito ci hanno spinti a semplificare il circuito per cercare di risolverli.

Il più evidente fin da subito è stato il \emph{bit stuck}: il terzo bit dei dati in formato binario è sempre 1. Gli istogrammi presentano delle lacune di \SI{4}{digit} ogni \SI{4}{digit}. Risolviamo questo problema istogrammando con canali più larghi di \SI{8}{digit}.

Poi abbiamo escluso l'\emph{or} dal circuito perché i segnali che ci arrivavano dalle coincidenze mostravano avere un'energia minore. Abbiamo risolto il problema togliendo l'\emph{or} dal circuito.

Il problema più interessante è dato dal fatto che, confrontando gli spettri del \na{}, abbiamo visto che i fotoni emessi dalla sorgente con attività maggiore avevano energia maggiore.
Il grafico in \autoref{distanze} mostra l'energia dei fotopicchi del sodio in funzione del rate misurato ponendo la sorgente ad attività elevata a varie distanze dal rivelatore.

\marginpar{Sto supponendo che prima di questo punto abbiamo scritto che ci sono due sorgenti di sodio. Ricordarsi di aggiungere il nadebole e la tabella dei dati.}

\begin{figure}[h]
\centering
\includegraphics[width=25 em]{immagini/naforte}
\caption{Posizione dei fotopicchi del sodio in funzione del rate di eventi ponendo la sorgente a diverse distanze dal rivelatore. In alcune delle acquisizioni non era visibile il picco del neon oppure era talmente deformato da rendere privo di senso il fit gaussiano.}
\label{distanze}
\end{figure}

Pensiamo che tale effetto sia dovuto ad un pile-up nella digitalizzazione in quanto l'ADC ha un tempo di conversione di \SI{60}{\micro s}. Dal grafico si evince che anche con un rate di eventi minore di \SI{10}{kHz} il problema persiste. La spiegazione è data dal fatto che la distribuzione temporale tra un evento e l'altro è esponenziale, quindi tra due eventi consecutivi intercorre spesso un piccolo intervallo di tempo. 
Abbiamo quindi deciso di utilizzare la sorgente ad attività minore per misurare la massa dell'elettrone e quella ad attività maggiore per le misure in cui non ci interessa l'energia degli eventi o cerchiamo eventi rari.

\marginpar{Qualcuno che conosce la distribuzione esponenziale è pregato di controllare la correttezza di questa frase.}

\subsection{Circuito B}

Dopo aver eseguito in maniera preliminare le misure richieste, abbiamo deciso di semplificare il circuito per tenere sotto controllo le criticità descritte nella sezione precedente.

\subsubsection{Descrizione del circuito}

Il nuovo circuito segue lo schema in \autoref{circuitone} in basso.
I segnali dei PMT vengono inviati ai discriminatori per ottenere i segnali logici e agli attenuatori per essere poi inviati all'ADC. I segnali logici attraversano un modulo di antiretrigger per poi essere usati come trigger di acquisizioni singole o per effettuare coincidenze. Una copia dei segnali finisce al contatore. Abbiamo usato il timer per far partire/fermare le acquisizioni come abbiamo fatto per il circuito precedente.

\subsubsection{Problemi riscontrati}
\label{ref}
Dopo aver acceso l'apparato il bit stuck è scomparso
(probabilmente era dovuto a un mancato contatto nel modulo \texttt{CAMAC})
e l'aumento di risoluzione ci ha permesso di notare un problema che era già presente nel circuito precedente.
Abbiamo iniziato a vedere uno sdoppiamento dello spettro, come evidenziato dalla \autoref{doppio} sinistra. 

\begin{figure}
\centering
\subfloat
{
\includegraphics[width=18 em]{immagini/doppio}
}
\subfloat
{
\includegraphics[width=18 em]{immagini/sdoppio}
}
\caption{A sinistra: acquisizione in cui l'annichilazione presenta due picchi.
A destra: divisione degli stessi dati in coincidenze (``c2'') e anticoincidenze (``not c2'').}
\label{doppio}
\end{figure}

Abbiamo poi scoperto la causa del problema, ovvero il crosstalk nell'ADC.
Come mostra il grafico di \autoref{doppio} destra, il doppio picco è la somma di quello osservato negli eventi in coincidenza con quello osservato in anticoincidenza,
ovvero in cui scatta il trigger del rivelatore 1 (quello su cui triggeriamo per salvare i dati) ma non quello del rivelatore 2.



\begin{figure}
\centering
\subfloat
{
\includegraphics[width=20 em]{immagini/dist}
}
\caption{Grafico in cui il PMT1 è vicino alla sorgente mentre il PMT2 è lontano dalla sorgente.}
\label{sdoppio}
\end{figure}

Siamo arrivati alla conclusione che si tratti di un crosstalk dal fatto che il picco in anticoincidenza coincide con quello di un singolo canale quando tutti gli altri sono scollegati.
Inoltre abbiamo visto, aumentando la distanza di un PMT dalla sorgente, che il picco delle coincidenze si rimpicciolisce (perché ha meno eventi) e il picco in anticoincidenza tende ad assomigliare sempre di più a quello dell'acquisizione con canale singolo (\autoref{sdoppio});
questo permette di convincersi della causa anche senza separare gli eventi come in \autoref{doppio}.

Tale problema non era visibile ad occhio nudo nelle precedenti acquisizioni, ma un fit eseguito in seguito su misure ad alta statistica ha mostrato che il picco era modellato bene da due gaussiane ma non da una.

In seguito abbiamo provato che questo problema si riscontra anche quando i canali da analizzare non sono collegati ad ingressi adiacenti dell'ADC. Inspiegabilmente, il problema si è ripresentato solo a giorni alterni. 


\section{Misura e analisi}

% Misure preliminari

\subsection{Punto di lavoro}

Sappiamo dalla precedente esperienza sull'effetto Compton che i PMT con scintillatore di NaI(Tl) lavorano bene in un range di tensioni che va da \SI{600}V a \SI{800}V e l'unica differenza tra una tensione e l'altra è una variazione della scala.
Allora scegliamo delle tensioni che seguano le caratteristiche dichiarate dal tecnico di laboratorio facendo in modo che, in presenza della sorgente, si abbiano dei rate simili tra i 3 PMT.
Allora mettiamo il PMT1 a \SI{703}V, il PMT2 a \SI{839}V ed il PMT3 a \SI{930}V.

Mettiamo le soglie dei discriminatori al minimo (circa \SI{-50}{mV}) e usiamo degli attenuatori per cambiare la scala delle misure in caso di necessità.
\marginpar{Dire che non abbiamo rumore casuale,
perché sarebbe il principale motivo per scegliere meglio soglie
e alimentazioni.}

\subsection{Gate dell'ADC}

Variamo la durata del gate dell'ADC e calcoliamo il rapporto $\sigma\!/$\!media per il picco di annichilazione rivelato dal PMT1, modellato come al solito con una gaussiana.
Su alcune misure abbiamo introdotto dei ritardi per migliorare o peggiorare la sincronizzazione con il gate al fine di quantificare quanto la messa in tempo influisca sulla nostra risoluzione. Il risultato di questa misura è in \autoref{fig:gate}, i dati sono in \autoref{tab:gate}.

\begin{figure}[h]
\centering
\includegraphics[width=20 em]{immagini/gate}
\caption{Grafico della risoluzione in funzione della durata del gate. Le misure sovrapposte sono state ottenute ritardando il segnale del PMT1.}
\label{fig:gate}
\end{figure}

\begin{table}[h]
\centering
\begin{tabular}{c|c|c|c}
durata (ritardo) [ns] & media [digit] & $\sigma$ [digit] & $\sigma\!/$\!media\\
\hline
 100 & 304.88$\,\pm\,$34 &  8.67$\,\pm\,$31 & 0.0284 $\,\pm\,$10 \\  
 190 & 492.82$\,\pm\,$50 & 13.59$\,\pm\,$65 & 0.0276 $\,\pm\,$13 \\
 190 & 513.05$\,\pm\,$80 & 14.1 $\,\pm\,$1.0 & 0.0274$\,\pm\,$20 \\
 300 & 307.68$\,\pm\,$36 & 12.81$\,\pm\,$59 & 0.0416 $\,\pm\,$19 \\
 400 & 344.25$\,\pm\,$25 & 14.10$\,\pm\,$34 & 0.0410 $\,\pm\,$10 \\
 500 & 524.01$\,\pm\,$42 & 15.16$\,\pm\,$64 & 0.0289 $\,\pm\,$12 \\
 700 & 482.54$\,\pm\,$85 & 15.5 $\,\pm\,$1.4 & 0.0321$\,\pm\,$28 \\
1000 & 539.43$\,\pm\,$61 & 16.0 $\,\pm\,$1.1 & 0.0297$\,\pm\,$20 \\
1000 & 553.88$\,\pm\,$71 & 15.9 $\,\pm\,$1.2 & 0.0288$\,\pm\,$22 \\
1000 & 542.68$\,\pm\,$41 & 15.75$\,\pm\,$69 & 0.0290 $\,\pm\,$13 
\end{tabular}

\caption{Andamento della risoluzione in funzione della durata del gate. I numeri tra parentesi indicano di quanto è stato ritardato il PMT1.}
\label{tab:gate}
\end{table}

Dall'analisi dei dati abbiamo potuto vedere che la risoluzione migliore si raggiunge quando il gate dura meno di \SI{200}{ns}
\marginpar{Con le incertezze riportate, mi sembra che, fatta eccezione per 300 e 400, le misure siano tutte compatibili.}
ma questo valore non è un buon punto di lavoro perché cambiare la durata del gate cambia la scala e, in questo caso, non ci permetteva di vedere il picco del neon.
\marginpar{Il motivo non è il picco del neon, perché accorciare il gate è come attenuare e quindi a maggior ragione si vedono i picchi più a destra. Inoltre usando gli attenuatori si può sempre compensare la variazione del gate.}
L'alternativa migliore sembrerebbe essere una durata del gate di \SI{1}{\micro s}, ma il manuale dell'ADC specifica che a tale valore (o anche maggiore) diminuisce la linearità della digitalizzazione. Per questo abbiamo scelto di usare un gate di \SI{550}{ns} nell'apparato A.
\marginpar{Il manuale dice che aumenta l'errore, non che peggiora la linearità (quella era per segnali sopra \SI1V). Inoltre come limite dà \SI{200}{ns}.}
Questa scelta è poi diventata \SI{1}{\micro s} nell'apparato B per evitare problemi derivanti dal jitter dei segnali. Infine notiamo che cambiare i ritardi sui segnali in ingresso non influisce significativamente sulla risoluzione dell'apparato.


\subsection{Stabilità}

Mostriamo l'instabilità del nostro sistema di rivelatori nei 2 apparati sopra descritti monitorando la posizione dei fotopicchi del sodio in funzione del tempo.
La prima misura è stata fatta con l'apparato A il 3 maggio e dura \SI{16}{ore}, la seconda  è stata eseguita con l'apparato B il 15 maggio e dura \SI{63}{ore}.
Nella prima misura sono presenti tutti i canali, nella seconda è presente solo il canale 1 per evitare il cross-talk riscontrato nella precedente.
Abbiamo usato l'energia nominale dei 2 picchi per ricavare una ``retta di calibrazione'' di pendenza $m$ e intercetta $q$ e abbiamo osservato l'evoluzione temporale di questi parametri. Usiamo l'espressione tra virgolette perché lo scopo dell'esperienza è fingere di non conoscere la massa dell'elettrone per poi misurarla.

% misura 1
\begin{figure}[h]
\centering
\includegraphics[width=\textwidth]{immagini/0503_picchi}
\caption{Misura di stabilità iniziata il 3 maggio alle 19. I grafici rappresentano la posizione dei picchi in funzione del tempo; ``beta'' indica il picco di annichilazione e ``neon'' indica il fotone emesso dal decadimento del neon.}
\label{picchi1}
\end{figure}

\begin{figure}[h]
\centering
\includegraphics[width=\textwidth]{immagini/0503_rette}
\caption{Misura di stabilità iniziata il 3 maggio alle 19. I grafici rappresentano il valore della pendenza e dell'intercetta della ``retta di calibrazione'' in funzione del tempo.}
\label{rette1}
\end{figure}

% misura 2
\begin{figure}[h]
\centering
\subfloat
{
\includegraphics[width=0.49\textwidth]{immagini/0515_picchi}
}
\subfloat
{
\includegraphics[width=0.49\textwidth]{immagini/0515_rette}
}
\caption{Misura di stabilità iniziata il 15 maggio alle 19.\\
A sinistra:  grafici che rappresentano la posizione dei picchi in funzione del tempo; ``beta'' indica il picco di annichilazione e ``neon'' indica il fotone emesso dal decadimento del neon. \\
A destra: grafici che rappresentano il valore della pendenza e dell'intercetta della ``retta di calibrazione'' in funzione del tempo.}
\label{picchi2}
\end{figure}

% discussione dei dati
Guardando i dati della prima misura (\autoref{picchi1} e \autoref{rette1}) vediamo che i vari rivelatori si scalibrano in modo diverso. L'andamento dei punti indica la notte come momento di massima stabilità e l'apertura del laboratorio come momento di massima instabilità.

Dai dati della seconda misura (\autoref{picchi2} sinistra e \autoref{picchi2} destra) si nota uno spostamento coerente dei fotopicchi del sodio che si stabilizza durante la notte. Anche qui si nota una scalibrazione significativa durante gli orari di apertura del laboratorio.
Le stesse considerazioni valgono anche per i parametri della retta di calibrazione.

\subsection{Rimbalzi}

Guardando l'istogramma 2D delle misure in coincidenza, abbiamo notato dei comportamenti non attesi che abbiamo supposto e poi verificato essere dei fotoni che rimbalzano da un rivelatore all'altro.

La \autoref{scatter} mostra un tipico istogramma 2D
nella configurazione in cui 2 rivelatori sono posti uno di fronte all'altro alla stessa distanza da una sorgente di \na{}.

\begin{figure}[h]
\centering
\includegraphics[width=\textwidth]{immagini/esempio}
\caption{Tipico istogramma 2D di una misura in coincidenza. La descrizione di questo grafico è presente nel testo.}
\label{scatter}
\end{figure}

Si vedono degli eccessi in alcuni punti del grafico che chiameremo in seguito \emph{strutture}. La più vistosa si manifesta quando entrambi i fotoni di annichilazione effettuano un processo fotoelettrico all'interno degli scintillatori. In basso a sinistra si nota invece la zona in cui entrambi hanno subito una diffusione Compton. Le bande arancioni intorno a questa zona rappresentano invece gli eventi in cui un fotone proveniente dall'annichilazione ha fatto fotoelettrico su un rivelatore e l'altro ha fatto Compton sull'altro.
La stessa cosa avviene con il fotone proveniente dal decadimento del neon, rappresentato dalle strutture nella parte centrale del grafico adiacente ai bordi della figura. La struttura più a destra (o più in alto) rappresenta l'arrivo simultaneo di un fotone di annichilazione insieme ad uno del neon. L'interpretazione degli eventi è analoga a quelli descritti precedentemente.
Le strutture non attese sono i due eccessi presenti nella zona in cui il fotone del neon e quello dell'annichilazione fanno entrambi scattering Compton. Queste strutture si manifestano a valori di energia coincidenti ai picchetti della spalla Compton.
\marginpar{``Picchetti della spalla Compton'', come aggiustarli?}

Per verificare la nostra ipotesi abbiamo messo i rivelatori nella configurazione di \autoref{spostati} sinistra, in modo da poterci aggiungere un mattone piombo come in \autoref{spostati} destra.
L'istogramma 2D corrispondente si trova in \autoref{spostato} sinistra: la struttura più popolata non è più data dalla rivelazione simultanea dell'annichilazione per effetto fotoelettrico, ma dalla somma degli eventi nelle bande che si incrociano.
In questa configurazione non si nota più uno degli eccessi inattesi nominati prima: quello ad energia minore.
Adesso sono diventate evidenti due strutture circolari nella zona in basso a sinistra del grafico: esse e l'ultimo tipo di struttura rimasta scompaiono completamente quando effettuiamo la misura nella configurazione mostrata in \autoref{spostati} destra, come mostrato in \autoref{spostato} destra.

Alla luce di quanto osservato e analizzando i grafici, si trova che le strutture sopra descritte si presentano a coppie e la somma delle loro energie vale rispettivamente $m_e$, $2m_e$ ed $E_{\gamma}=\SI{1274}{keV}$.
Tali valori sono giustificati dal fatto che, in alcuni casi, un fotone rimbalza sul primo scintillatore su cui è arrivato e viene assorbito dall'altro. Se questo ha già assorbito l'altro fotone (possibile solo nell'annichilazione) la somma delle energie è proprio il doppio dell'energia del fotone singolo.

\begin{figure}[h]
\centering
\subfloat
{\includegraphics[width=0.49\textwidth]{immagini/alter}}
\hfill
\subfloat
{\includegraphics[width=0.49\textwidth]{immagini/spostati2}}
\caption{
A sinistra:
rivelatori posti uno di fronte all'altro senza schermatura.
La sorgente è nascosta nella casella L1 della scacchiera.
Il righello arancione serve per controllare che i due fotoni contrapposti di annichilazione
non possano raggiungere entrambi gli scintillatori.\\
A destra: aggiunta della schermatura di piombo per sopprimere i rimbalzi.}
\label{spostati}
\end{figure}

\begin{figure}[h]
\centering
\hspace{-2.5 cm}
\subfloat
{
\includegraphics[width=21 em]{immagini/0518_rimbalzi}
}
\subfloat
{
\includegraphics[width=21 em]{immagini/0518_piombo}
}
\caption{A sinistra: misura eseguita nella configurazione di \autoref{spostati} sinistra. \\
A destra: misura eseguita nella configurazione di \autoref{spostati} destra.  \\
Le fasce densamente popolate adiacenti ai bordi della figura sono date dagli eventi in cui un rivelatore registra un evento mentre l'altro acquisisce uno zero.
Una descrizione più dettagliata di questi grafici è presente nel testo.}
\label{spostato}
\end{figure}

\subsubsection{Misura con un solo fotone}

Abbiamo effettuato la misura nella configurazione di \autoref{solo} sinistra e poi \autoref{solo} destra usando la sorgente di \cs{}: lo scopo della misura è evidenziare l'importanza dei rimbalzi tra scintillatori vicini.

\begin{figure}[h]
\centering
\subfloat
{\includegraphics[width=0.49\textwidth]{immagini/rimb.jpg}}
\hfill
\subfloat
{\includegraphics[width=0.49\textwidth]{immagini/norimb}}
\caption{\label{solo}
A sinistra:
configurazione usata per evidenziare la presenza di rimbalzi.
La sorgente è nel cilindretto blu.\\
A destra:
aggiunta del piombo che blocca i rimbalzi.}
\end{figure}

\begin{figure}[h]
\centering
\includegraphics[width=\textwidth]{immagini/rimb}
\caption{\label{ce}
Spettro delle coincidenze tra i due scintillatori.
A sinistra: misura eseguita nella configurazione di \autoref{solo} sinistra.\\
A~destra: misura eseguita nella configurazione di \autoref{solo} destra.}
\end{figure}

L'istogramma 2D corrispondente alla \autoref{solo} sinistra è in \autoref{ce} sinistra: notiamo due accumuli proprio dove si trovavano in \autoref{spostato} sinistra. Anche questi scompaiono dopo aver aggiunto il piombo (\autoref{ce} destra). Gli eventi misurati non possono che essere coincidenze casuali.
\marginpar{Limare la fine della frase.
\\\emph{Non possono essere fotoni che passano il piombo? O che fanno doppio rimbalzo?}}



\subsection{Misure con TDC}

Collegando le uscite discriminate dei PMT 1 e 2 ai due ingressi dell'unico TDC funzionante%
\footnote{All'apparenza.}%
\,che abbiamo trovato in laboratorio, possiamo misurare i ritardi tra le risposte di due PMT posti uno di fronte all'altro sfruttando i fotoni dell'annichilazione. Per eseguire la misura usiamo come \emph{start} le coincidenze e ritardiamo i segnali dei due PMT, dato che lo start è successivo alla rivelazione dei fotoni. \`E stato scelto un ritardo di \SI{60}{ns} in modo che la differenza tra i tempi di arrivo $\Delta t=t_1-t_2$ possa essere sia positiva che negativa.

\marginpar{L'ultima frase non torna, sembra come se il fatto che sia \SI{60}{ns} sia particolarmente utile.
Che sia positiva che negativa dipendeva dal fatto che li avessimo ritardati.
Comunque la spiegazione è poco chiara per uno che non ha fatto le misure.}

Prima di eseguire la misura calibriamo il TDC con il generatore di funzioni usando come start un'onda quadra e come \emph{stop} lo stesso segnale ma ritardato in modo arbitrario.
Eseguiamo questa calibrazione per i due fondiscala disponibili: \SI{102}{ns} e \SI{510}{ns}.
Le calibrazioni hanno mostrato una buona linearità, ma la misura dei ritardi ha mostrato che il TDC, per un motivo a noi ignoto, non funziona.
\marginpar{In realtà a suo arbitrio può decidere di non funzionare anche quando lo testiamo con il generatore di segnali.}
Il grafico di \autoref{confronto} sinistra è uguale a quello di \autoref{confronto} destra nonostante le acquisizioni siano state fatte con due fondiscala diversi. Se nei due grafici ci fosse la stessa cosa, vedremmo la figura allargata di un fattore dato dal rapporto dei due fondiscala. Inoltre l'offset delle scale del TDC vale circa \SI{1}{digit}, quindi lo spostamento di circa \SI{50}{digit} del picco presente nei grafici non può essere spiegato da nessun effetto fisico.

\begin{figure}[h]
\centering
\subfloat
{\includegraphics[width=18 em]{immagini/100}
}
\subfloat
{\includegraphics[width=18 em]{immagini/500}
}

\caption{Misura di ritardo vista da entrambe le scale del TDC.}
\label{confronto}
\end{figure}

Questo problema ci impedisce di fare i facoltativi.

% Misure importanti

\subsection{Massa dell'elettrone}

Per ricavare la massa dell'elettrone misuriamo l'energia del picco di annichilazione del \na{}
calibrando la scala di energia con i fotopicchi di \co{} e \cs{}.
Poiché la scala di energia varia significativamente nell'arco di tempo in cui facciamo le misure,
misuriamo contemporaneamente lo spettro di tutte le sorgenti.
Triggeriamo su un singolo PMT,
collegando solo quel PMT all'ADC, per evitare il crosstalk tra i canali dell'ADC.
Ripetiamo la misura con i 3 rivelatori disponibili.
Questa misura è stata fatta con il circuito B.

\subsubsection{Fit dei picchi}

\begin{figure}
	\includegraphics[width=\textwidth]{immagini/mass18-peaks}
	\caption{\label{fig:mass18-peaks}
	Spettri e fit dei picchi delle sorgenti \na{}, \cs{} e \co{} con i tre PMT,
	usati per ricavare la massa dell'elettrone,
	ovvero l'energia del picco di annichilazione $\na_\beta$.
	Nel fit dei picchi $\co_{1.17}$ e $\co_{1.33}$ c'è una terza gaussiana
	perché tra i due c'è il picco $\na_\gamma$ a \SI{1.28}{MeV},
	ma ha un rate parecchio minore quindi lo fittiamo come fondo ma non lo usiamo per ricavare la massa.
	Il picco $\na_{\beta+\gamma}$ è dovuto al caso in cui lo scintillatore
	assorbe sia il fotone dell'annichilazione sia quello del neon;
	analogamente $\co_{1.17+1.33}$ è il caso in cui vengono assorbiti entrambi i fotoni del cobalto.}
\end{figure}

Fittiamo ogni picco, scegliendo a mano l'intervallo di canali su cui fittare,
con una gaussiana più un esponenziale come fondo.
Il fit è ai minimi quadrati sull'istogramma.
I bin fittati contengono tutti almeno 5 eventi.
Controlliamo che cambiare il ribinnaggio dei canali dell'ADC non cambia significativamente il risultato.
I~fit sono riportati in \autoref{fig:mass18-peaks}.
Tutti i fit hanno un p-value ragionevole;
il test di Kolmogorov-Smirnov sull'uniformità dei p-value dà un p-value del \SI{18}\%.
Per i picchi del cobalto e quello del neon, che sono sovrapposti, il fit è unico.
Con dei test vediamo che il risultato per la media del picco del neon,
che è praticamente nascosto da quelli del cobalto,
è instabile, quindi lo teniamo nel fit come fondo ma non lo usiamo per ricavare la massa.

\subsubsection{Fit della massa}

\begin{figure}
	\centering
	\includegraphics[width=25em]{immagini/mass18-cal}
	\caption{\label{fig:mass18-cal}
	Fit per ricavare la massa dell'elettrone.
	Sulle ordinate sono riportate le medie dei picchi di \autoref{fig:mass18-peaks}.
	Sulle ascisse sono riportati i valori noti delle energie
	per $\cs$, $\co_{1.17}$, $\co_{1.33}$, $\co_{1.17+1.33}$ (pallini),
	mentre per $\na_\beta$ e $\na_{\beta+\gamma}$ (croci) sono riportati i valori fittati.
	Le incertezze non sono visibili a questa scala.}
\end{figure}

\begin{table}
	\hspace{-3em}
	\begin{tabular}{c|cccc|cc}
		PMT & $a$ & $b$ [\si{keV^{-1}}] & $c$ [\si{keV^{-2}}] & $m$ [\si{keV}] & $\chi^2$/dof & $F$ \\
		\hline
		1 &   \num{7.31(54)} &  \num{0.3584(11)} &   \num{2.60(26)} & \num{519.95(50)} & 33.3 & 7.0 \\
		2 & \num{228.02(31)} & \num{0.22802(56)} &  \num{-3.55(11)} & \num{516.44(55)} & 24.4 & 6.2 \\
		3 & \num{113.47(44)} & \num{0.46701(84)} & \num{-32.94(18)} & \num{495.25(44)} &  4.1 & 2.2
	\end{tabular}
	\caption{\label{tab:massfit}
	Risultati dei fit di calibrazione e massa dell'elettrone per ogni PMT.
	La curva di calibrazione è $E_\text{adc}=2cE^2+bE+a$.
	In tutti i fit le correlazioni risultano circa \SI{50}\% tra $m$ e gli altri parametri
	e circa \SI{95}\% tra $a$, $b$ e $c$.
	$F$ è il fattore per cui riscalare l'incertezza su $m$
	ricavato testando il fit sui picchi di calibrazione.}
\end{table}

Inizialmente fittiamo le medie dei picchi in funzione dell'energia con una retta
\begin{equation}
	\label{eq:retta}
	E_\text{adc} = b \cdot E + a,
\end{equation}
dove, a seconda dei casi, $E$ è
\begin{itemize}
	\item il valore noto dell'energia per i picchi di calibrazione;
	\item la massa $m$ dell'elettrone (parametro di fit) per il picco di annichilazione;
	\item $m$ più l'energia del fotone del neon per il picco $\na_{\beta+\gamma}$.
\end{itemize}
Nel fit teniamo conto della correlazione tra i picchi del cobalto.
Il p-value del fit è praticamente 0,
ovvero le incertezze statistiche sono sufficientemente piccole da rigettare il modello \eqref{eq:retta}.

Poiché abbiamo ancora 3 gradi di libertà,
rendiamo il modello più generico aggiungendo il termine quadratico:
\begin{equation}
	\label{eq:parabola}
	E_\text{adc} = 2c \cdot E^2 + b \cdot E + a.
\end{equation}
Anche questo modello viene rigettato con forza per i PMT 1 e 2, ma non per il 3.
Non vogliamo avere meno di due gradi di libertà nel fit,
quindi in mancanza di un modello adeguato, stimiamo un'incertezza sistematica aggiuntiva in questo modo:
per ogni picco di calibrazione ripetiamo il fit lasciando libera,
oltre all'energia del picco di annichilazione,
anche l'energia del picco di calibrazione scelto.
Per ognuno di questi fit calcoliamo il rapporto tra
la differenza tra l'energia fittata e l'energia nota del picco di calibrazione
e la deviazione standard stimata dal fit sull'energia fittata.
Riscaliamo l'incertezza sulla massa del fit principale
per la media quadratica di questi rapporti.
Nel caso dei due picchi del cobalto $\co_{1.17}$ e $\co_{1.33}$,
per il picco $\co_{1.17+1.33}$ scriviamo $E$ come somma di un parametro di fit e di un valore noto
allo stesso modo del picco $\na_{\beta+\gamma}$.
Il grafico del fit è riportato in \autoref{fig:mass18-cal},
i risultati dettagliati sono riportati in \autoref{tab:massfit}.
Le masse fittate, con l'incertezza riscalata, risultano
\begin{center}
	\begin{tabular}{cc}
		PMT 1: & $m=\SI{520.0(35)}{keV}$, \\
		PMT 2: & $m=\SI{516.4(34)}{keV}$, \\
		PMT 3: & $m=\SI{495.2(10)}{keV}$.
	\end{tabular}
\end{center}
I risultati non sono complessivamente compatibili tra loro,
allora calcoliamo media e deviazione standard dei tre valori,
trascurando le incertezze riportate perché sono abbastanza minori della discrepanza.
Risulta $m=\SI{511(13)}{keV}$.
% Come si vede chiaramente in \autoref{fig:mass18-cal}
% il PMT 3 è molto meno lineare di 1 e 2,
% però avendo stimato l'incertezza testando il fit sui picchi di calibrazione
% non è giustificato escluderlo.

\paragraph{Instabilità della calibrazione}

Pur avendo usato tutte le sorgenti contemporaneamente,
controlliamo se l'instabilità della scala di energia ha un effetto sulla misura.
Dividendo a metà l'acquisizione e analizzando separatamente le due parti
si osserva uno spostamento visibile dei picchi ma minore della loro deviazione standard.
I due risultati per la massa, separati per rivelatore,
sono compatibili all'incertezza statistica
(dell'ordine di quella riportata in \autoref{tab:massfit}),
quindi concludiamo che l'effetto è trascurabile.


\begin{thebibliography}{99}

\bibitem[1]{cross}
M.J. Berger, J.H. Hubbell, S.M. Seltzer, J. Chang, J.S. Coursey, R. Sukumar, D.S. Zucker, and K. Olsen, XCOM: Photon Cross Sections Database, \emph{National Institute of Standards and Technology} (2010).

\bibitem[2] {2}
S.Y.F. Chu, L.P. Ekström, R.B. Firestone, The Lund/LBNL Nuclear Data Search Version 2.0, \emph{Berkeley National Laboratory} (1999).

\bibitem[3]{3}
G.F. Knoll, Radiation Detection and Measurement - 3rd edition (Capitolo 10), \emph{Wiley} (1999).

\bibitem[4]{4}
S. C. Pevovar, M. H. Weber, and K. G. Lynn: Ratio of positron annihilation into three photons versus two,
\emph{phys. stat. sol. (c) 4}, No. 10, 3447 – 3450 (2007) / DOI 10.1002/pssc.200675786

\end{thebibliography}

\appendix


\end{document}