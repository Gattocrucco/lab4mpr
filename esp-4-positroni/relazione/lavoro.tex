\subsection{Punto di lavoro}

% Sappiamo dalla precedente esperienza sull'effetto Compton che i PMT con scintillatore di NaI(Tl) lavorano bene in un range di tensioni che va da \SI{600}V a \SI{800}V e l'unica differenza tra una tensione e l'altra è una variazione della scala.
% la frase non ha senso perché la tensione c'entra con il PMT e non con lo scintillatore.
Per i PMT scegliamo delle tensioni che seguano le caratteristiche dichiarate dal tecnico di laboratorio facendo in modo che, in presenza della sorgente, si abbiano dei rate simili tra i 3 PMT.
Allora mettiamo il PMT1 a \SI{703}V, il PMT2 a \SI{839}V ed il PMT3 a \SI{930}V.

Impostiamo le soglie dei discriminatori al minimo (circa \SI{-50}{mV}) perché il rumore casuale è
% da dove esce fuori 1 %??
%al massimo l'\SI{1}{\%} degli eventi che misuriamo usando le sorgenti meno attive
trascurabile.
% gac
%L'utilizzo delle coincidenze rende questo rumore molto più raro.

% l'abbiamo già detto nel circuito
%Usiamo degli attenuatori per cambiare la scala delle misure in caso di necessità.
