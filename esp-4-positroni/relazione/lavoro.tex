\subsection{Punto di lavoro}

Sappiamo dalla precedente esperienza sull'effetto Compton che i PMT con scintillatore di NaI(Tl) lavorano bene in un range di tensioni che va da \SI{600}V a \SI{800}V e l'unica differenza tra una tensione e l'altra è una variazione della scala.
Allora scegliamo delle tensioni che seguano le caratteristiche dichiarate dal tecnico di laboratorio facendo in modo che, in presenza della sorgente, si abbiano dei rate simili tra i 3 PMT.
Allora mettiamo il PMT1 a \SI{703}V, il PMT2 a \SI{839}V ed il PMT3 a \SI{930}V.

Mettiamo le soglie dei discriminatori al minimo (circa \SI{-50}{mV}) e usiamo degli attenuatori per cambiare la scala delle misure in caso di necessità.