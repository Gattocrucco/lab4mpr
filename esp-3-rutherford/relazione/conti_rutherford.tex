\appendix
\section{Conti}
\label{sec:conti}

Calcoliamo la sezione d'urto Rutherford nel caso di nucleo non fisso.
La formula con nucleo fisso è
\begin{equation}
	\label{eq:ruthfisso}
	\dv\sigma\Omega = \left( \frac {zZ\alpha\hbar c} {2T(1-\cos\theta)} \right)^2.
\end{equation}
Il problema con il nucleo non fisso si rinconduce a quello fisso,
che chiamiamo <<problema ridotto>>, con queste sostituzioni:
\begin{center}
	\begin{tabular}{ll}
		Nucleo non fisso                 & Nucleo fisso               \\
		\hline
		Distanza tra particella e nucleo & Posizione della particella \\
		Massa della particella           & Massa ridotta              
	\end{tabular}
\end{center}
La \eqref{eq:ruthfisso} è da intendersi scritta nelle variabili del problema ridotto,
che dobbiamo esprimere in termini di quelle non ridotte.
Sia $\vec v$ la velocità della particella $\alpha$ (di massa $m$),
sia $\vec V$ la velocità del nucleo (di massa $M$), sia
\begin{equation*}
	\mu = \frac{mM}{m + M}
\end{equation*}
la massa ridotta.
Indichiamo con il pedice ``rid'' le variabili del problema ridotto.
La velocità ridotta è
\begin{equation*}
	\vec v_\text{rid} = \dv{}t(\vec r_\alpha - \vec r_\text{nucl}) = \vec v - \vec V.
\end{equation*}
