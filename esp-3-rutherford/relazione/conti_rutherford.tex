\appendix
\section{Conti}
\label{sec:conti}

Calcoliamo la sezione d'urto Rutherford nel caso di nucleo non fisso.
La formula con nucleo fisso è
\begin{equation}
	\label{eq:ruthfisso}
	\dv\sigma\Omega = \left( \frac {zZ\alpha\hbar c} {2T(1-\cos\theta)} \right)^2.
\end{equation}
Il problema con il nucleo non fisso si rinconduce a quello fisso,
che chiamiamo <<problema ridotto>>, con queste sostituzioni:
\begin{center}
	\begin{tabular}{ll}
		Nucleo non fisso                 & Nucleo fisso               \\
		\hline
		Distanza tra particella e nucleo & Posizione della particella \\
		Massa della particella           & Massa ridotta              
	\end{tabular}
\end{center}
La \eqref{eq:ruthfisso} è da intendersi scritta nelle variabili del problema ridotto,
che dobbiamo esprimere in termini di quelle non ridotte.
Sia $\vec v$ la velocità della particella $\alpha$ (di massa $m$),
sia $\vec V$ la velocità del nucleo (di massa $M$), sia
\begin{equation*}
	\mu = \frac{mM}{m + M}
\end{equation*}
la massa ridotta.
Indichiamo con il pedice ``rid'' le variabili del problema ridotto.
La velocità ridotta è
\begin{equation*}
	\vec v_\text{rid} = \dv{}t(\vec r_\alpha - \vec r_\text{nucl}) = \vec v - \vec V.
\end{equation*}
Per conservazione dell'energia nel problema ridotto
$|\vec v_\text{rid,in}| = |\vec v_\text{rid,out}|$,
inoltre
$\vec v_\text{rid,in}
= \vec v_\text{in} - \vec V_\text{in}
= \vec v_\text{in}$.
Calcoliamo l'energia ridotta:
\begin{align*}
	\begin{cases}
		T_\text{rid} = \frac12 \mu v_\text{rid}^2 = \frac12 \mu v_\text{in}^2 \\
		T_\text{in}  = \frac12 m v_\text{in}^2
	\end{cases} \implies
	T_\text{rid} = \frac\mu m T_\text{in} = \frac1{1+\frac mM} T_\text{in},
\end{align*}
dunque abbiamo $T_\text{rid}(T_\text{in})$.
Ora ricaviamo $T_\text{out}$ in funzione di $T_\text{in}$ e $\cos\theta$.
Equivalentemente ricaviamo la relazione per le velocità anziché per le energie.
\begin{align*}
	\begin{cases}
		\vec v_\text{rid,out} = \vec v_\text{out} - \vec V_\text{out} \text{ (def. di $\vec v_\text{rid}$)}\\
		m \vec v_\text{in} = m \vec v_\text{out} + M \vec V_\text{out} \text{ (conservazione impulso)}
	\end{cases}
\end{align*}
ricaviamo $\vec V_\text{out}$ dalla seconda equazione:
\begin{align*}
	\vec V_\text{out} = \frac mM (\vec v_\text{in} - \vec v_\text{out})
\end{align*}
sostituiamo nella prima:
\begin{align}
	\label{eq:star}
	\vec v_\text{rid,out}
	= \vec v_\text{out} \left(1 + \frac mM\right) - \frac mM \vec v_\text{in}
\end{align}
prendiamo il modulo nell'ultima equazione:
\begin{align*}
	v_\text{in}^2
	= v_\text{out}^2 \left(1 + \frac mM\right)^2 + \left(\frac mM\right)^2 v_\text{in}^2
	- 2 \frac mM \left(1 + \frac mM\right) \vec v_\text{in} \vec v_\text{out}
\end{align*}
ma $\vec v_\text{in}\vec v_\text{out} = v_\text{in} v_\text{out} \cos\theta$.
Risolviamo per $v_\text{out}$:
\begin{align*}
	0
	&= v_\text{out}^2 \left(1 + \frac mM\right)^2
	- v_\text{out} 2\frac mM\left(1 + \frac mM\right) v_\text{in} \cos\theta
	- v_\text{in}^2 \left(1 - \left(\frac mM\right)^2 \right) = \\
	&= \left(1 + \frac mM\right) \Bigg(
   v_\text{out}^2 \left(1 + \frac mM\right)
  	- v_\text{out} 2\frac mM v_\text{in} \cos\theta
  	- v_\text{in}^2 \left(1 - \frac mM\right)
	\Bigg) \implies \\
	\implies v_\text{out}
	&= v_\text{in} \frac
	{\frac mM \cos\theta + \sqrt{\left(\frac mM\right)^2 \cos^2\theta + \left(1-\frac mM\right)\left(1+\frac mM\right)}}
	{1 + \frac mM} = \\
	&= v_\text{in} \frac
	{\frac mM \cos\theta + \sqrt{1 - \left(\frac mM\right)^2 (1-\cos^2\theta)}}
	{1 + \frac mM} = \\
	&= v_\text{in} \frac
	{\sqrt{1 - \left(\frac mM\sin\theta\right)^2} + \frac mM\cos\theta}
	{1 + \frac mM}
\end{align*}
dunque abbiamo $v_\text{out}(\vec v_\text{in},\cos\theta)$.
Ora ricaviamo $\cos\theta_\text{rid}$ in funzione di $\cos\theta$.
Prendiamo la proiezione su $\vec v_\text{in}$ della \eqref{eq:star}:
\begin{align*}
	\cos\theta_\text{rid}
	&= \frac{v_\text{out}}{v_\text{in}} \cos\theta \left(1+\frac mM\right) - \frac mM = \\
	&= \left( \sqrt{1 - \left(\frac mM \sin\theta\right)^2} + \frac mM \cos\theta \right) \cos\theta - \frac mM.
\end{align*}
