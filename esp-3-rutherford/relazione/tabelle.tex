\section{Dati}

\begin{table}[h]
\centering
\begin{tabular}{c|c|c}

pressione [mbar] & rate [\si{s^{-1}}] & moda [digit] \\
\hline
$ 0.0190 \pm 0.0010 $ & $ 44.6 \pm 0.9 $ & $ 3183 $ \\ 
$ 0.180 \pm 0.010 $ & $ 46.4 \pm 1.0 $ & $ 3182 $ \\ 
$ 1.00 \pm 0.20 $ & $ 44.8 \pm 0.9 $ & $ 3181 $ \\ 
$ 8.90 \pm 0.10 $ & $ 45.4 \pm 0.8 $ & $ 3172 $ \\ 
$ 23.0 \pm 1.0 $ & $ 44.9 \pm 0.9 $ & $ 3153 $ \\ 
$ 63.0 \pm 1.0 $ & $ 43.1 \pm 0.8 $ & $ 3103 $ \\ 
$ 81.0 \pm 1.0 $ & $ 44.0 \pm 0.7 $ & $ 3078 $ \\ 
$ 110 \pm 10 $ & $ 47.1 \pm 1.0 $ & $ 3049 $ \\ 
$ 150 \pm 10 $ & $ 44.4 \pm 0.9 $ & $ 3027 $ \\ 
$ 220 \pm 10 $ & $ 43.4 \pm 0.8 $ & $ 2984 $ \\ 
$ 470 \pm 10 $ & $ 45.9 \pm 0.9 $ & $ 2787 $ \\ 
$ 700 \pm 10 $ & $ 41.7 \pm 0.9 $ & $ 1712 $ \\ 
$ 860 \pm 10 $ & $ 0.13 \pm 0.04 $ & $ 521 $ \\ 

\end{tabular}
\caption{Dati per stimare l'effetto della pressione sulle misure.
L'incertezza sulla pressione è la cifra meno significativa del vacuometro digitale.
L'ultima misura (\SI{860}{mbar}) corrisponde alla pressione atmosferica
(il vacuometro si scalibra frequentemente).}
\label{tab:press}
\end{table}


\begin{table}[h]
\centering

\begin{tabular}{cc||cc|cc}
	\multicolumn{2}{c||}{15 marzo} & \multicolumn{4}{c}{19 marzo} \\
	\hline
soglia & rate [\si{s^{-1}}] & soglia & rate [\si{s^{-1}}] & soglia & rate [\si{s^{-1}}] \\
\hline
$ 0.0 $ & $ 45.4 \pm 1.5 $      & $ 0.0 $ & $ 43.7 \pm 0.9 $    & $ 0.6 $ & $ 42.0 \pm 0.9 $  \\
$ 0.5 $ & $ 45.0 \pm 1.5 $      & $ 0.1 $ & $ 44 \pm 1 $        & $ 0.7 $ & $ 44.0 \pm 0.9 $  \\
$ 0.7 $ & $ 43.2 \pm 2.1 $      & $ 0.2 $ & $ 45 \pm 1 $        & $ 0.8 $ & $ 36.5 \pm 0.7 $  \\
$ 0.8 $ & $ 38 \pm 2 $          & $ 0.3 $ & $ 43.9 \pm 0.8 $    & $ 0.9 $ & $ 5.6 \pm 0.2 $   \\
$ 0.9 $ & $ 10 \pm 1 $          & $ 0.4 $ & $ 43.6 \pm 0.9 $    & $ 1.0 $ & $ 0.05 \pm 0.03 $ \\
$ 1.0 $ & $\;\; 0.05 \pm 0.05 $ & $ 0.5 $ & $ 41.9 \pm 0.9 $    &&
\end{tabular}

\caption{Valori presenti nel pannello a sinistra di \autoref{fig:sogliaritardo}.}
\label{tab:soglia}
\end{table}



\begin{table}[h]
\centering

\begin{tabular}{c|c @{\,$\pm$\,} l}
soglia & \multicolumn{2}{c}{ritardo [\si{ns}]} \\
\hline
$ 0.0 $ & $ 88 $ & $ 2 $\\ 
$ 0.1 $ & $ 88 $ & $ 2 $\\ 
$ 0.2 $ & $ 116 $ & $ 2 $\\ 
$ 0.3 $ & $ 180 $ & $ 2 $\\ 
$ 0.4 $ & $ 288 $ & $ 2 $\\ 
$ 0.5 $ & $ 384 $ & $ 2 $\\ 
$ 0.6 $ & $ 464 $ & $ 4 $\\ 
$ 0.7 $ & $ 560 $ & $ 10 $\\ 
$ 0.8 $ & $ 720 $ & $ 60 $\\ 
$ 0.9 $ & $ 910 $ & $ 60 $\\ 
\end{tabular}

\caption{Dati per la misura in cui si quantifica il ritardo del segnale discriminato in funzione della soglia del discriminatore.
In pratica descrivono la forma del segnale.
Il ritardo è stato misurato con i cursori dell'oscilloscopio.
Le ultime misure hanno un errore maggiore a causa del jitter del segnale.}
\label{tab:rit}
\end{table}


\begin{table}[h]
\hspace{-3em}
\begin{tabular}{c|c|c||c|c|c}
	angolo [\si{\degree}] & rate  [\si{s^{-1}cm^2}] & moda [digit] & angolo [\si{\degree}] & rate  [\si{s^{-1}cm^2}] & moda [digit] \\
	\hline
	$ -42.6 \pm 1.5 $ & $ 5.2 \pm 1.3 $ & $ 1256 $                & $ 7.2 \pm 0.5 $ & $ 1.59 \pm 0.09   \cdot 10^{3} $ & $ 3140 $ \\  
	$ -38.0 \pm 1.3 $ & $ 257 \pm 16 $ & $ 2993 $                 & $ 9.6 \pm 0.5 $ & $ 1.56 \pm 0.09   \cdot 10^{3} $ & $ 3137 $ \\ 
	$ -33.3 \pm 1.1 $ & $ 538 \pm 31 $ & $ 3048 $                 & $ 12.0 \pm 0.6 $ & $ 1.45 \pm 0.08  \cdot 10^{3} $ & $ 3130 $ \\ 
	$ -31.0 \pm 1.0 $ & $ 6.6 \pm 0.4   \cdot 10^{2} $ & $ 3070 $ & $ 14.4 \pm 0.6 $ & $ 1.49 \pm 0.08  \cdot 10^{3} $ & $ 3125 $ \\ 
	$ -28.6 \pm 1.0 $ & $ 8.1 \pm 0.5   \cdot 10^{2} $ & $ 3087 $ & $ 16.7 \pm 0.7 $ & $ 1.36 \pm 0.08  \cdot 10^{3} $ & $ 3114 $ \\ 
	$ -26.2 \pm 0.9 $ & $ 9.7 \pm 0.5   \cdot 10^{2} $ & $ 3105 $ & $ 19.1 \pm 0.7 $ & $ 1.32 \pm 0.07  \cdot 10^{3} $ & $ 3101 $ \\ 
	$ -23.9 \pm 0.8 $ & $ 1.01 \pm 0.06 \cdot 10^{3} $ & $ 3113 $ & $ 21.5 \pm 0.8 $ & $ 1.28 \pm 0.07  \cdot 10^{3} $ & $ 3095 $ \\ 
	$ -21.5 \pm 0.8 $ & $ 1.10 \pm 0.06 \cdot 10^{3} $ & $ 3119 $ & $ 23.9 \pm 0.8 $ & $ 1.23 \pm 0.07  \cdot 10^{3} $ & $ 3087 $ \\ 
	$ -19.1 \pm 0.7 $ & $ 1.22 \pm 0.07 \cdot 10^{3} $ & $ 3129 $ & $ 26.2 \pm 0.9 $ & $ 1.19 \pm 0.07  \cdot 10^{3} $ & $ 3066 $ \\ 
	$ -16.7 \pm 0.7 $ & $ 1.22 \pm 0.07 \cdot 10^{3} $ & $ 3135 $ & $ 28.6 \pm 1.0 $ & $ 1.04 \pm 0.06  \cdot 10^{3} $ & $ 3047 $ \\ 
	$ -14.4 \pm 0.6 $ & $ 1.37 \pm 0.08 \cdot 10^{3} $ & $ 3134 $ & $ 31.0 \pm 1.0 $ & $ 9.5 \pm 0.5    \cdot 10^{2} $ & $ 3022 $ \\ 
	$ -12.0 \pm 0.6 $ & $ 1.39 \pm 0.08 \cdot 10^{3} $ & $ 3142 $ & $ 33.3 \pm 1.1 $ & $ 8.2 \pm 0.5    \cdot 10^{2} $ & $ 3002 $ \\ 
	$ -9.6 \pm 0.5 $ & $ 1.43 \pm 0.08  \cdot 10^{3} $ & $ 3144 $ & $ 35.7 \pm 1.2 $ & $ 6.5 \pm 0.4    \cdot 10^{2} $ & $ 2966 $ \\ 
	$ -7.2 \pm 0.5 $ & $ 1.45 \pm 0.08  \cdot 10^{3} $ & $ 3147 $ & $ 38.0 \pm 1.3 $ & $ 531 \pm 30 $ & $ 2937 $ \\  
	$ -4.8 \pm 0.5 $ & $ 1.47 \pm 0.08  \cdot 10^{3} $ & $ 3145 $ & $ 40.3 \pm 1.4 $ & $ 374 \pm 21 $ & $ 2892 $ \\  
	$ -2.4 \pm 0.5 $ & $ 1.55 \pm 0.08  \cdot 10^{3} $ & $ 3148 $ & $ 42.6 \pm 1.5 $ & $ 237 \pm 14 $ & $ 2840 $ \\  
	$ 0.0 \pm 0.5 $ & $ 1.56 \pm 0.09   \cdot 10^{3} $ & $ 3149 $ & $ 44.9 \pm 1.6 $ & $ 94 \pm 6 $ & $ 2717 $ \\  
	$ 2.4 \pm 0.5 $ & $ 1.56 \pm 0.09   \cdot 10^{3} $ & $ 3145 $ & $ 47.1 \pm 1.8 $ & $ 2.6 \pm 0.9 $ & $ 964 $ \\  
	$ 4.8 \pm 0.5 $ & $ 1.66 \pm 0.09   \cdot 10^{3} $ & $ 3141 $ & $ 7.2 \pm 0.5 $ & $ 1.59 \pm 0.09   \cdot 10^{3} $ & $ 3140 $ \\ 
\end{tabular}

\caption{Dati per la forma del fascio mostrati in (\autoref{fig:forma}).
Le incertezze sugli angoli sono correlate perché nel calcolo entrano le lunghezze
$L=\SI{28.5(10)}{mm}$ e $D=\SI{31(1)}{mm}$.
Gli angoli impostati sulla scala graduata sono \SI{-90}{\degree}, \SI{-80}{\degree},
da \SI{-70}{\degree} a \SI{100}{\degree} in passi da \SI{5}{\degree}.}
\label{tab:forma}
\end{table}


\begin{table}[h]
\centering

\begin{tabular}{c|c|c}
angolo [\si{\degree}] & rate  [\si{s^{-1}}] & moda [digit] \\
\hline
$ -15.00 $ & $ 0.0 \pm 0 $ & $ 0 $ \\ 
$ -10.00 $ & $ 2.76 \pm 0.18 $ & $ 3177 $ \\ 
$ -7.50 $ & $ 27.8 \pm 0.6 $ & $ 3183 $ \\ 
$ -5.00 $ & $ 42.4 \pm 0.9 $ & $ 3186 $ \\ 
$ -2.50 $ & $ 43.9 \pm 1.0 $ & $ 3181 $ \\ 
$ 0.00 $ & $ 44.4 \pm 0.9 $ & $ 3185 $ \\ 
$ 2.50 $ & $ 44.7 \pm 1.0 $ & $ 3187 $ \\ 
$ 5.00 $ & $ 47.1 \pm 1.0 $ & $ 3182 $ \\ 
$ 7.50 $ & $ 43.2 \pm 0.9 $ & $ 3180 $ \\ 
$ 10.00 $ & $ 15.6 \pm 0.4 $ & $ 3188 $ \\ 
$ 15.00 $ & $ 0.0 \pm 0 $ & $ 0 $ \\ 

\end{tabular}

\caption{Rate in funzione dell'angolo con il collimatore da 5\! mm.}
\label{tab:coll5}
\end{table}




\begin{table}[h]
\centering

\begin{tabular}{c|c|c}
angolo [\si{\degree}] & rate  [\si{s^{-1}}] & moda [digit] \\
\hline
$ -2.50 $ & $ 0.08 \pm 0.04 $ & $ 3156 $ \\ 
$ -1.25 $ & $ 2.48 \pm 0.20 $ & $ 3171 $ \\ 
$ 0.00 $ & $ 14.63 \pm 0.32 $ & $ 3174 $ \\ 
$ 1.25 $ & $ 31.5 \pm 0.7 $ & $ 3178 $ \\ 
$ 2.50 $ & $ 34.8 \pm 0.8 $ & $ 3185 $ \\ 
$ 3.75 $ & $ 26.9 \pm 0.6 $ & $ 3184 $ \\ 
$ 5.00 $ & $ 12.70 \pm 0.31 $ & $ 3188 $ \\ 
$ 6.25 $ & $ 2.81 \pm 0.22 $ & $ 3193 $ \\ 
$ 7.50 $ & $ 0.11 \pm 0.04 $ & $ 3185 $ \\ 

\end{tabular}

\caption{Dati per le misure con il collimatore da 1\! mm.}
\label{tab:coll1}
\end{table}



% \begin{table}
% \centering
%
% \begin{tabular}{c|c|c}
% angolo [\si{\degree}] & rate  [\si{s^{-1}}] & moda [digit] \\
% \hline
% $ -2.50 $ & $ 0.27 \pm 0.04 $ & $ 3213 $ \\
% $ -1.25 $ & $ 1.20 \pm 0.11 $ & $ 3198 $ \\
% $ 0.00 $ & $ 2.36 \pm 0.15 $ & $ 3190 $ \\
% $ 1.25 $ & $ 2.56 \pm 0.16 $ & $ 3187 $ \\
% $ 2.50 $ & $ 2.41 \pm 0.15 $ & $ 3192 $ \\
% $ 3.75 $ & $ 1.32 \pm 0.09 $ & $ 3187 $ \\
% $ 5.00 $ & $ 0.32 \pm 0.05 $ & $ 3193 $ \\
%
% \end{tabular}
%
% \caption{Dati per le misure con il collimatore a croce.}
% \label{tab:super}
% \end{table}
%
%
%
