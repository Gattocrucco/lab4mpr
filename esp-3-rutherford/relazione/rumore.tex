\subsection{Rumore}

Verifichiamo che il circuito può venire triggerato da scariche elettriche nei dintorni.
Ciò avviene sistematicamente azionando interruttori in potenza
(cioè non interruttori elettronici come ad esempio dei computer)
o collegando spine vicino all'apparato;
a distanza maggiore (ad esempio l'interruttore della luce) avviene più di rado.
Inoltre viene triggerato accendendo i neon,
questo effetto però è ottico perché coprendo la camera con un panno nero l'effetto si annulla.

Confrontando gli eventi salvati dall'ADC con il conteggio del contatore
e provocando intenzionalmente rumori, osserviamo che:
\begin{itemize}
	\item
	Ci sono rumori ``intensi'' che producono più di un conteggio
	e spesso il contatore segnala più eventi di quelli salvati dall'ADC, mai di meno.
	I conteggi multipli hanno lo stesso timestamp.
	\item
	Quasi sempre (sicuramente più del \SI{99.9}{\percent} delle volte)
	il canale 2 dell'ADC, che non usiamo ma che viene letto, segna 0.
	Le letture diverse da zero sono correlate, ma non totalmente, con i rumori intensi.
	\item
	Negli spettri si nota un accumulo vicino a zero anche per spettri lontani dallo zero.
	Poiché il rate di zeri cresce con il rate di segnale, è slegato da quello di rumore
	e non ha coincidenza di timestamp a meno di rate alto rispetto alla risoluzione temporale,
	riteniamo che sia un errore di lettura di qualche tipo piuttosto che un rumore.
\end{itemize}
I rumori sono significativi nelle misure a basso rate.
Limitiamo i problemi in questo modo:
\begin{itemize}
	\item
	Anziché usare il conteggio del contatore usiamo il numero di eventi dell'ADC
	eliminando quelli che hanno canale 2 non nullo e,
	per rate minore di \SI1{min^{-1}},
	gruppi di eventi con lo stesso timestamp.
	\item
	Teniamo la camera coperta con un panno nero.
	\item
	Badiamo a non attaccare spine o azionare interruttori.
\end{itemize}
