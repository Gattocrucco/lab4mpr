\subsection{Rumore}

Verifichiamo che il circuito può venire triggerato da scariche elettriche nei dintorni.
Ciò avviene sistematicamente azionando interruttori in potenza
(cioè non interruttori elettronici come ad esempio dei computer)
o collegando spine vicino all'apparato,
a distanza maggiore (ad esempio l'interruttore della luce) più di rado.
Inoltre viene triggerato accendendo i neon,
questo effetto però è ottico perché coprendo la camera con un panno nero l'effetto si annulla.

Confrontando gli eventi salvati dall'ADC con il contatore
e provocando intenzionalmente rumori, osserviamo che:
\begin{itemize}
	\item
	Ci sono rumori ``intensi'' che producono più di un conteggio
	e spesso il contatore segnala più eventi di quelli salvati dall'ADC, mai di meno.
	I conteggi multipli hanno lo stesso timestamp.
	\item
	Quasi sempre (sicuramente più del \SI{99.9}\percent delle volte)
	il canale 2 dell'ADC, che non usiamo ma che viene letto, segna 0.
	Le letture diverse da zero sono correlate, ma non totalmente, con i rumori intensi.
	\item
	Le 
\end{itemize}