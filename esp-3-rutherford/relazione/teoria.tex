\subsection{Sorgente}

La sorgente \am{} emette $\alpha$ a due energie cinetiche\footnote{PDG 2016 \S 37.}: \SI{5.443}{MeV}, \SI{5.486}{MeV}. La sorgente emette anche raggi $\gamma$ a \SI{60}{keV} (\SI{36}\%),
per i quali la lunghezza di attenuazione nel silicio è circa\footnote{PDG 2016 fig. 33.18.}
$\SI{3}{g\,cm^{-2}} / \SI{2.6}{g\,cm^{-3}} = \SI{1.1}{cm}$.
Lo spessore del rivelatore è dell'ordine dei \si{\micro m}
e i fotoni vengono assorbiti con una probabilità circa del \SI{50}\%\footnote{NIST XCOM database \url{https://physics.nist.gov/PhysRefData/Xcom/html/xcom1.html}},
allora la probabilità di assorbimento è circa \SI{1/20000}{\micro m^{-1}}.
Dalla scheda sappiamo l'area del fotodiodo (\SI{20}{mm^2})
e la distanza della sorgente dal fotodiodo ad angolo \SI0{\degree} è \SI{6}{cm},
quindi il rate atteso di fotoni che rilasciano tutta l'energia è
$\SI{315}{kBq} \times \SI{36}\% \times \SI{20}{mm^2} / (4 \pi (\SI{6}{cm})^2) \times \SI{1/20000}{\micro m^{-1}}
= \SI{0.003}{Hz\,\micro m^{-1}}$.
\marginpar{Ma quando vengono assorbiti i fotoni, poi vengono anche rivelati? Suppongo di sì.}

\subsection{Scattering Rutherford}

La sezione d'urto differenziale in approssimazione non relativistica di una particella di carica $+ze$ ed energia cinetica $T$ su un campo elettrostatico di carica $+Ze$ è\footnote{Wikipedia: Rutherford scattering \url{https://en.wikipedia.org/wiki/Rutherford_scattering}.}
\begin{equation}
	\label{eq:rutherford}
	\dv\sigma\Omega = \left( \frac {zZ\alpha\hbar c} {2T(1-\cos\theta)} \right)^2.
\end{equation}
Per un campo generato da una massa $M$ di carica $+Ze$,
applicando nella \eqref{eq:rutherford} le trasformazioni del problema in coordinate relative con la massa ridotta,
\marginpar{Mettere dimostrazione in appendice apposita.}
con $m$ massa della particella incidente e $T_\text{in}$, $T_\text{out}$ energia cinetica rispettivamente prima e dopo l'interazione, si ottiene
\begin{align*}
	T_\text{out}
	&= T_\text{in} \left( 1 - 2\frac mM(1-\cos\theta) + O\left(\frac mM\right)^2 \right), \\
	\dv\sigma\Omega &= \left( \frac {zZ\alpha\hbar c} {2T_\text{in}(1-\cos\theta)} \right)^2
	\left( 1 + O\left(\frac mM\right)^2 \right).
\end{align*}
Notiamo che la correzione al primo ordine nel rapporto delle masse alla sezione d'urto è nulla.
