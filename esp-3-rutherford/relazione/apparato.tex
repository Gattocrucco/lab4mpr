\section{Apparato}

La parte principale del nostro apparato è costituita da una camera a vuoto cilindrica con un diametro interno di  \SI{16.3(1)}{cm} e altezza 11 cm a cui è collegata la relativa pompa e vari 
manometri, vacuometri, vuotometri e sfigmomanometri. 
\marginpar{i numeri senza errore sono presi dalla scheda}
Nel centro è possibile inserire un bersaglio che può essere ruotato per variare l'angolo tra il proiettile e il rivelatore fisso, un fotodiodo al silicio.
I bersagli a nostra disposizione sono lamine metalliche di diversi spessori: \marginpar{diametro finestra=12 mm}
\begin{itemize}

\item oro: \SI{0.2}{\micro m}, \SI{5}{\micro m}, \SI{20}{\micro m} ($\times$2);
\item alluminio: \SI{8}{\micro m}, spessore ignoto;
\item acciaio: \SI{10}{\micro m}.

\end{itemize} 

I nostri proiettili sono le particelle $\alpha$ emesse da una sorgente di \am{} con attività \SI{330}{kBq} nel 1990 che si riduce a \SI{315}{kBq} nel 2018.
Possiamo scegliere di collimare il fascio incidente con due collimatori in plastica aventi una fessura rettangolare larga \SI{1}{mm} o \SI{5}{mm} ed altezza di \SI{5}{cm}.
Il fotodiodo ha davanti a sé un collimatore rettangolare di dimensioni $2\times 6$\! mm rotabile a piacere.
\marginpar{spiegare in seguito che lo lasceremo sempre in verticale}
