\subsection{Apparato}

La parte principale del nostro apparato è costituita da una camera a vuoto cilindrica con un diametro interno di  \SI{16.3(1)}{cm} e altezza 11 cm a cui è collegata la relativa pompa e i vacuometri.
\marginpar{i numeri senza errore sono presi dalla scheda\\
\emph{11 cm? Ma non avevi misurato 8 cm?}}
Nel centro è possibile inserire un bersaglio che può essere ruotato solidalmente alla sorgente radioattiva $\alpha$
per variare l'angolo tra il proiettile e il rivelatore fisso, un fotodiodo al silicio.
La scheda afferma che il rivelatore assorbe completamente le particelle $\alpha$,
permettendo di misurarne l'energia.

I bersagli a nostra disposizione sono lamine metalliche di diversi spessori:
\begin{itemize}
	\item oro: \SI{3}{\micro m}, \SI{5}{\micro m}, \SI{20}{\micro m} ($\times$2);
	\item alluminio: \SI{8}{\micro m} ($\times$2);
	\item acciaio: \SI{10}{\micro m};
	\item oro ``calibrato'': \SI2{\micro m};
	\item alluminio ``calibrato'': \SI8{\micro m}.
\end{itemize}
Le lamine ``calibrate'' sono quelle per le quali ci è garantito il valore dello spessore,
le altre lamine erano intese di prova.
Poiché abbiamo saputo solo alla fine dell'esperimento dell'esistenza di questa distinzione,
tutte le misure sono fatte con le lamine ``non calibrate'',
con le lamine ``calibrate'' abbiamo solo fatto delle misure di controllo.

I nostri proiettili sono le particelle $\alpha$ emesse da una sorgente di \am{} con attività \SI{330}{kBq} nel 1990 che si riduce a \SI{315}{kBq} nel 2018.
Possiamo scegliere di collimare il fascio incidente con due collimatori in plastica aventi una fessura rettangolare larga \SI{1}{mm} o \SI{5}{mm} ed altezza di \SI{5}{cm}.
La lamina bersaglio è in ogni caso accoppiata a una maschera circolare di diametro \SI{12}{mm}.
Il fotodiodo ha davanti a sé un collimatore rettangolare di dimensioni $2\times 6$\! mm rotabile a piacere.
\marginpar{spiegare in seguito che lo lasceremo sempre in verticale}

\paragraph{Elettronica di misura}

\marginpar{Stavolta Jack ci delizierà con i suoi fantastici disegni?}

I segnali analizzati in questa esperienza provengono da un fotodiodo al silicio la cui uscita è inviata ad un insieme di discriminatori e preamplificatori. Essi forniscono un segnale logico TTL ed un segnale analogico NIM preamplificato, entrambi della durata di pochi \si{\micro s}.
Disponiamo di un amplificatore che genera un segnale la cui altezza di picco è proporzionale all'energia rilasciata nel fotodiodo.
\marginpar{do per scontato che nella \emph{Teoria} sia scritto che il fotodiodo assorba tutta l'energia delle $\alpha$}
Tale picco viene letto da un'ADC a \SI{12}{bit}. Gli altri strumenti sono i soliti moduli (contatori, timer, coincidenze...) presenti in laboratorio.

