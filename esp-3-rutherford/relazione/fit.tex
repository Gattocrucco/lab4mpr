\subsubsection{Fit}

Eseguiamo il fit ai minimi quadrati del rate in funzione dell'angolo,
assegnando l'incertezza poissoniana ai rate e \SI{\pm1}{\degree} agli angoli.
Nel caso di collimatore da \SI1{mm} usiamo una funzione proporzionale alla \eqref{eq:rutherford}:
\begin{align}
	\label{eq:fit1}
	R_1(\theta;B,\theta_0)
	&= \frac {B} {(1-\cos(y(\theta-\theta_0)))^2}, \\
	y(\theta)
	&= \theta + (\operatorname{sgn} \theta) \theta_\text{min} e^{-|\theta|/\theta_\text{min}},
	\quad \theta_\text{min} = \SI{0.2}{\degree}, \notag
\end{align}
dove $y$ serve solo a regolarizzare la funzione per il conto numerico ($\theta_\text{min}$ non influisce sul fit).
Per il collimatore da \SI5{mm} mediamo la \eqref{eq:fit1}
supponendo che il fascio incidente abbia una distribuzione uniforme sul collimatore:
\begin{align*}
	R_5(\theta;B,\theta_0)
	&= \int_{-\ell/2}^{\ell/2} \frac{\de a}{\ell}
	\frac B {(1-\cos(y(\theta_R(a, \theta-\theta_0))))^2}, \\
	\theta_R(a, \theta)
	&= \tan^{-1} \left( \tan(\theta) - \frac a {L\cos\theta} \right) - \tan^{-1}\frac aD,
\end{align*}
dove $a$ è la coordinata sul collimatore, $\theta_R$ è l'angolo di cui effettivamente viene deviata la traiettoria attraversando il bersaglio,
$L$ e $D$ sono definite in \autoref{sec:forma}, $\ell$ è la larghezza del collimatore.
I dati con sovraimposte le curve di fit sono riportati in \autoref{fig:fit},
i risultati sono riportati in \autoref{tab:fit}.

\begin{figure}
	\hspace{-0.2\textwidth}
	{\includegraphics[height=0.5\textwidth]{immagini/all}}
	~
	{\includegraphics[height=0.5\textwidth]{immagini/oro5}}
	\includegraphics[height=0.5\textwidth]{immagini/oro3}
	\caption{\label{fig:fit}
	cippa}
\end{figure}

\begin{table}
	\centering
	\begin{tabular}{cccrr}
		Bersaglio & $B$ [\si{s^{-1}}] & $\theta_0$ [\si\degree] & corr. & $\chi^2$/dof \\
		\hline
		\texttt{al8coll1} & \num{5.7(11)e-6 } & \num{1.5 \pm 0.5  } & \SI{41.5} \% & 3.1 / 4  \\
		\texttt{al8coll5} & \num{2.19(19)e-5} & \num{1.2 \pm 0.4  } & \SI{8.8 }\%  & 10.7 / 8 \\
		\texttt{au3coll1} & \num{1.22(10)e-4} & \num{3.02 \pm 0.35} & \SI{9.1 }\%  & 3.8 / 7  \\
		\texttt{au3coll5} & \num{5.26(22)e-4} & \num{1.1 \pm 0.4  } & \SI{5.8 }\%  & 11.1 / 9 \\
		\texttt{au5coll1} & \num{1.21(11)e-4} & \num{2.2 \pm 0.4  } & \SI{-1.4} \% & 7.9 / 4  \\
		\texttt{au5coll5} & \num{6.8(3)e-4  } & \num{-0.5 \pm 0.4 } & \SI{-6.2} \% & 80.5 / 7
	\end{tabular}
	\caption{\label{tab:fit}
	Risultati del fit. Il bersaglio è indicato come
	\texttt{<elemento><spessore [\si{\micro m}]>coll<collimatore [\si{mm}]>}.}
\end{table}

\subsubsection{Rapporto delle cariche nucleari}

Comparando i parametri di ampiezza dei fit con l'oro con quello eseguito sull'alluminio, possiamo estrarre lo Z di quest'ultimo. Le ampiezze di fit hanno la forma $$ B=\mathcal{L} \left( \frac {zZ\alpha\hbar c} {2T} \right)^2 = \mathcal{L} A . $$
La nostra luminosità vale $\mathcal{L}=r n_2 l$, in cui $r$ è il rate di particelle incidenti, $n$  è la densità di bersagli e $l$ è lo spessore della targhetta.
Nel nostro caso $n_1$ è uguale per tutte le misure con lo stesso collimatore.   \marginpar{c'è bisogno di precisare perché?}
Da queste considerazioni possiamo trarre lo Z dell'alluminio dalla relazione \eqref{zeta} usando i parametri di fit per i dati con lo stesso collimatore.

\begin{equation}
Z_{\text{al}}=Z_{\text{au}} \sqrt{ \frac{B_{\text{al}}}{B_{\text{au}}} \frac{n_{\text{au}} l_{\text{au}}}{n_{\text{al}} l_{\text{al}}} }
\label{zeta}
\end{equation}

Otteniamo i seguenti risultati:

\begin{align*}
\text{Oro } \SI3{\micro m}&:\\
\text{collimatore } \SI1{mm}&: Z_{\text{al}}=\num{10.4(4)} \\
\text{collimatore } \SI5{mm}&: Z_{\text{al}}=\num{9.2(4)}. \\
\text{Oro } \SI5{\micro m}&:\\
\text{collimatore } \SI1{mm}&: Z_{\text{al}}=\num{13.5(6)} \\
\text{collimatore } \SI5{mm}&: Z_{\text{al}}=\num{70(2)}. \\
\end{align*}

Soltanto una misura è compatibile con lo $Z$ dell'alluminio atteso. Gli altri risultati sono influenzati dallo scattering multiplo all'interno dell'oro, fenomeno praticamente assente nell'alluminio.
L'ultimo risultato è molto lontano dal valore atteso perché lo scattering multiplo influisce pesantemente su una lamina d'oro così spessa e il fatto che il collimatore da \SI5{mm} selezioni più angoli rispetto a quello da \SI1{mm} non fa che peggiorare la situazione.