\subsection{Fit}

Misuriamo il rate di eventi al variare dell'angolo con i seguenti materiali:
\begin{itemize}
\item oro \SI{3}{\micro m}
\item oro \SI5{\micro m}
\item alluminio \SI8{\micro m}.
\end{itemize}

Prendiamo una funzione di fit della forma $R=\sigma \mathcal{L}$, dove $\mathcal{L}$ è la luminosità, avendo $B$ e $\theta_0$ come parametri di fit.

\begin{equation}
R=\frac{B}{\sin^4{\left(\frac{\theta-\theta_0}{2}\right)}}
\label{rute}
\end{equation}

Abbiamo aggiunto il parametro libero $\theta_0$ perché, come detto in \autoref{spiegazione}, gli angoli indicati dalle tacche sono diversi da quelli tra sorgente e rivelatore.