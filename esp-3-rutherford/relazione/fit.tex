\subsection{Fit}

Misuriamo il rate di eventi al variare dell'angolo con i seguenti materiali:
\begin{itemize}
\item oro \SI{3}{\micro m}
\item oro \SI5{\micro m}
\item alluminio \SI8{\micro m}.
\end{itemize}




\subsubsection{Z dell'alluminio}

Comparando i parametri di ampiezza dei fit con l'oro con quello eseguito sull'alluminio, possiamo estrarre lo Z di quest'ultimo. Le ampiezze di fit hanno la forma $$ B=\mathcal{L} \left( \frac {zZ\alpha\hbar c} {2T} \right)^2 = A \mathcal{L} $$.
La nostra luminosità vale $\mathcal{L}=n_1 n_2 l$, in cui $n_1$ è il rate di particelle incidenti, $n_2$  è la densità di bersagli e $l$ è lo spessore della targhetta.
Nel nostro caso $n_1$ è uguale per tutte le misure con lo stesso collimatore.   \marginpar{c'è bisogno di precisare perché?}
Da queste considerazioni possiamo trarre lo Z dell'alluminio dalla relazione \eqref{zeta} usando i parametri di fit per i dati con lo stesso collimatore.
\begin{equation}
Z_{\text{al}}=Z_{\text{au}} \sqrt{ \frac{B_{\text{al}}}{B_{\text{au}}} \frac{n_{\text{au}} l_{\text{au}}}{n_{\text{al}} l_{\text{al}}} }
\label{zeta}
\end{equation}