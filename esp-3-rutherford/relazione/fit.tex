\subsection{Fit}

Misuriamo il rate di eventi al variare dell'angolo con i seguenti materiali:
\begin{itemize}
\item oro \SI{3}{\micro m}
\item oro \SI5{\micro m}
\item alluminio \SI8{\micro m}.
\end{itemize}




\subsubsection{Z dell'alluminio}

Comparando i parametri di ampiezza dei fit con l'oro con quello eseguito sull'alluminio, possiamo estrarre lo Z di quest'ultimo. Le ampiezze di fit hanno la forma $$ B=\mathcal{L} \left( \frac {zZ\alpha\hbar c} {2T} \right)^2 = \mathcal{L} A . $$
La nostra luminosità vale $\mathcal{L}=r n_2 l$, in cui $r$ è il rate di particelle incidenti, $n$  è la densità di bersagli e $l$ è lo spessore della targhetta.
Nel nostro caso $n_1$ è uguale per tutte le misure con lo stesso collimatore.   \marginpar{c'è bisogno di precisare perché?}
Da queste considerazioni possiamo trarre lo Z dell'alluminio dalla relazione \eqref{zeta} usando i parametri di fit per i dati con lo stesso collimatore.

\begin{equation}
Z_{\text{al}}=Z_{\text{au}} \sqrt{ \frac{B_{\text{al}}}{B_{\text{au}}} \frac{n_{\text{au}} l_{\text{au}}}{n_{\text{al}} l_{\text{al}}} }
\label{zeta}
\end{equation}

Otteniamo i seguenti risultati:

\begin{align*}
\text{Oro } \SI3{\micro m}&:\\
\text{collimatore } \SI1{mm}&: Z_{\text{al}}=\num{10.4(4)} \\
\text{collimatore } \SI5{mm}&: Z_{\text{al}}=\num{9.2(4)}. \\
\text{Oro } \SI5{\micro m}&:\\
\text{collimatore } \SI1{mm}&: Z_{\text{al}}=\num{13.5(6)} \\
\text{collimatore } \SI5{mm}&: Z_{\text{al}}=\num{70(2)}. \\
\end{align*}

Soltanto una misura è compatibile con lo $Z$ dell'alluminio atteso. Gli altri risultati sono influenzati dallo scattering multiplo all'interno dell'oro, fenomeno praticamente assente nell'alluminio.
L'ultimo risultato è molto lontano dal valore atteso perché lo scattering multiplo influisce pesantemente su una lamina d'oro così spessa e il fatto che il collimatore da \SI5{mm} selezioni più angoli rispetto a quello da \SI1{mm} non fa che peggiorare la situazione.