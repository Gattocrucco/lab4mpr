\subsection{Backscattering}

Poniamo le nostre lamine a \SI{150}{\degree} (il valore più grande sulla scala graduata) per verificare la presenza di backscattering. Usiamo 2 lamine attaccate per essere sicuri che nessuna particella attraversi il bersaglio. Questo accorgimento ci permette di evitare che delle particelle $\alpha$ entrino nel rivelatore dopo essere rimbalzate sulle pareti. Abbiamo controllato che questa ipotesi sia sensata contando le particelle che attraversano 2 lamine dello stesso materiale (e spessore) a \SI{0}{\degree}.

Facendo misure con \SI{40}{\micro m} di oro non abbiamo osservato nessun evento nei \SI{1406}s\,$\approx$\,\SI{23}{minuti} della misura. La situazione è diversa per \SI{16}{\micro m} di alluminio: abbiamo un rate di \SI{0.57(8)}{s^{-1}}.
Questo risultato però non è il nostro fondo perché nelle misure a \SI{150}{\degree} sarà rappresentato da quelle particelle $\alpha$ che, rimbalzando sulle pareti, arrivano nel fotodiodo. Sappiamo anche che il fondo maggiore è rappresentato dalle particelle che arrivano al rivelatore dopo aver rimbalzato sui supporti di plastica delle lamine.

Abbiamo quantificato quest'ultimo effetto ponendo un supporto senza lamina al centro della camera e misurando il rate a \SI{150}{\degree}. Otteniamo $R_{\text{buco}}=\SI{3.9(6)e-4}{s^{-1}}$.
Siccome in presenza dell'alluminio le particelle che raggiungono la parete opposta alla sorgente sono ancora meno che nella situazione precedente, possiamo affermare che il fondo è minore di quello misurato. Ci aspettiamo che il rate di fondo sia trascurabilmente minore di $R_{\text{buco}}$ sia per motivi geometrici che dinamici. La particella $\alpha$ in questione dovrebbe colpire un piccolo rivelatore rimbalzando in una grande camera a vuoto e la sezione d'urto differenziale Rutherford decresce molto velocemente per angoli inferiori a \SI{10}{\degree}. Inoltre, nei rimbalzi ad angolo maggiore, la perdita di energia è più elevata. Questi meccanismi sfavoriscono diffusioni multiple all'interno della camera.

Di seguito riportiamo i risultati delle nostre misure di backscattering:
\begin{align*}
R_{\text{oro}} &= \SI{1.3(2)e-3}{s^{-1}} \\
R_{\text{alluminio}} &= \SI{5.4(6)e-4}{s^{-1}} \\
R_{\text{fondo}} \simeq R_{\text{buco}} &= \SI{3.9(6)e-4}{s^{-1}} \\
\text{differenza oro-fondo} &= \SI{5.9(4)}{\sigma} \\
\text{differenza alluminio-fondo} &= \SI{2.6(11)}{\sigma} \\
\end{align*}

Possiamo affermare in modo indiscutibile l'esistenza del backscattering nel caso dell'oro, ma non possiamo fare altrettanto con l'alluminio, almeno in questa misura.
Per poter osservare un indiscutibile backscattering anche con l'alluminio, abbiamo posto 2 dissipatori di un pc in prossimità del fotodiodo e la sorgente a \SI{150}{\degree}.
\marginpar{inserire dimensioni dissipatori}
Le dimensioni dei dissipatori impediscono alle particelle $\alpha$ di attraversarli e sono colpiti dalla quasi totalità del fascio, restituendoci una misura praticamente priva di fondo.
L'autopassivazione dell'alluminio crea uno strato di ossido spesso da 1 a \SI5{nm},%
\footnote{\url{http://www.valocchi.eu/logistica/alluminio.htm}}
pertanto la diffusione Rutherford su di esso è trascurabile.
\marginpar{Forse è meglio scrivere questa cosa all'inizio perché l'abbiamo trascurata anche per la lamina sottile.}

Abbiamo ottenuto \SI{54(7)}{eventi} in circa \SI{6.5}{ore}.