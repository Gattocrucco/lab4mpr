\documentclass[a4paper]{article}

\usepackage[utf8]{inputenc}
\usepackage[T1]{fontenc}
\usepackage[italian]{babel}

\usepackage[margin=4.2cm, top=1.5cm, bottom=2.5cm]{geometry}

\usepackage{siunitx}
\usepackage{amsmath}
\usepackage{amssymb}
\usepackage{xfrac}
\usepackage{esint}
\usepackage[hidelinks]{hyperref}
\usepackage{graphicx}
\usepackage[font={sf}]{caption}
\usepackage{pdflscape}
\usepackage{makecell}
\usepackage{float}
\usepackage{subfig}
\usepackage{wasysym}
\usepackage{booktabs}

\setlength{\marginparwidth}{95pt}
\let\oldmarginpar\marginpar
\renewcommand\marginpar[1]{\oldmarginpar{\scriptsize\sffamily #1}}
\newcommand*\de{\mathrm{d}}
\newcommand*\pdv[2]{\frac{\partial #1}{\partial #2}}
\newcommand*\dv[2]{\frac{\de #1}{\de #2}}
\DeclareMathOperator\Ei{Ei}
\newcommand*\is{\equiv}
\newcommand\cs{$^{\text{137}}\text{Cs}$}
\newcommand\co{$^{\text{60}}\text{Co}$}
\newcommand\na{$^{\text{22}}\text{Na}$}
\newcommand\am{$^{\text{241}}\text{Am}$}
\newcommand\sr{$^{\text{90}}\text{Sr}$}

\sisetup{%
separate-uncertainty=true,
multi-part-units=single,
exponent-product=\cdot}

\frenchspacing

\title{Relazione di laboratorio:\\
Esperienza 3. Diffusione di Rutherford}
\author{Andrea Marasciulli
\and Giacomo Petrillo
\and Roberto Ribatti}
\date{13 marzo -- 20 aprile 2018}

\begin{document}

\maketitle

\begin{abstract}
	Verifichiamo l'andamento della sezione d'urto differenziale Rutherford,
	misuriamo il rapporto tra gli $Z$ di alluminio e oro,
	proviamo l'esistenza del backscattering.
	Non riusciamo a studiare quantitativamente lo spettro energetico.
\end{abstract}

{\tableofcontents}

\newpage
\section{Introduzione}

\subsection{Obiettivo}
Gli obiettivi principali dell'esperienza sono:
\begin{itemize}
	\item la misura della massa dell'elettrone a partire dallo spettro energetico di fotoni emessi dall'annichilazione di positroni;
	\item la misura del rapporto tra i branching ratio del decadimento del  \na\; per cattura elettronica e per emissione di positroni;
	\item la misura del rapporto tra le sezioni d'urto dell'annichilazione del $\beta^+$ in $2\gamma$ e quella in $3\gamma$ .
\end{itemize}

\subsection{Apparato di misura}
Gli strumenti principali a disposizione per questa esperienza (fatta eccezione per i soliti moduli \texttt{NIM}) consistono in:
\begin{itemize}
	\item una sorgente radioattiva principale di \na\; e altre sorgenti meno attive per la calibrazione: \cs\;, \co\;;
	\item i PMT1, 2 e 3\footnote{i PMT1 e 2 sono dello stesso modello}: rivelatori basati su cristalli scintillatori di NaI(Tl) che useremo come spettrometri;
	\item un'ADC \texttt{CAMAC} (8 canali, 12 bit) a integrazione di carica.
\end{itemize}

\subsubsection{Sorgenti radioattive}
La sorgente radioattiva principale di \na\; decade al $90.6\%$ $\beta^+$ e al $9.6\%$ per cattura elettronica in uno stato eccitato di $^{22}$Ne che a sua volta decade $\gamma$ emettendo un fotone di $\SI{1275}{keV}$. Nello spettro si vedranno anche i fotoni prodotti dall'annichilazione del positrone nella materia.
I principali modi di decadimento delle sorgenti di calibrazione a disposizione sono schematizzati in \autoref{tab:sorgenti_cal}.

\begin{table}[h]
	\centering
	\begin{tabular}{cccc}
		\toprule
		sorgenti & \multicolumn{3}{c}{principali modi di decadimento} \\
		\midrule
		\co & $\beta^{-} (\SI{318}{keV})$ & $\gamma (\SI{1173}{keV})$ & $\gamma (\SI{1332}{keV})$  \\
		\cs & $\beta^{-} (\SI{512}{keV})$ & $\gamma (\SI{662}{keV})$ \\
		\na & $\beta^{+} (\SI{546}{keV})$ & $\gamma (\SI{1275}{keV})$ \\
		\bottomrule
	\end{tabular}
	\caption{\label{tab:sorgenti_cal} Principali modi di decadimento delle sorgenti a disposizione \cite{2}.}
\end{table}



\section{Apparato}

La parte principale del nostro apparato è costituita da una camera a vuoto cilindrica con un diametro interno di  \SI{16.3(1)}{cm} e altezza 11 cm 
\marginpar{i numeri senza errore sono presi dalla scheda } 

\section{Teoria}

\subsection{Sorgente}

La sorgente \am{} emette $\alpha$ a due energie cinetiche\footnote{PDG 2016 \S{} 37.}: \SI{5.443}{MeV}, \SI{5.486}{MeV}.
\marginpar{Oro attaccato alla sorgente fa calare l'energia.}
La sorgente emette anche raggi $\gamma$ a \SI{60}{keV} (\SI{36}\%),
per i quali la lunghezza di attenuazione nel silicio è circa\footnote{PDG 2016 fig. 33.18.}
$\SI{3}{g\,cm^{-2}} / \SI{2.6}{g\,cm^{-3}} = \SI{1.1}{cm}$.
Lo spessore del rivelatore è dell'ordine dei \si{\micro m}
e i fotoni vengono assorbiti con una probabilità circa del \SI{50}\%\footnote{NIST XCOM database \url{https://physics.nist.gov/PhysRefData/Xcom/html/xcom1.html}.},
allora la probabilità di assorbimento è circa \SI{1/20000}{\micro m^{-1}}.
Dalla scheda sappiamo l'area del fotodiodo (\SI{20}{mm^2})
e la distanza della sorgente dal fotodiodo ad angolo \SI0{\degree} è \SI{6}{cm},
quindi il rate atteso di fotoni che rilasciano tutta l'energia è
$\SI{315}{kBq} \times \SI{36}\% \times \SI{20}{mm^2} / (4 \pi (\SI{6}{cm})^2) \times \SI{1/20000}{\micro m^{-1}}
= \SI{0.003}{Hz\,\micro m^{-1}}$.
\marginpar{Ma quando vengono assorbiti i fotoni, poi vengono anche rivelati? Suppongo di sì.}

\subsection{Scattering Rutherford}

La sezione d'urto differenziale in approssimazione non relativistica di una particella di carica $+ze$ ed energia cinetica $T$ su un campo elettrostatico di carica $+Ze$ è\footnote{Wikipedia: Rutherford scattering \url{https://en.wikipedia.org/wiki/Rutherford_scattering}.}
\begin{equation}
	\label{eq:rutherford}
	\dv\sigma\Omega = \left( \frac {zZ\alpha\hbar c} {2T(1-\cos\theta)} \right)^2.
\end{equation}
Per un campo generato da una massa $M$ di carica $+Ze$,
applicando nella \eqref{eq:rutherford} le trasformazioni del problema in coordinate relative con la massa ridotta
(vedi \autoref{sec:conti}),
con $m$ massa della particella incidente e $T_\text{in}$, $T_\text{out}$ energia cinetica rispettivamente prima e dopo l'interazione, si ottiene
\begin{align}
	T_\text{out}
	&= T_\text{in} \left( 1 - 2\frac mM(1-\cos\theta) + O\left(\frac mM\right)^2 \right),
	\label{eq:tout} \\
	\dv\sigma\Omega &= \left( \frac {zZ\alpha\hbar c} {2T_\text{in}(1-\cos\theta)} \right)^2
	\left( 1 + O\left(\frac mM\right)^2 \right). \notag
\end{align}
Notiamo che la correzione al primo ordine nel rapporto delle masse alla sezione d'urto è nulla.

\subsection{Multiplo scattering}

Le particelle $\alpha$, passando nel bersaglio,
vengono in maggior parte deviate di piccoli angoli per interazione con molti atomi
e occasionalmente di un grande angolo per interazione con un singolo nucleo secondo la \eqref{eq:rutherford}.
Nel secondo caso comunque subiscono anche una deviazione ulteriore di un piccolo angolo
prima e dopo l'interazione ravvicinata con il nucleo.
La deviazione standard della distribuzione dei piccoli angoli,
nel limite non relativistico,
con l'angolo riferito a un piano fissato nello spazio anziché rispetto a un asse,
è data da\footnote{PDG 2016 \S{} 33.3.}
\begin{equation}
	\label{eq:ms}
	\theta_0 = \frac {\SI{6.8}{MeV}} {T_\text{in}} z \sqrt{\frac{x}{X_0}} \left(1+0.038\log\frac x{X_0}\right)
\end{equation}
dove $x$ è lo spessore attraversato e $X_0$ è la lunghezza di radiazione,
che vale\footnote{PDG 2016 \S{} 6.} \SI{8.9}{cm} per l'alluminio e \SI{0.33}{cm} per l'oro.
Ad esempio con $x = \SI5{\micro m}$ e $T_\text{in} = \SI5{MeV}$
risulta $\theta_0 = \SI{9}\degree$ per l'oro e \SI{1.5}{\degree} per l'alluminio.

\subsection{Simulazione}

\begin{figure}
	\centering
	\includegraphics[width=\textwidth]{immagini/schemamc}
	\caption{\label{fig:schemamc}
	Schema della simulazione.
	Il raggio esce dalla sorgente con un angolo $\theta_S$,
	fa multiplo scattering di $\theta_\text{MS1}$,
	scattering Rutherford $\theta_R$ alla profondità $d_R$
	e di nuovo multiplo scattering $\theta_\text{MS2}$.
	L'angolo a cui bisognerebbe mettere il rivelatore per osservare la traiettoria simulata è $\theta$,
	che dipende solo da $\theta_S$.}
\end{figure}

Implementiamo una simulazione 2D semplificata dell'esperimento,
illustrata in \autoref{fig:schemamc}.
Passaggi della simulazione:
\begin{enumerate}
	\item
	Partiamo con un'energia di \SI{5.46}{MeV}
	e simuliamo la perdita di energia nei \SI{3}{\micro m} di oro che incapsulano la sorgente.
	La perdita di energia è calcolata da $\mathrm dE/\mathrm dx$ tabulati\footnotemark,
	\footnotetext{NIST ASTAR \url{https://physics.nist.gov/PhysRefData/Star/Text/ASTAR.html}.}
	integrati per passi discreti di \SI{0.5}{\micro m};
	alla perdita di energia è aggiunta una variazione casuale gaussiana di deviazione standard \SI{10}\%.
	\item
	Estraiamo un angolo di uscita dalla sorgente $\theta_S$ con distribuzione uniforme
	delimitata dalla larghezza del collimatore.
	\item
	Estraiamo uniformemente la profondità nel bersaglio $d_R$ a cui faremo avvenire lo scattering Rutherford.
	\item
	Simuliamo la perdita di energia fino alla profondità $d_R$
	e deviamo la traiettoria di un angolo di multiplo scattering $\theta_\text{MS1}$
	estratto da una gaussiana di deviazione standard $\theta_0$ (\autoref{eq:ms}).
	\item
	Estraiamo il coseno dell'angolo di scattering Rutherford $\theta_R$
	dalla distribuzione $(1-\cos\theta_R)^{-1}$,
	tagliata a $\theta_R>\epsilon$ dove $\epsilon$ è un parametro della simulazione.
	Calcoliamo l'energia persa con la \eqref{eq:tout}.
	Salviamo il peso statistico $w = (T_\text{in}^2 (1-\cos\theta_R))^{-1}$
	per tenere conto della distribuzione delle energie a cui avviene lo scattering
	e della distribuzione con cui abbiamo estratto $\theta_R$.
	\item
	Simuliamo perdita di energia e multiplo scattering fino all'uscita dal bersaglio.
	\item
	Troviamo l'intersezione della traiettoria finale con la circonferenza delle possibili posizioni del rivelatore,
	da cui calcoliamo l'angolo $\theta$ che è quello che impostiamo nell'esperimento.
	\item
	Facciamo un istogramma degli eventi simulati nella variabile $\theta$,
	contando ogni evento non come +1 ma con il suo peso $w$.
	\marginpar{Aggiungere magari figura di un montecarlo.
	Senza dati sennò Morello si spaventa.}
	\marginpar{Aggiungere fattore globale $w$.}
\end{enumerate}


\section{Misura e analisi}

\subsection{Accorgimenti}

\paragraph{Angoli}
\label{spiegazione}

Nel posizionare il coperchio della camera a vuoto c'è un gioco dell'ordine di \SI{1}{\degree}.
Eliminiamo questa sistematica spingendo sempre verso un certo lato il coperchio quando chiudiamo la camera;
una volta fatto il vuoto la pressione atmosferica blocca saldamente il coperchio.
Riduciamo la parallasse nella lettura degli angoli
attaccando un pezzo di carta con una tacca al supporto della sorgente (vedi \autoref{fig:parallasse}).

\begin{figure}
	\centering
	\includegraphics[width=0.6\textwidth]{immagini/parallasse}
	\caption{\label{fig:parallasse}
	Spessore con tacca usato per eliminare la parallasse nella lettura degli angoli.}
\end{figure}

% \paragraph{Energia}
% Per acquisire al meglio i segnali amplificati dobbiamo costruire un circuito che sincronizzi il contatore e l'ADC all'inizio e alla fine di ogni acquisizione.
% Vogliamo comandare l'inizio e la fine di ogni acquisizione agendo sulla levetta di un timer.
% I conteggi vengono effettuati collegando l'uscita TTL del discriminatore ad un generatore di impulsi non retriggerabile, la cui uscita va ad un contatore. Per far partire e fermare un'acquisizione elettronicamente è sufficiente collegare l'uscita del timer allo \emph{start} del contatore e l'\emph{end marker} allo \emph{stop}.
% Per fare la stessa cosa con l'ADC colleghiamo l'uscita del timer (con durata impostata su $\infty$) ad un modulo di coincidenze a cui è connessa, ad un altro ingresso, l'uscita TTL del generatore di impulsi convertita in NIM. Il segnale di coincidenza (convertito in TTL) sarà il trigger dell'ADC, che deve arrivare all'omonimo ingresso \SI{1}{\micro s} prima del segnale affinché ne selezioni il picco.
% Concludiamo la procedura inserendo i ritardi opportuni.

\paragraph{Input ADC}

Evitiamo che segnali superiori a \SI{3.3}{V} danneggino l'ADC ponendo un attenuatore da \SI{0.9}{dB} all'uscita dell'amplificatore.

\paragraph{Interruzioni di corrente}

Quando il crate viene acceso il timer si triggera e avvia la catena di acquisizione.
Questo è utile se la corrente dovesse mancare mentre non siamo presenti.
L'ADC anche se non è connesso al computer salva i dati in un buffer interno di circa 1500 slot,
dal quale possono essere recuperati in seguito.

\subsection{Taratura trigger ADC}

Il ritardo del trigger dell'ADC
rispetto al fronte di salita del segnale discriminato
è stato tarato ponendo la sorgente a \SI{0}{\degree} senza bersaglio
e cercando di massimizzare la lettura del segnale,
in quanto l'output dell'amplificatore ha una forma a campana
con ampiezza proporzionale all'energia in ingresso.
La durata dell'amplificatore è impostata al massimo (\SI6{\micro s}).
Abbiamo acquisito lo spettro a vari angoli e verificato che non dipende dall'angolo in assenza di bersaglio. 
% \marginpar{il fit e la figura sono degli scalda posto per sapere come verrà in seguito}
% Abbiamo fittato lo spettro a \SI{0}{\degree} con una Gaussiana lasciandone libera anche la normalizzazione.\\
% Abbiamo ottenuto:
% \begin{align*}
% \mu &=3115\pm2 \\
% \sigma &= 87\pm2 \\
% N &=(1.18\pm0.02)\cdot10^5 \\
% \chi^2 &=2313\pm15 \\
% \text{dof} &=15 \\
% \end{align*}
% Il valore elevato del $\chi^2$ mostra chiaramente la non gaussianità del picco osservato.

% \begin{figure}[h]
% \centering
% \includegraphics[width=23 em]{immagini/cal_provv}
% \caption{Risoluzione in energia dell'ADC. Il pannello superiore mostra l'istogramma dei dati acquisiti con la funzione di fit, quello inferiore mostra il loro valore in funzione del tempo.}
% \label{tara}
% \end{figure}

\subsection{Discriminatore}

\subsubsection{Soglia}

Variamo la soglia del discriminatore collegato al fotodiodo e registriamo il corrispondente rate di eventi a \SI0{\degree} senza collimatori.
I valori di soglia riportati non hanno unità di misura perché sullo strumento sono solo presenti dei pallini e dei numeri interi. 
Non avendo trovato il manuale dello strumento, ci limitiamo a indicare la posizione del potenziometro.
Scopriamo (\autoref{fig:soglia}) che il valore della soglia è ininfluente fino a 0.7 e si ha una repentina variazione dopo questo valore.

\begin{figure}[h]
\centering
\includegraphics[width=30 em]{immagini/soglia}
\caption{Rate a 0$^{\circ}$ al variare della soglia del discriminatore con 2 set di dati.}
\label{fig:soglia}
\end{figure}

Per sapere se la soglia ha un effetto minore di quanto misurato dobbiamo aumentare la statistica. I rates misurati hanno (nel migliore dei casi) una precisione del 2\%.
Siccome il rate diminuisce di molto all'aumentare dell'angolo, le misure ad angoli maggiori avranno un'accuratezza minore. Quindi possiamo affermare che la soglia non avrà alcun effetto sulle misure di rate. 
Pertanto la teniamo a 0 per tutta l'esperienza.

\subsubsection{Ritardo}

Guardando all'oscilloscopio i segnali da digitalizzare, ci siamo accorti che aumentare la soglia ritarda l'arrivo del segnale di trigger rendendo inutile la calibrazione dell'ADC ogni volta che la soglia viene modificata. Abbiamo quantificato questo effetto misurando il ritardo in funzione della soglia (\autoref{tab:rit}). 

\begin{figure}[h]
\centering
\includegraphics[width=30 em]{immagini/ritardo}
\caption{Grafico che rappresenta il ritardo del segnale discriminato al variare della soglia. Le ultime misure hanno un errore maggiore a causa del jitter del segnale}
\label{fig:rit}
\end{figure}



\subsection{Risposta dello spettrometro}

\paragraph{Stabilità} Facciamo una misura lunga circa 20 ore con la sorgente a \SI{70}{\degree} senza bersaglio per valutare la stabilità in risposta dello spettrometro (ovvero il sistema rivelatore-preampli-ampli-ADC).
La scelta dell'angolo è forzata dal rate di campionamento massimo (\SI{40}{Hz}) dell'ADC.

Mostriamo in \autoref{stab} i valori campionati in funzione del tempo.
Non osserviamo nessuna scalibrazione.

\begin{figure}[h]
\centering
\includegraphics[width=23 em]{immagini/stab.png}
\caption{Valori acquisiti in funzione del tempo. Gli spazi vuoti tra una serie di dati e l'altra vengono dal fatto che sono state unite 4 acquisizioni quasi consecutive.
In questo grafico è presente soltanto un punto ogni 100 perché la rappresentazione completa rendeva illeggibile la figura.}
\label{stab}
\end{figure}

\paragraph{Forma dello spettro}

A titolo di esempio in \autoref{fig:example}
riportiamo lo spettro completo della presa dati per la verifica di stabilità.

\begin{figure}
	\centering
	\includegraphics[width=25em]{immagini/example}
	\caption{\label{fig:example}
	Spettro senza bersaglio preso durante la notte.}
\end{figure}

\paragraph{Missing codes}

Guardando l'istogramma dei dati per la stabilità (\autoref{picchi})
ci siamo accorti che l'ADC presenta degli accumuli nei canali del tipo $2^{k} - 1$ con $k\in\{1,\dots,5\}$,
dove l'accumulo è maggiore all'aumentare di $k$.
L'ADC probabilmente confronta l'input con una rampa crescente,
quando si raggiungono le varie potenze di 2 deve cambiare più cifre da 1 a 0,
impiega più tempo a cambiare valore e la lettura va nel canale precedente a quello giusto.
Per aggirare questo problema ribinniamo i dati usando i bin del tipo $(0,32]+32n$ con $n$ intero.

\begin{figure}[h]
\centering
\includegraphics[width=25 em]{immagini/rebin}
\caption{Istogramma di una parte dei dati della misura di stabilità in cui si vedono gli accumuli a cui è soggetta l'ADC e la nostra soluzione.}
\label{picchi}
\end{figure}

\paragraph{Guadagno dell'amplificatore}

Abbiamo verificato se la risposta dello spettrometro dipende come atteso dal guadagno dell'amplificatore
e abbiamo notato un comportamento inatteso dell'apparato.
Variando il guadagno dell'amplificatore agendo sulla manopola \texttt{coarse}
osserviamo gli spettri riportati in \autoref{fig:guadagno}.
La forma dello spettro non varia con un semplice fattore moltiplicativo:
la larghezza del picco è circa uguale per un guadagno di 25 e 100
e la posizione del picco non è quella attesa.
Questa dipendenza incompresa della forma dello spettro da questo parametro
non ci permette di trovare un modello fisico per la stessa.

\begin{figure}[h]
	\centering
	\includegraphics[width=30 em]{immagini/all8}
	\caption{Risposta dello spettrometro al variare del guadagno dell'amplificatore. Gli spettri non variano come atteso.}
	\label{fig:guadagno}
\end{figure}

\subsection{Pressione}

Variamo la pressione nella camera tenendo la sorgente a \SI{0}{\degree} per cercare in quali condizioni l'aria residua non influisce sulla misura.
Registriamo per ogni valore della pressione il rate di eventi ed il relativo spettro. In \autoref{tab:press} sono presenti i dati ed in \autoref{fig:press} il loro andamento.

\marginpar{qui mi riferisco agli angoli segnati sulla scala graduata, ricordiamoci che gli angoli rispetto al fotodiodo sono altri}

\marginpar{aggiungere gli errori delle mode}


Dal grafico di \autoref{fig:press} non si nota nessuna variazione dei conteggi fino ad \SI{1}{atm}, ma la moda dello spettro inizia a decrescere quando la pressione è maggiore di \SI{200}{mbar}. 
Come atteso, la distribuzione di energia persa dalle particelle $\alpha$ in aria diventa sempre più larga all'aumentare della pressione.
Questo risultato ci assicura una grande indipendenza dalla condizione di vuoto della camera, soprattutto durante le misure notturne. In tale periodo della giornata è vietato tenere la pompa accesa e, come abbiamo potuto verificare, la pressione risale fino al mbar anche dopo due giorni dalla chiusura della pompa. Quindi possiamo tranquillamente fare misure di lunga durata.

\begin{figure}[h]
\centering
\includegraphics[width=30 em]{immagini/press}
\caption{Effetti della pressione su conteggi e spettri. Il pannello superiore mostra il variare del rate al variare la pressione, quello sottostante contiene la moda dei relativi spettri con errore un intervallo del 90\% di credibilità.}
\label{fig:press}
\end{figure}



\marginpar{secondo me dobbiamo mostrare questa cosa con l'istogramma di alcuni spettri perché dà un'idea più immediata rispetto al fornire l'intervallo di credibilità}



\subsection{Forma del fascio}

Abbiamo fatto delle misure di conteggio a vari angoli con e senza collimatori per studiare la forma del fascio.

\subsubsection{Assenza di collimatori}

La misura in assenza di collimatori ci permette di indagare la forma del fascio di particelle $\alpha$ in uscita dalla sorgente di \am{}.
\marginpar{disegno provvisorio}
Bisogna notare che l'angolo segnato sulla scala graduata non coincide con quello tra la sorgente ed il rivelatore perché il fotodiodo non è al centro della camera a vuoto, inoltre all'aumentare (in modulo) dell'angolo la distanza tra sorgente e rivelatore diminuisce. 
Lo schema di \autoref{fma} illustra la situazione.

\begin{figure}[h]
\centering
\includegraphics[width=30 em]{immagini/fma_provv}
\caption{Schema raffigurante gli angoli tra rivelatore, centro della camera a vuoto e sorgente.}
\label{fma}
\end{figure}

Applicate le dovute correzioni e usando le variabili di \autoref{fma} troviamo che l'angolo tra rivelatore e sorgente è
\begin{equation}
\cos{\alpha}= -\frac{\vec{X} \cdot \vec{D} }{ |\vec{X}| |\vec{D}| } = \frac{ L \cos{\theta} + D }{ \sqrt{ L^2+2LD\cos{\theta}+D^2  } }.
\end{equation}

Teniamo conto della variazione della distanza al variare dell'angolo moltiplicando ogni rate per $|\vec{X}|^2$.
Tale fattore è giustificato dal fatto che supponiamo isotropa l'emissione di particelle $\alpha$ da parte della sorgente.
Se $C$ è una costante, abbiamo che $$ \frac{dN}{d^3r}=C \implies \frac{dN}{dr}=C 4\pi r^2. $$

In virtù delle precedenti affermazioni, abbiamo calcolato il rate in \si{s^{-1}cm^2}.

Il risultato di questa misura è presente in \autoref{fig:forma}, invece la \autoref{tab:forma} contiene i dati di tale grafico.
Si evince che il fascio ha un'estensione angolare di quasi \SI{50}{\degree}. 

\begin{figure}[h]
\centering
\includegraphics[width=30 em]{immagini/forma}
\caption{Grafico rappresentante la forma del fascio. Nel pannello superiore è presente l'estensione angolare del fascio estrapolata dai conteggi, in quello inferiore sono presenti le mode dei relativi istogrammi.}
\label{fig:forma}
\end{figure}

\begin{table}[h]
\centering

\begin{tabular}{c|c|c}
angolo [\si{\degree}] & rate  [\si{s^{-1}cm^2}] & moda [digit] \\
\hline
$ -42.6 \pm 1.5 $ & $ 5.2 \pm 1.3 $ & $ 1256 $ \\ 
$ -38.0 \pm 1.3 $ & $ 257 \pm 16 $ & $ 2993 $ \\ 
$ -33.3 \pm 1.1 $ & $ 538 \pm 31 $ & $ 3048 $ \\ 
$ -31.0 \pm 1.0 $ & $ 6.6 \pm 0.4   \cdot 10^{2} $ & $ 3070 $ \\ 
$ -28.6 \pm 1.0 $ & $ 8.1 \pm 0.5   \cdot 10^{2} $ & $ 3087 $ \\ 
$ -26.2 \pm 0.9 $ & $ 9.7 \pm 0.5   \cdot 10^{2} $ & $ 3105 $ \\ 
$ -23.9 \pm 0.8 $ & $ 1.01 \pm 0.06 \cdot 10^{3} $ & $ 3113 $ \\ 
$ -21.5 \pm 0.8 $ & $ 1.10 \pm 0.06 \cdot 10^{3} $ & $ 3119 $ \\ 
$ -19.1 \pm 0.7 $ & $ 1.22 \pm 0.07 \cdot 10^{3} $ & $ 3129 $ \\ 
$ -16.7 \pm 0.7 $ & $ 1.22 \pm 0.07 \cdot 10^{3} $ & $ 3135 $ \\ 
$ -14.4 \pm 0.6 $ & $ 1.37 \pm 0.08 \cdot 10^{3} $ & $ 3134 $ \\ 
$ -12.0 \pm 0.6 $ & $ 1.39 \pm 0.08 \cdot 10^{3} $ & $ 3142 $ \\ 
$ -9.6 \pm 0.5 $ & $ 1.43 \pm 0.08  \cdot 10^{3} $ & $ 3144 $ \\ 
$ -7.2 \pm 0.5 $ & $ 1.45 \pm 0.08  \cdot 10^{3} $ & $ 3147 $ \\ 
$ -4.8 \pm 0.5 $ & $ 1.47 \pm 0.08  \cdot 10^{3} $ & $ 3145 $ \\ 
$ -2.4 \pm 0.5 $ & $ 1.55 \pm 0.08  \cdot 10^{3} $ & $ 3148 $ \\ 
$ 0.0 \pm 0.5 $ & $ 1.56 \pm 0.09   \cdot 10^{3} $ & $ 3149 $ \\ 
$ 2.4 \pm 0.5 $ & $ 1.56 \pm 0.09   \cdot 10^{3} $ & $ 3145 $ \\ 
$ 4.8 \pm 0.5 $ & $ 1.66 \pm 0.09   \cdot 10^{3} $ & $ 3141 $ \\ 
$ 7.2 \pm 0.5 $ & $ 1.59 \pm 0.09   \cdot 10^{3} $ & $ 3140 $ \\ 
$ 9.6 \pm 0.5 $ & $ 1.56 \pm 0.09   \cdot 10^{3} $ & $ 3137 $ \\ 
$ 12.0 \pm 0.6 $ & $ 1.45 \pm 0.08  \cdot 10^{3} $ & $ 3130 $ \\ 
$ 14.4 \pm 0.6 $ & $ 1.49 \pm 0.08  \cdot 10^{3} $ & $ 3125 $ \\ 
$ 16.7 \pm 0.7 $ & $ 1.36 \pm 0.08  \cdot 10^{3} $ & $ 3114 $ \\ 
$ 19.1 \pm 0.7 $ & $ 1.32 \pm 0.07  \cdot 10^{3} $ & $ 3101 $ \\ 
$ 21.5 \pm 0.8 $ & $ 1.28 \pm 0.07  \cdot 10^{3} $ & $ 3095 $ \\ 
$ 23.9 \pm 0.8 $ & $ 1.23 \pm 0.07  \cdot 10^{3} $ & $ 3087 $ \\ 
$ 26.2 \pm 0.9 $ & $ 1.19 \pm 0.07  \cdot 10^{3} $ & $ 3066 $ \\ 
$ 28.6 \pm 1.0 $ & $ 1.04 \pm 0.06  \cdot 10^{3} $ & $ 3047 $ \\ 
$ 31.0 \pm 1.0 $ & $ 9.5 \pm 0.5    \cdot 10^{2} $ & $ 3022 $ \\ 
$ 33.3 \pm 1.1 $ & $ 8.2 \pm 0.5    \cdot 10^{2} $ & $ 3002 $ \\ 
$ 35.7 \pm 1.2 $ & $ 6.5 \pm 0.4    \cdot 10^{2} $ & $ 2966 $ \\ 
$ 38.0 \pm 1.3 $ & $ 531 \pm 30 $ & $ 2937 $ \\ 
$ 40.3 \pm 1.4 $ & $ 374 \pm 21 $ & $ 2892 $ \\ 
$ 42.6 \pm 1.5 $ & $ 237 \pm 14 $ & $ 2840 $ \\ 
$ 44.9 \pm 1.6 $ & $ 94 \pm 6 $ & $ 2717 $ \\ 
$ 47.1 \pm 1.8 $ & $ 2.6 \pm 0.9 $ & $ 964 $ \\ 
\end{tabular}

\caption{Dati utilizzati per trovare la forma del fascio. L'errore sulla moda è dato dal 90\% di credibilità del relativo istogramma.}
\label{tab:forma}
\end{table}

Useremo queste informazioni (se sarà necessario) nell'analizzare i dati sulla sezione d'urto.

\subsubsection{Collimatore da 5\! mm}

Analizziamo la forma del fascio con un collimatore da \SI{5}{mm} che ci dà un errore minore sull'angolo di incidenza delle particelle $\alpha$.
Con i dati presenti in \autoref{tab:coll5} abbiamo il grafico di \autoref{fig:coll5}.

In questo caso i conteggi sono non nulli soltanto nell'intervallo \SI{\mp10}{\degree}. La parte piatta al centro del grafico è dovuta all'effetto descritto nella sezione precedente (non corretto), ancora visibile con un'apertura di tale larghezza.

\begin{figure}[h]
\centering
\includegraphics[width=30 em]{immagini/coll5}
\caption{Grafico del rate in funzione dell'angolo con il collimatore da 5\! mm.}
\label{fig:coll5}
\end{figure}

\begin{table}[h]
\centering

\begin{tabular}{c|c|c}
angolo [\si{\degree}] & rate  [\si{s^{-1}}] & moda [digit] \\
\hline
$ -15.00 $ & $ 0.0 \pm 0 $ & $ 0 $ \\ 
$ -10.00 $ & $ 2.76 \pm 0.18 $ & $ 3177 $ \\ 
$ -7.50 $ & $ 27.8 \pm 0.6 $ & $ 3183 $ \\ 
$ -5.00 $ & $ 42.4 \pm 0.9 $ & $ 3186 $ \\ 
$ -2.50 $ & $ 43.9 \pm 1.0 $ & $ 3181 $ \\ 
$ 0.00 $ & $ 44.4 \pm 0.9 $ & $ 3185 $ \\ 
$ 2.50 $ & $ 44.7 \pm 1.0 $ & $ 3187 $ \\ 
$ 5.00 $ & $ 47.1 \pm 1.0 $ & $ 3182 $ \\ 
$ 7.50 $ & $ 43.2 \pm 0.9 $ & $ 3180 $ \\ 
$ 10.00 $ & $ 15.6 \pm 0.4 $ & $ 3188 $ \\ 
$ 15.00 $ & $ 0.0 \pm 0 $ & $ 0 $ \\ 

\end{tabular}

\caption{Rate in funzione dell'angolo con il collimatore da 5\! mm.
		L'errore sugli angoli è stimato essere 1.25$^{\circ}$.}
\label{tab:coll5}
\end{table}

\subsubsection{Collimatore da 1\! mm}

Studiamo la forma del fascio con il collimatore da \SI1{mm} che ci permette di avere il più piccolo errore possibile sull'angolo di incidenza.
Di contro abbiamo un rate minore che nei casi precedenti.
I dati presenti in \autoref{tab:coll1} e \autoref{fig:coll1} sono stati analizzati come in precedenza.
Stavolta abbiamo conteggi solo in un intervallo di \SI{5}{\degree}.

\begin{table}[h]
\centering

\begin{tabular}{c|c|c}
angolo [\si{\degree}] & rate  [\si{s^{-1}}] & moda [digit] \\
\hline
$ -2.50 $ & $ 0.08 \pm 0.04 $ & $ 3156 $ \\ 
$ -1.25 $ & $ 2.48 \pm 0.20 $ & $ 3171 $ \\ 
$ 0.00 $ & $ 14.63 \pm 0.32 $ & $ 3174 $ \\ 
$ 1.25 $ & $ 31.5 \pm 0.7 $ & $ 3178 $ \\ 
$ 2.50 $ & $ 34.8 \pm 0.8 $ & $ 3185 $ \\ 
$ 3.75 $ & $ 26.9 \pm 0.6 $ & $ 3184 $ \\ 
$ 5.00 $ & $ 12.70 \pm 0.31 $ & $ 3188 $ \\ 
$ 6.25 $ & $ 2.81 \pm 0.22 $ & $ 3193 $ \\ 
$ 7.50 $ & $ 0.11 \pm 0.04 $ & $ 3185 $ \\ 

\end{tabular}

\caption{Dati per le misure con il collimatore da 1\! mm.}
\label{tab:coll1}
\end{table}


\begin{figure}[h]
\centering
\includegraphics[width=30 em]{immagini/coll1}
\caption{Grafico del rate in funzione dell'angolo con il collimatore da 1\! mm.}
\label{fig:coll1}
\end{figure}

\subsubsection{Collimatore a croce}

Cerchiamo infine di collimare il fascio il più possibile sovrapponendo 2 collimatori da \SI1{mm} di cui uno ruotato di \SI{90}{\degree} rispetto all'altro. In questo modo abbiamo una piccola finestra che seleziona le particelle sia verticalmente che orizzontalmente.
Prendiamo le misure in \autoref{tab:super} e guardiamo il grafico di \autoref{fig:super}.

\begin{table}
\centering

\begin{tabular}{c|c|c}
angolo [\si{\degree}] & rate  [\si{s^{-1}}] & moda [digit] \\
\hline
$ -2.50 $ & $ 0.27 \pm 0.04 $ & $ 3213 $ \\ 
$ -1.25 $ & $ 1.20 \pm 0.11 $ & $ 3198 $ \\ 
$ 0.00 $ & $ 2.36 \pm 0.15 $ & $ 3190 $ \\ 
$ 1.25 $ & $ 2.56 \pm 0.16 $ & $ 3187 $ \\ 
$ 2.50 $ & $ 2.41 \pm 0.15 $ & $ 3192 $ \\ 
$ 3.75 $ & $ 1.32 \pm 0.09 $ & $ 3187 $ \\ 
$ 5.00 $ & $ 0.32 \pm 0.05 $ & $ 3193 $ \\ 

\end{tabular}

\caption{Dati per le misure con il collimatore a croce.}
\label{tab:super}
\end{table}

\begin{figure}[h]
\centering
\includegraphics[width=30 em]{immagini/supercoll}
\caption{Grafico del rate in funzione dell'angolo con il collimatore a croce.}
\label{fig:super}
\end{figure}

Questa configurazione è ottimale per avere un fascio collimato, ma non la useremo a causa del rate incredibilmente basso.





\subsection{Rumore}

Verifichiamo che il circuito può venire triggerato da scariche elettriche nei dintorni.
Ciò avviene sistematicamente azionando interruttori in potenza
(cioè non interruttori elettronici come ad esempio dei computer)
o collegando spine vicino all'apparato,
a distanza maggiore (ad esempio l'interruttore della luce) più di rado.
Inoltre viene triggerato accendendo i neon,
questo effetto però è ottico perché coprendo la camera con un panno nero l'effetto si annulla.

Confrontando gli eventi salvati dall'ADC con il contatore
e provocando intenzionalmente rumori, osserviamo che:
\begin{itemize}
	\item
	Ci sono rumori ``intensi'' che producono più di un conteggio
	e spesso il contatore segnala più eventi di quelli salvati dall'ADC, mai di meno.
	I conteggi multipli hanno lo stesso timestamp.
	\item
	Quasi sempre (sicuramente più del \SI{99.9}\percent delle volte)
	il canale 2 dell'ADC, che non usiamo ma che viene letto, segna 0.
	Le letture diverse da zero sono correlate, ma non totalmente, con i rumori intensi.
	\item
	Le 
\end{itemize}

\subsection{Misure}

Abbiamo eseguito le misure di rate al variare dell'angolo
con le lamine di oro da \SI{3}{\micro m} e \SI5{\micro m} e con quella di alluminio da \SI8{\micro m},
sia con il collimatore da \SI1{mm} che da \SI5{mm}.
In quasi tutte le misure abbiamo salvato lo spettro.




\subsection{Fit}

Misuriamo il rate di eventi al variare dell'angolo con i seguenti materiali:
\begin{itemize}
\item oro \SI{3}{\micro m}
\item oro \SI5{\micro m}
\item alluminio \SI8{\micro m}.
\end{itemize}




\subsubsection{Z dell'alluminio}

Comparando i parametri di ampiezza dei fit con l'oro con quello eseguito sull'alluminio, possiamo estrarre lo Z di quest'ultimo. Le ampiezze di fit hanno la forma $$ B=\mathcal{L} \left( \frac {zZ\alpha\hbar c} {2T} \right)^2 = A \mathcal{L} $$.
La nostra luminosità vale $\mathcal{L}=n_1 n_2 l$, in cui $n_1$ è il rate di particelle incidenti, $n_2$  è la densità di bersagli e $l$ è lo spessore della targhetta.
Nel nostro caso $n_1$ è uguale per tutte le misure con lo stesso collimatore.   \marginpar{c'è bisogno di precisare perché?}
Da queste considerazioni possiamo trarre lo Z dell'alluminio dalla relazione \eqref{zeta} usando i parametri di fit per i dati con lo stesso collimatore.
\begin{equation}
Z_{\text{al}}=Z_{\text{au}} \sqrt{ \frac{B_{\text{al}}}{B_{\text{au}}} \frac{n_{\text{au}} l_{\text{au}}}{n_{\text{al}} l_{\text{al}}} }
\label{zeta}
\end{equation}

\subsection{Backscattering}

Poniamo le nostre lamine a \SI{150}{\degree} (il valore più grande sulla scala graduata) per verificare la presenza di backscattering. Usiamo 2 lamine attaccate per essere sicuri che nessuna particella attraversi il bersaglio. Questo accorgimento ci permette di evitare che delle particelle $\alpha$ entrino nel rivelatore dopo essere rimbalzate sulle pareti. Abbiamo controllato che questa ipotesi sia sensata contando le particelle che attraversano 2 lamine dello stesso materiale (e spessore) a \SI{0}{\degree}.

Facendo misure con \SI{40}{\micro m} di oro non abbiamo osservato nessun evento nei \SI{1406}s\,$\approx$\,\SI{23}{minuti} della misura. La situazione è diversa per \SI{16}{\micro m} di alluminio: abbiamo un rate di \SI{0.57(8)}{s^{-1}}.
Questo risultato però non è il nostro fondo perché nelle misure a \SI{150}{\degree} sarà rappresentato da quelle particelle $\alpha$ che, rimbalzando sulle pareti, arrivano nel fotodiodo. Sappiamo anche che il fondo maggiore è rappresentato dalle particelle che arrivano al rivelatore dopo aver rimbalzato sui supporti di plastica delle lamine.

Abbiamo quantificato quest'ultimo effetto ponendo un supporto senza lamina al centro della camera e misurando il rate a \SI{150}{\degree}. Otteniamo $R_{\text{buco}}=\SI{3.9(6)e-4}{s^{-1}}$.
Siccome in presenza dell'alluminio le particelle che raggiungono la parete opposta alla sorgente sono ancora meno che nella situazione precedente, possiamo affermare che il fondo è minore di quello misurato. Ci aspettiamo che il rate di fondo sia trascurabilmente minore di $R_{\text{buco}}$ sia per motivi geometrici che dinamici. La particella $\alpha$ in questione dovrebbe colpire un piccolo rivelatore rimbalzando in una grande camera a vuoto e la sezione d'urto differenziale Rutherford decresce molto velocemente per angoli inferiori a \SI{10}{\degree}. Inoltre, nei rimbalzi ad angolo maggiore, la perdita di energia è più elevata. Questi meccanismi sfavoriscono diffusioni multiple all'interno della camera.

Di seguito riportiamo i risultati delle nostre misure di backscattering:
\begin{align*}
R_{\text{oro}} &= \SI{1.3(2)e-3}{s^{-1}} \\
R_{\text{alluminio}} &= \SI{5.4(6)e-4}{s^{-1}} \\
R_{\text{fondo}} \simeq R_{\text{buco}} &= \SI{3.9(6)e-4}{s^{-1}} \\
\text{differenza oro-fondo} &= \SI{5.9(4)}{\sigma} \\
\text{differenza alluminio-fondo} &= \SI{2.6(11)}{\sigma} \\
\end{align*}

Possiamo affermare in modo indiscutibile l'esistenza del backscattering nel caso dell'oro, ma non possiamo fare altrettanto con l'alluminio, almeno in questa misura.
Per poter osservare un indiscutibile backscattering anche con l'alluminio, abbiamo posto 2 dissipatori di un pc in prossimità del fotodiodo e la sorgente a \SI{150}{\degree}.
\marginpar{inserire dimensioni dissipatori}
Le dimensioni dei dissipatori impediscono alle particelle $\alpha$ di attraversarli e sono colpiti dalla quasi totalità del fascio, restituendoci una misura praticamente priva di fondo.
L'autopassivazione dell'alluminio crea uno strato di ossido spesso da 1 a \SI5{nm}, che viene portato a 50--\SI{200}{nm} con processi industriali.%
\footnote{\url{http://www.valocchi.eu/logistica/alluminio.htm}}
Data la sottigliezza di tale spessore, consideriamo trascurabile la variazione del rate dovuta a questo strato che è dovuta alla diversa densità dell'ossido e alla presenza dei nuclei di ossigeno.

\marginpar{Forse è meglio scrivere questa cosa all'inizio perché l'abbiamo trascurata anche per la lamina sottile.}

Abbiamo ottenuto \SI{54(7)}{eventi} in circa \SI{6.5}{ore}.

\paragraph{Perdita di energia per scattering all'indietro} Dalla \autoref{eq:tout}, per uno scattering a \SI{150}{\degree}) ci aspetteremmo che le particelle $\alpha$ arrivino al rivelatore con un energia pari rispettivamente al $\sim$ 45\% e 92\% dell'energia cinetica iniziale per diffusione su alluminio e oro. Bisogna tener conto del fatto che essendo le lamine spesse (rispetto alla lunghezza di radiazione) lo scattering Rutherford può avvenire in profondità per cui l'energia osservata non è quella attesa per il singolo urto ma sostanzialmente minore a causa dell'energia persa nel mezzo per interazione con gli elettroni. Gli spettri osservati per gli eventi di backscattering su oro e alluminio sono mostrati in \autoref{fig:backscattering}.
La bassa statistica non permette di concludere molto: gli spettri sembrano piccare entrambi alla stessa energia: questo può essere qualitativamente spiegato dal fatto che sebbene le particelle $\alpha$ dopo lo scattering su oro abbiano circa il doppio dell'energia rispetto all'urto su alluminio, la lamina di oro usata è più spessa e l'energia persa in unità di lunghezza maggiore rispetto all'alluminio.

\begin{figure}[h]
	\centering
	\includegraphics[width=30 em]{immagini/backscattering}
	\caption{Spettro delle particelle diffuse all'indietro per oro e alluminio. In verde lo spettro degli $\alpha$ dell'americio provenienti dalla nostra sorgente.}
	\label{fig:backscattering}
\end{figure}

\appendix
%\appendix
\section{Conti}
\label{sec:conti}

Calcoliamo la sezione d'urto Rutherford nel caso di nucleo non fisso.
La formula con nucleo fisso è
\begin{equation}
	\label{eq:ruthfisso}
	\dv\sigma\Omega = \left( \frac {zZ\alpha\hbar c} {2T(1-\cos\theta)} \right)^2.
\end{equation}
Il problema con il nucleo non fisso si rinconduce a quello fisso,
che chiamiamo <<problema ridotto>>, con queste sostituzioni:
\begin{center}
	\begin{tabular}{ll}
		Nucleo non fisso                 & Nucleo fisso               \\
		\hline
		Distanza tra particella e nucleo & Posizione della particella \\
		Massa della particella           & Massa ridotta              
	\end{tabular}
\end{center}
La \eqref{eq:ruthfisso} è da intendersi scritta nelle variabili del problema ridotto,
che dobbiamo esprimere in termini di quelle non ridotte.
Sia $\vec v$ la velocità della particella $\alpha$ (di massa $m$),
sia $\vec V$ la velocità del nucleo (di massa $M$), sia
\begin{equation*}
	\mu = \frac{mM}{m + M}
\end{equation*}
la massa ridotta.
Indichiamo con il pedice ``rid'' le variabili del problema ridotto.
La velocità ridotta è
\begin{equation*}
	\vec v_\text{rid} = \dv{}t(\vec r_\alpha - \vec r_\text{nucl}) = \vec v - \vec V.
\end{equation*}
Per conservazione dell'energia nel problema ridotto
$|\vec v_\text{rid,in}| = |\vec v_\text{rid,out}|$,
inoltre
$\vec v_\text{rid,in}
= \vec v_\text{in} - \vec V_\text{in}
= \vec v_\text{in}$.
Calcoliamo l'energia ridotta:
\begin{align*}
	\begin{cases}
		T_\text{rid} = \frac12 \mu v_\text{rid}^2 = \frac12 \mu v_\text{in}^2 \\
		T_\text{in}  = \frac12 m v_\text{in}^2
	\end{cases} \implies
	T_\text{rid} = \frac\mu m T_\text{in} = \frac1{1+\frac mM} T_\text{in},
\end{align*}
dunque abbiamo $T_\text{rid}(T_\text{in})$.
Ora ricaviamo $T_\text{out}$ in funzione di $T_\text{in}$ e $\cos\theta$.
Equivalentemente ricaviamo la relazione per le velocità anziché per le energie.
\begin{align*}
	\begin{cases}
		\vec v_\text{rid,out} = \vec v_\text{out} - \vec V_\text{out} \text{ (def. di $\vec v_\text{rid}$)}\\
		m \vec v_\text{in} = m \vec v_\text{out} + M \vec V_\text{out} \text{ (conservazione impulso)}
	\end{cases}
\end{align*}
ricaviamo $\vec V_\text{out}$ dalla seconda equazione:
\begin{align*}
	\vec V_\text{out} = \frac mM (\vec v_\text{in} - \vec v_\text{out})
\end{align*}
sostituiamo nella prima:
\begin{align}
	\label{eq:star}
	\vec v_\text{rid,out}
	= \vec v_\text{out} \left(1 + \frac mM\right) - \frac mM \vec v_\text{in}
\end{align}
prendiamo il modulo nell'ultima equazione:
\begin{align*}
	v_\text{in}^2
	= v_\text{out}^2 \left(1 + \frac mM\right)^2 + \left(\frac mM\right)^2 v_\text{in}^2
	- 2 \frac mM \left(1 + \frac mM\right) \vec v_\text{in} \vec v_\text{out}
\end{align*}
ma $\vec v_\text{in}\vec v_\text{out} = v_\text{in} v_\text{out} \cos\theta$.
Risolviamo per $v_\text{out}$:
\begin{align*}
	0
	&= v_\text{out}^2 \left(1 + \frac mM\right)^2
	- v_\text{out} 2\frac mM\left(1 + \frac mM\right) v_\text{in} \cos\theta
	- v_\text{in}^2 \left(1 - \left(\frac mM\right)^2 \right) = \\
	&= \left(1 + \frac mM\right) \Bigg(
   v_\text{out}^2 \left(1 + \frac mM\right)
  	- v_\text{out} 2\frac mM v_\text{in} \cos\theta
  	- v_\text{in}^2 \left(1 - \frac mM\right)
	\Bigg) \implies \\
	\implies v_\text{out}
	&= v_\text{in} \frac
	{\frac mM \cos\theta + \sqrt{\left(\frac mM\right)^2 \cos^2\theta + \left(1-\frac mM\right)\left(1+\frac mM\right)}}
	{1 + \frac mM} = \\
	&= v_\text{in} \frac
	{\frac mM \cos\theta + \sqrt{1 - \left(\frac mM\right)^2 (1-\cos^2\theta)}}
	{1 + \frac mM} = \\
	&= v_\text{in} \frac
	{\sqrt{1 - \left(\frac mM\sin\theta\right)^2} + \frac mM\cos\theta}
	{1 + \frac mM}
\end{align*}
dunque abbiamo $v_\text{out}(v_\text{in},\cos\theta)$.
Ora ricaviamo $\cos\theta_\text{rid}$ in funzione di $\cos\theta$.
Prendiamo la proiezione su $\vec v_\text{in}$ della \eqref{eq:star}:
\begin{align*}
	\cos\theta_\text{rid}
	&= \frac{v_\text{out}}{v_\text{in}} \cos\theta \left(1+\frac mM\right) - \frac mM = \\
	&= \left( \sqrt{1 - \left(\frac mM \sin\theta\right)^2} + \frac mM \cos\theta \right) \cos\theta - \frac mM.
\end{align*}


\clearpage
\section{Dati}

\begin{table}[h]
\centering
\begin{tabular}{c|c|c}

pressione [mbar] & rate [\si{s^{-1}}] & moda [digit] \\
\hline
$ 0.0190 \pm 0.0010 $ & $ 44.6 \pm 0.9 $ & $ 3183 $ \\ 
$ 0.180 \pm 0.010 $ & $ 46.4 \pm 1.0 $ & $ 3182 $ \\ 
$ 1.00 \pm 0.20 $ & $ 44.8 \pm 0.9 $ & $ 3181 $ \\ 
$ 8.90 \pm 0.10 $ & $ 45.4 \pm 0.8 $ & $ 3172 $ \\ 
$ 23.0 \pm 1.0 $ & $ 44.9 \pm 0.9 $ & $ 3153 $ \\ 
$ 63.0 \pm 1.0 $ & $ 43.1 \pm 0.8 $ & $ 3103 $ \\ 
$ 81.0 \pm 1.0 $ & $ 44.0 \pm 0.7 $ & $ 3078 $ \\ 
$ 110 \pm 10 $ & $ 47.1 \pm 1.0 $ & $ 3049 $ \\ 
$ 150 \pm 10 $ & $ 44.4 \pm 0.9 $ & $ 3027 $ \\ 
$ 220 \pm 10 $ & $ 43.4 \pm 0.8 $ & $ 2984 $ \\ 
$ 470 \pm 10 $ & $ 45.9 \pm 0.9 $ & $ 2787 $ \\ 
$ 700 \pm 10 $ & $ 41.7 \pm 0.9 $ & $ 1712 $ \\ 
$ 860 \pm 10 $ & $ 0.13 \pm 0.04 $ & $ 521 $ \\ 

\end{tabular}
\caption{Dati per stimare l'effetto della pressione sulle misure. L'errore sulla moda è dato dall'intervallo di credibilità del 90\% calcolato sul relativo istogramma.}
\label{tab:press}
\end{table}


\begin{table}[h]
\centering

\begin{tabular}{c|c}
soglia & rate [\si{s^{-1}}] \\
\hline
$ 0.0 $ & $ 45.4 \pm 1.5 $\\ 
$ 0.5 $ & $ 45.0 \pm 1.5 $\\ 
$ 0.7 $ & $ 43.2 \pm 2.1 $\\ 
$ 0.8 $ & $ 38 \pm 2 $\\ 
$ 0.9 $ & $ 10 \pm 1 $\\ 
$ 1.0 $ & $\;\; 0.05 \pm 0.05 $\\ 
\hline
$ 0.0 $ & $ 43.7 \pm 0.9 $\\ 
$ 0.1 $ & $ 44 \pm 1 $\\ 
$ 0.2 $ & $ 45 \pm 1 $\\ 
$ 0.3 $ & $ 43.9 \pm 0.8 $\\ 
$ 0.4 $ & $ 43.6 \pm 0.9 $\\ 
$ 0.5 $ & $ 41.9 \pm 0.9 $\\ 
$ 0.6 $ & $ 42.0 \pm 0.9 $\\ 
$ 0.7 $ & $ 44.0 \pm 0.9 $\\ 
$ 0.8 $ & $ 36.5 \pm 0.7 $\\ 
$ 0.9 $ & $ 5.6 \pm 0.2 $\\ 
$ 1.0 $ & $ 0.05 \pm 0.03 $\\ 

\end{tabular}

\caption{Valori presenti nel grafico di \autoref{fig:soglia}. La linea orizzontale divide quelli del 15 marzo (in alto) da quelli del 19 marzo (in basso).}
\label{tab:soglia}
\end{table}



\begin{table}[h]
\centering

\begin{tabular}{c|c @{\,$\pm$\,} l}
soglia & \multicolumn{2}{c}{ritardo [\si{ns}]} \\
\hline
$ 0.0 $ & $ 88 $ & $ 2 $\\ 
$ 0.1 $ & $ 88 $ & $ 2 $\\ 
$ 0.2 $ & $ 116 $ & $ 2 $\\ 
$ 0.3 $ & $ 180 $ & $ 2 $\\ 
$ 0.4 $ & $ 288 $ & $ 2 $\\ 
$ 0.5 $ & $ 384 $ & $ 2 $\\ 
$ 0.6 $ & $ 464 $ & $ 4 $\\ 
$ 0.7 $ & $ 560 $ & $ 10 $\\ 
$ 0.8 $ & $ 720 $ & $ 60 $\\ 
$ 0.9 $ & $ 910 $ & $ 60 $\\ 
\end{tabular}

\caption{Dati per la misura in cui si quantifica il ritardo del segnale discriminato in funzione della soglia del discriminatore.
Le ultime misure hanno un errore maggiore a causa del jitter del segnale.}
\label{tab:rit}
\end{table}


\begin{table}[h]
\hspace{-3em}
\begin{tabular}{c|c|c||c|c|c}
	angolo [\si{\degree}] & rate  [\si{s^{-1}cm^2}] & moda [digit] & angolo [\si{\degree}] & rate  [\si{s^{-1}cm^2}] & moda [digit] \\
	\hline
	$ -42.6 \pm 1.5 $ & $ 5.2 \pm 1.3 $ & $ 1256 $                & $ 7.2 \pm 0.5 $ & $ 1.59 \pm 0.09   \cdot 10^{3} $ & $ 3140 $ \\  
	$ -38.0 \pm 1.3 $ & $ 257 \pm 16 $ & $ 2993 $                 & $ 9.6 \pm 0.5 $ & $ 1.56 \pm 0.09   \cdot 10^{3} $ & $ 3137 $ \\ 
	$ -33.3 \pm 1.1 $ & $ 538 \pm 31 $ & $ 3048 $                 & $ 12.0 \pm 0.6 $ & $ 1.45 \pm 0.08  \cdot 10^{3} $ & $ 3130 $ \\ 
	$ -31.0 \pm 1.0 $ & $ 6.6 \pm 0.4   \cdot 10^{2} $ & $ 3070 $ & $ 14.4 \pm 0.6 $ & $ 1.49 \pm 0.08  \cdot 10^{3} $ & $ 3125 $ \\ 
	$ -28.6 \pm 1.0 $ & $ 8.1 \pm 0.5   \cdot 10^{2} $ & $ 3087 $ & $ 16.7 \pm 0.7 $ & $ 1.36 \pm 0.08  \cdot 10^{3} $ & $ 3114 $ \\ 
	$ -26.2 \pm 0.9 $ & $ 9.7 \pm 0.5   \cdot 10^{2} $ & $ 3105 $ & $ 19.1 \pm 0.7 $ & $ 1.32 \pm 0.07  \cdot 10^{3} $ & $ 3101 $ \\ 
	$ -23.9 \pm 0.8 $ & $ 1.01 \pm 0.06 \cdot 10^{3} $ & $ 3113 $ & $ 21.5 \pm 0.8 $ & $ 1.28 \pm 0.07  \cdot 10^{3} $ & $ 3095 $ \\ 
	$ -21.5 \pm 0.8 $ & $ 1.10 \pm 0.06 \cdot 10^{3} $ & $ 3119 $ & $ 23.9 \pm 0.8 $ & $ 1.23 \pm 0.07  \cdot 10^{3} $ & $ 3087 $ \\ 
	$ -19.1 \pm 0.7 $ & $ 1.22 \pm 0.07 \cdot 10^{3} $ & $ 3129 $ & $ 26.2 \pm 0.9 $ & $ 1.19 \pm 0.07  \cdot 10^{3} $ & $ 3066 $ \\ 
	$ -16.7 \pm 0.7 $ & $ 1.22 \pm 0.07 \cdot 10^{3} $ & $ 3135 $ & $ 28.6 \pm 1.0 $ & $ 1.04 \pm 0.06  \cdot 10^{3} $ & $ 3047 $ \\ 
	$ -14.4 \pm 0.6 $ & $ 1.37 \pm 0.08 \cdot 10^{3} $ & $ 3134 $ & $ 31.0 \pm 1.0 $ & $ 9.5 \pm 0.5    \cdot 10^{2} $ & $ 3022 $ \\ 
	$ -12.0 \pm 0.6 $ & $ 1.39 \pm 0.08 \cdot 10^{3} $ & $ 3142 $ & $ 33.3 \pm 1.1 $ & $ 8.2 \pm 0.5    \cdot 10^{2} $ & $ 3002 $ \\ 
	$ -9.6 \pm 0.5 $ & $ 1.43 \pm 0.08  \cdot 10^{3} $ & $ 3144 $ & $ 35.7 \pm 1.2 $ & $ 6.5 \pm 0.4    \cdot 10^{2} $ & $ 2966 $ \\ 
	$ -7.2 \pm 0.5 $ & $ 1.45 \pm 0.08  \cdot 10^{3} $ & $ 3147 $ & $ 38.0 \pm 1.3 $ & $ 531 \pm 30 $ & $ 2937 $ \\  
	$ -4.8 \pm 0.5 $ & $ 1.47 \pm 0.08  \cdot 10^{3} $ & $ 3145 $ & $ 40.3 \pm 1.4 $ & $ 374 \pm 21 $ & $ 2892 $ \\  
	$ -2.4 \pm 0.5 $ & $ 1.55 \pm 0.08  \cdot 10^{3} $ & $ 3148 $ & $ 42.6 \pm 1.5 $ & $ 237 \pm 14 $ & $ 2840 $ \\  
	$ 0.0 \pm 0.5 $ & $ 1.56 \pm 0.09   \cdot 10^{3} $ & $ 3149 $ & $ 44.9 \pm 1.6 $ & $ 94 \pm 6 $ & $ 2717 $ \\  
	$ 2.4 \pm 0.5 $ & $ 1.56 \pm 0.09   \cdot 10^{3} $ & $ 3145 $ & $ 47.1 \pm 1.8 $ & $ 2.6 \pm 0.9 $ & $ 964 $ \\  
	$ 4.8 \pm 0.5 $ & $ 1.66 \pm 0.09   \cdot 10^{3} $ & $ 3141 $ & $ 7.2 \pm 0.5 $ & $ 1.59 \pm 0.09   \cdot 10^{3} $ & $ 3140 $ \\ 
\end{tabular}

\caption{Dati utilizzati per trovare la forma del fascio. L'errore sulla moda è dato dal 90\% di credibilità del relativo istogramma.}
\label{tab:forma}
\end{table}


\begin{table}[h]
\centering

\begin{tabular}{c|c|c}
angolo [\si{\degree}] & rate  [\si{s^{-1}}] & moda [digit] \\
\hline
$ -15.00 $ & $ 0.0 \pm 0 $ & $ 0 $ \\ 
$ -10.00 $ & $ 2.76 \pm 0.18 $ & $ 3177 $ \\ 
$ -7.50 $ & $ 27.8 \pm 0.6 $ & $ 3183 $ \\ 
$ -5.00 $ & $ 42.4 \pm 0.9 $ & $ 3186 $ \\ 
$ -2.50 $ & $ 43.9 \pm 1.0 $ & $ 3181 $ \\ 
$ 0.00 $ & $ 44.4 \pm 0.9 $ & $ 3185 $ \\ 
$ 2.50 $ & $ 44.7 \pm 1.0 $ & $ 3187 $ \\ 
$ 5.00 $ & $ 47.1 \pm 1.0 $ & $ 3182 $ \\ 
$ 7.50 $ & $ 43.2 \pm 0.9 $ & $ 3180 $ \\ 
$ 10.00 $ & $ 15.6 \pm 0.4 $ & $ 3188 $ \\ 
$ 15.00 $ & $ 0.0 \pm 0 $ & $ 0 $ \\ 

\end{tabular}

\caption{Rate in funzione dell'angolo con il collimatore da 5\! mm.
		L'errore sugli angoli è stimato essere 1.25$^{\circ}$.}
\label{tab:coll5}
\end{table}




\begin{table}[h]
\centering

\begin{tabular}{c|c|c}
angolo [\si{\degree}] & rate  [\si{s^{-1}}] & moda [digit] \\
\hline
$ -2.50 $ & $ 0.08 \pm 0.04 $ & $ 3156 $ \\ 
$ -1.25 $ & $ 2.48 \pm 0.20 $ & $ 3171 $ \\ 
$ 0.00 $ & $ 14.63 \pm 0.32 $ & $ 3174 $ \\ 
$ 1.25 $ & $ 31.5 \pm 0.7 $ & $ 3178 $ \\ 
$ 2.50 $ & $ 34.8 \pm 0.8 $ & $ 3185 $ \\ 
$ 3.75 $ & $ 26.9 \pm 0.6 $ & $ 3184 $ \\ 
$ 5.00 $ & $ 12.70 \pm 0.31 $ & $ 3188 $ \\ 
$ 6.25 $ & $ 2.81 \pm 0.22 $ & $ 3193 $ \\ 
$ 7.50 $ & $ 0.11 \pm 0.04 $ & $ 3185 $ \\ 

\end{tabular}

\caption{Dati per le misure con il collimatore da 1\! mm.}
\label{tab:coll1}
\end{table}



\begin{table}
\centering

\begin{tabular}{c|c|c}
angolo [\si{\degree}] & rate  [\si{s^{-1}}] & moda [digit] \\
\hline
$ -2.50 $ & $ 0.27 \pm 0.04 $ & $ 3213 $ \\ 
$ -1.25 $ & $ 1.20 \pm 0.11 $ & $ 3198 $ \\ 
$ 0.00 $ & $ 2.36 \pm 0.15 $ & $ 3190 $ \\ 
$ 1.25 $ & $ 2.56 \pm 0.16 $ & $ 3187 $ \\ 
$ 2.50 $ & $ 2.41 \pm 0.15 $ & $ 3192 $ \\ 
$ 3.75 $ & $ 1.32 \pm 0.09 $ & $ 3187 $ \\ 
$ 5.00 $ & $ 0.32 \pm 0.05 $ & $ 3193 $ \\ 

\end{tabular}

\caption{Dati per le misure con il collimatore a croce.}
\label{tab:super}
\end{table}





\end{document}