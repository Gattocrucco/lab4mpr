\section{Conclusioni}

La nostra stima della massa dell'elettrone $m_e=\SI{0.466 \pm 0.020}{MeV}$
risulta leggermente in tensione con il valore noto e ha un'incertezza relativa del \SI4\%.
La componente principale dell'incertezza è la stabilità della calibrazione dello spettrometro.
Non riteniamo che questa sia una misura affidabile perché è molto soggetta al bias dello sperimentatore.
In particolare:
\begin{enumerate}
	\item Non abbiamo completato le prese dati a causa della rottura,
	quindi abbiamo usato prese di prova
	di cui abbiamo dovuto stimare molto indirettamente le incertezze.
	\item Di sole quattro prese dati ne abbiamo scartata una come outlier.
	\item In generale abbiamo eseguito tutta l'analisi a posteriori guardando i risultati.
\end{enumerate}
Inoltre notiamo dalla \autoref{fig:me} come la media sia dominata dalla misura $\SI{15}{\degree}_\text{fine}$,
mentre le misure a grandi angoli, che a parità di condizioni permetterebbero la misura di massa più precisa,
sono penalizzate dall'assenza dell'informazione temporale.
