\section{Conclusioni}

La nostra stima della massa dell'elettrone $m_e^{(\text{comp})}=\SI{0.485 \pm 0.020}{MeV}$
risulta compatibile con il valore noto e ha un'incertezza relativa del \SI4\%.
La componente principale dell'incertezza è la stabilità della calibrazione dello spettrometro.
Non riteniamo che questa sia una misura affidabile perché è molto soggetta al bias dello sperimentatore.
In particolare:
\begin{enumerate}
	\item Non abbiamo completato le prese dati a causa della rottura,
	quindi abbiamo usato prese di prova
	di cui abbiamo dovuto stimare molto indirettamente le incertezze.
	\item Di sole quattro prese dati ne abbiamo scartata una come outlier.
	\item Abbiamo imposto a mano la risoluzione della leggera incompatibilità
	tra $m_e^{(1.33)}$ e $m_e^{(1.17)}$.
	\item In generale abbiamo eseguito tutta l'analisi a posteriori guardando i risultati.
\end{enumerate}
Abbiamo comunque spazzato un intervallo di angoli soddisfacente, da \SI{15}{\degree} a~\SI{60}{\degree}.
