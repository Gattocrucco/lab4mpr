\section{Strumentazioni}
Gli strumenti principali a disposizione per questa esperienza (fatta eccezione per i soliti moduli \texttt{NIM}) consiste in:
\begin{itemize}
	\item una sorgente radioattiva principale di \co e altre sorgenti meno attive per la calibrazione: \cs, \na, \am;
	\item il PMT1: un rivelatore basato su uno scintillatore organico (che fungerà da bersaglio);
	\item il PMT2: un rivelatore basato su un cristallo scintillatore di NaI(Tl) 2"$\times$2" dotato di un preamplificatore in cascata al fotomoltiplicatore;
	\item un formatore/amplificatore che produce in uscita un segnale gaussiano con altezza di picco proporzionale all'energia rilasciata e duarata $\sim \SI{10}{\micro s}$;
	\item un ADC a 13 bit che può sia essere triggerata dall'esterno, sia campionare automaticamente sul picco tutti i segnali che superano una certa soglia interna\footnote{anche molto bassa CIPPALIPPA} (chiameremo questa modalità \emph{automatica}).
\end{itemize}