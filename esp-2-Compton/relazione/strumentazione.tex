\subsection{Apparato di misura}

Gli strumenti principali a disposizione per questa esperienza (fatta eccezione per i soliti moduli \texttt{NIM}) consiste in:
\begin{itemize}
	\item una sorgente radioattiva principale di \co\! e altre sorgenti meno attive per la calibrazione: \cs, \na, \am;
	\item il PMT1: un rivelatore basato su uno scintillatore organico (che fungerà da bersaglio);
	\item il PMT2: un rivelatore basato su un cristallo scintillatore di NaI(Tl) 2"$\times$2" dotato di un preamplificatore in cascata al fotomoltiplicatore (che useremo come spettrometro);
	\item un formatore/amplificatore che produce in uscita un segnale gaussiano con altezza di picco proporzionale all'energia rilasciata e durata $\sim \SI{10}{\micro s}$;
	\item un ADC a 13 bit che può sia essere triggerata dall'esterno, sia campionare automaticamente sul picco tutti i segnali che superano una certa soglia interna\footnote{anche molto bassa CIPPALIPPA} (chiameremo questa modalità \emph{automatica}).
\end{itemize}

\subsubsection{Sorgenti radioattive}
La sorgente radioattiva principale di \co\; decade $\beta^-$ in uno stato eccitato di $^{60}$Ni che a sua volta decade $\gamma$ due volte in cascata emettendo fotoni di $\SI{1.17}{MeV}$ e $\SI{1.33}{MeV}$.

La sorgente ha un'attività dichiarata di \SI{74}{MBq} al febbraio 1997 e si trova in un contenitore di piombo con collimatore a sezione circolare\footnote{Di un materiale diverso e a Z più basso.}. 
Data l'emivita del \co\; di $\tau_2 = \SI{5.3}{yr}$ alla data della nostra esperienza l'attività stimata della sorgente è $\sim 2^{21/5.3} \cdot \SI{74}{MBq} = \SI{4.7}{MBq}$.

Dalla costante di dose del \co\;  ($\SI{0.35}{\micro Sv\;m^2\;MBq^{-1}\;h^{-1}}$) FONTE CIPPALIPPA è possibile stimare la dose assorbita nel corso dell'esperienza: $\SI{0.35}{\micro Sv\;m^2\;MBq^{-1}\;h^{-1}} \cdot \SI{4.7}{MBq} \cdot (\SI{1}{m})^2= \SI{1.6}{\micro Sv\;h^{-1}}$ (a un metro di distanza) che va confrontato con il fondo naturale $\sim\SI{0.3}{\micro Sv\;h^{-1}}$. Si tratta di una stima per eccesso poiché considera la sorgente isotropa, ma la nostra sorgente è schermata ed emette solo in un piccolo angolo solido (non nella nostra direzione).

Le altre sorgenti di calibrazione a disposizione hanno un'attività minore di $\SI{72}{kBq}$\footnote{Si tratta dell'attività dichiarata dalla ditta produttrice alla data di vendita che è ignota.}\marginpar{questo dettaglio non è particolarmente rilevante ma bho (Bob)} e i loro principali modi di decadimento sono schematizzati in \autoref{tab:sorgenti_cal}. Notiamo inoltre che il \na\; decade $\beta^+$, ci aspettiamo perciò di osservare nel suo spettro il segnale dei fotoni $\gamma$ a \SI{511}{keV} prodotti nell'annichilazione.

\begin{table}[h]
	\centering
	\begin{tabular}{c|ccc}
		sorgenti & \multicolumn{3}{c}{principali modi di decadimento} \\ \hline
		\co & $\beta^{-} (\SI{318}{keV})$ & $\gamma (\SI{1173}{keV})$ & $\gamma (\SI{1332}{keV})$  \\
		\cs & $\beta^{-} (\SI{512}{keV})$ & $\gamma (\SI{662}{keV})$ \\
		\na & $\beta^{+} (\SI{546}{keV})$ & $\gamma (\SI{1275}{keV})$ \\
		\am & $\alpha (\SI{5486}{keV})$ & $\gamma (\SI{59.5}{keV})$ \\ \hline
	\end{tabular}
	\caption{\label{tab:sorgenti_cal} Principali modi di decadimento delle sorgenti a disposizione FONTE CIPPLIPPA}
\end{table} 
