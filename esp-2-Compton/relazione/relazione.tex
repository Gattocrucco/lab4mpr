\documentclass[a4paper]{article}

\usepackage[utf8]{inputenc}
\usepackage[T1]{fontenc}
\usepackage[italian]{babel}

\usepackage[margin=4.2cm, top=1.5cm, bottom=2.5cm]{geometry}

\usepackage{siunitx}
\usepackage{amsmath}
\usepackage{amssymb}
\usepackage{xfrac}
\usepackage{esint}
\usepackage[hidelinks]{hyperref}
\usepackage{graphicx}
\usepackage[font={sf}]{caption}
\usepackage{pdflscape}
\usepackage{makecell}
\usepackage{float}
\usepackage{subfig}
\usepackage{wasysym}
\usepackage{booktabs}

\sisetup{separate-uncertainty}

\setlength{\marginparwidth}{95pt}
\let\oldmarginpar\marginpar
\renewcommand\marginpar[1]{\oldmarginpar{\scriptsize\sffamily #1}}
\newcommand*\de{\mathrm{d}}
\newcommand*\pdv[2]{\frac{\partial #1}{\partial #2}}
\newcommand*\dv[2]{\frac{\de #1}{\de #2}}
\DeclareMathOperator\Ei{Ei}
\newcommand*\is{\equiv}
\newcommand\cs{$^{\text{137}}\text{Cs}$}
\newcommand\co{$^{\text{60}}\text{Co}$}
\newcommand\na{$^{\text{22}}\text{Na}$}
\newcommand\am{$^{\text{241}}\text{Am}$}
\newcommand\sr{$^{\text{90}}\text{Sr}$}

\sisetup{%
separate-uncertainty=true,
multi-part-units=single,
exponent-product=\cdot}

\frenchspacing

\title{Relazione di laboratorio:\\
Esperienza 2. Scattering Compton}
\author{Andrea Marasciulli
\and Giacomo Petrillo
\and Roberto Ribatti}
\date{15 febbraio -- 9 marzo 2018}

\begin{document}

\maketitle

\begin{abstract}
	Ricaviamo la massa dell'elettrone misurando l'energia del fotone uscente nello scattering Compton.
	Risulta \SI{0.485 \pm 0.020}{MeV},
	dove l'incertezza è dominata dalla stabilità della calibrazione dello spettrometro.
\end{abstract}

{ \tableofcontents}

\newpage
\section{Introduzione}

%\section{Introduzione}

\subsection{Obiettivo}

Vogliamo misurare il rate per unità di superficie orizzontale
dei raggi cosmici che passano nel laboratorio.

\subsection{Apparato}

\begin{figure}
	\center
	\includegraphics[width=\textwidth]{apparato}
	\caption{\label{fig:apparato}
	Apparato di misura.
	La struttura portante non è disegnata.
	La guida ottica e il tubo fotomoltiplicatore (PMT) sono disegnati solo per il PM6.}
\end{figure}

\begin{figure}
	\center
	\includegraphics[width=23em]{miniscint}
	\caption{\label{fig:miniscint}
	Piccolo scintillatore mobile, che chiamiamo \emph{miniscint}.}
\end{figure}

La parte principale dell'apparato sono 6 lastre di scintillatore plastico
posizionate orizzontalmente e allineate verticalmente,
che non possiamo spostare,
collegate a tubi fotomoltiplicatori (vedi \autoref{fig:apparato}).

Disponiamo anche di un piccolo scintillatore mobile (il <<miniscint>>) (\autoref{fig:miniscint})
che può essere appoggiato sulle lastre principali.

\section{Strumentazioni}
Gli strumenti principali a disposizione per questa esperienza (fatta eccezione per i soliti moduli \texttt{NIM}) consiste in:
\begin{itemize}
	\item una sorgente radioattiva principale di \co e altre sorgenti meno attive per la calibrazione: \cs, \na, \am;
	\item il PMT1: un rivelatore basato su uno scintillatore organico (che fungerà da bersaglio);
	\item il PMT2: un rivelatore basato su un cristallo scintillatore di NaI(Tl) 2"$\times$2" dotato di un preamplificatore in cascata al fotomoltiplicatore;
	\item un formatore/amplificatore che produce in uscita un segnale gaussiano con altezza di picco proporzionale all'energia rilasciata e duarata $\sim \SI{10}{\micro s}$;
	\item un ADC a 13 bit che può sia essere triggerata dall'esterno, sia campionare automaticamente sul picco tutti i segnali che superano una certa soglia interna\footnote{anche molto bassa CIPPALIPPA} (chiameremo questa modalità \emph{automatica}).
\end{itemize}

\section{Teoria}

\section{Spettrometria $\gamma$ con gli scintillatori}
Un punto cruciale di questa esperienza é...

I principali meccanismi di interazione dei fotoni con la materia sono:
\begin{itemize}
	\item scattering Rayleigh (detto anche scattering elastico);
	\item assorbimento fotoelettrico (detto anche scattering anelastico);
	\item scattering Compton;
	\item produzione di coppie.
\end{itemize}
 Di questi, gli ultimi 3 sono i modi nel quale i fotoni sono rivelati.
 Gli scintillatori rivelano le particelle che gli attraversano producendo luce in funzione dell'energia rilasciata all'interno: t
 
 La sezione d'urto di questi processi varia con l'energia dei fotoni e col numero atomico del materiale con cui interagiscono.
 Per qualsiasi numero atomico è vero che il processo predominante a basse energie ($< \SI{10}{keV}$) è l'assorbimento fotoelettrico mentre ad alte energie (($< \SI{100}{MeV}$) domina la produzione di coppie. Ad energie intermedie l'interazione principale è lo scattering Compton, il cui effetto diventa via via più importante al diminuire del numero atomico. 
 
 Il nostro rivelatore è un cristallo di NaI per il quale sono riportate le sezioni d'urto al variare dell'energia in \autoref{cross_section_NaI}.
 Come è chiaro dal grafico gli unici processi rilevanti nel range di energie di interesse ($\SI{100}{keV} \sim \SI{1.5}{MeV}$) sono l'assorbimento fotoelettrico e lo scattering Compton.
 
 \paragraph{Assorbimento fotoelettrico}
 Il fotone incidente è completamente assorbito e ionizza l'atomo estraendo quindi un elettrone con energia cinetica pari a $E_e = E_{\gamma} - E_b$ dove $E_{\gamma}$ è l'energia del fotone incidente e $E_b$ è l'energia di legame, solitamente piccola se paragonata a $E_{\gamma}$. Il risultato finale è una particella carica nello scintillatore con (praticamente) la stessa energia del fotone incidente.
 
 \paragraph{Scattering Rayleigh}
 Nello scattering Rayleigh l'intero atomo funge da bersaglio perciò dopo l'urto il fotone cambia direzione e l'atomo rincula per conservare il momento. Data la grande massa dell'atomo (rispetto all'energia del fotone) l'energia scambiata è trascurabile e il fotone uscente ha praticamente la stessa energia iniziale. E' chiaro quindi che i fotoni non potranno essere rivelati in uno scintillatore con questo meccanismo perché nessuna energia è stata rilasciata all'interno.
 
 \paragraph{Scattering Compton}
 Se il fotone è abbastanza energetico può urtare un singolo elettrone. Per fotoni di questo tipo l'energia di legame può essere trascurata e si può trattare l'elettrone come libero. Il risultato di questo scattering è 
 
 \subsection{Risposta del detector}
 Se il detector fosse abbastanza esteso tutta l'energia dei fotoni, indipendentemente dalla complessità dell'interazione, verrebbe rilasciata nel rivelatore. Per una sorgente di fotoni ad una energia fissata, come potrebbe essere una sorgente di raggi $\gamma$\footnote{I raggi $\gamma$ delle sorgenti radioattive hanno tipicamente larghezze $\Gamma << \SI{1}{eV}$ e possono quindi considerarsi mono-energetici.} come il Cs-137, lo spettro energetico prodotto da un tale scintillatore sarebbe un singolo fotopicco all'energia dei fotoni $\gamma$ con una larghezza dovuta essenzialmente alla risoluzione dello strumento, come analizzeremo nel seguito.
 
 Nel nostro caso disponiamo di un rivelatore $2"\times2"$ e a partire dal noto grafico in FONTE che ci da la lunghezza di radiazione al variare dell'energia del fotone in vari materiali e semplice calcolare che per il nostro rivelatore di NaI la lunghezza di radiazione vale $\sim \SI{5}{cm}$, quindi paragonabile alla dimensione del cristallo, ne segue che solo il $\sim 60\%$ dei fotoni interagisce nel cristallo\footnote{E l'efficienza varia con l'energia come vedremo nel seguito}, per cui l'approssimazione di detector esteso non è applicabile.
 Lo spettro prodotto sarà la somma di molti effetti:
 \begin{itemize}
 	\item l'assorbimento fotoelettrico produrrà un fotopicco all'energia dei fotoni incidenti;
 	\item i fotoni che fanno scattering Compton rilasceranno solo parte della loro energia producendo il caratteristico profilo...
 	\item i fotoni che fanno scattering Compton sui materiali che circondano il detector possono tornare nello stesso producendo un picco, detto di backscattering nella regione a basse energie.
 	\item spesso sarà visibile un picco all'energia caratteristica dei raggi X emessi dell'assorbitore, in questo caso gli atomi di sodio e iodio che costituiscono il rivelatore. 
 \end{itemize}
  
 
 \subsection{Efficienza}
 
 \subsection{Linearità}
 Un rivelatore di radiazione ideale dovrebbe essere perfettamente lineare 
 \subsection{Risoluzione}
 
 
 
 
 

\subsection{Simulazione}

Implementiamo una simulazione Monte Carlo dello scattering Compton nel bersaglio
e della rivelazione nello spettrometro.
Ogni fotone è simulato in questo modo:
\begin{enumerate}
	\item Viene estratta una direzione iniziale da una gaussiana per tenere conto della larghezza del fascio.
	\item Viene simulato lo scattering Compton nel bersaglio,
	usando la formula \eqref{klein-nishina} per estrarre gli angoli e \eqref{energia_compton} per calcolare l'energia.
	Lo scattering avviene in un centro fissato (non teniamo conto della dimensione del bersaglio).
	\item Se il fotone passa per lo spettrometro,
	viene calcolata la probabilità che faccia Compton oppure fotoelettrico nel cristallo.
	L'energia rilasciata per il fotoelettrico è quella del fotone,
	per il Compton è quella cinetica dell'elettrone, trascurando l'energia di legame
	(non simuliamo che il fotone possa fare Compton più di una volta).
	\item Aggiungiamo a ogni energia un'estrazione gaussiana
	per simulare la risoluzione.
\end{enumerate}
Gli spettri di fotoelettrico e Compton sono ottenuti con un istogramma pesato con le probabilità di interazione calcolate.
La normalizzazione relativa di fotoelettrico e Compton è macroscopicamente errata,
supponiamo perché non abbiamo tenuto conto dei fotoni che fanno Compton più di una volta,
quindi non la teniamo in considerazione.


\section{Misura e analisi}

\subsection{Misure e osservazioni preliminari}

\paragraph{Punto di lavoro del PMT2}
Vogliamo trovare un buon punto di lavoro per il PMT2 che è alimentato con tensioni positive fino a $\sim \SI{800}{V}$. 
Mandiamo il segnale preamplificato del PMT2 all'amplificatore e la sua uscita all'ADC in modalità automatica.
Scegliamo di alimentare il PMT2 alla tensione consigliata di \SI{650}V. A questa tensione (e con guadagno dell'amplificatore impostato a $\times 1$) il fotopicco generato dai fotoni più energetici tra le sorgenti a disposizione (ovvero il picco a \SI{1.33}{MeV} del \co) si trova a $2/3$ della scala. Questo permette, agendo sul fattore di amplificazione del formatore/amplificatore di portare il fotopicco a fondo scala  e massimizzare la risoluzione dello spettrometro, mantenendo allo stesso tempo flessibilità nel caso la risposta del rivelatore non si rivelasse stabile\footnote{Cosa che abbiamo verificato essere vera: la posizione dei fotopicci può variare nel tempo fino al $\sim 10\%$.}.

\paragraph{Stabilità della risposta del PMT2}
Osservando l'istogramma dei campionamenti e notiamo che la risposta dello spettrometro è molto sensibile alle variazioni della tensione di alimentazione: da \SI{600}V a \SI{575}V, il valore in uscita si dimezza.
Il manuale dell'alimentatore riporta una stabilità dello $0.5\permil$ su \SI{24}h e dell' $1\permil$ su un mese. Volendo fare una stima otteniamo variazioni del $2/25 \times 600/1000 = 5\%$ in un mese, la metà in \SI{24}h. Sembra quindi che la stabilità della risposta del nostro spettrometro sia un limite importante nella nostra misura per cui nel seguito analizzeremo approfonditamente questo parametro.
%%%%%%%%%%%%%%%%%%%%%%%%%%%%%%%%%%%%%%%%%%%%%%%%%%%%%%%%%%%

\paragraph{Interazione con il bersaglio}
Poniamo lo scintillatore plastico (PMT1) davanti alla sorgente in modo che i raggi $\gamma$ possano interagire e subire scattering Compton. Osserviamo l'istogramma dei campionamenti del nostro spettrometro a vari angoli rispetto al bersaglio mentre l'ADC campiona in modalità automatica. 
Come è possibile vedere dalla \autoref{4ang} a piccolo angolo, quando lo spettrometro è direttamente attraversato dai fotoni che non interagiscono con il bersaglio, lo spettro è praticamente indistinguibile da quello acquisito senza bersaglio. A grandi angoli dovremmo vedere i fotoni che hanno fatto scattering Compton sul bersaglio e che dovrebbero riprodurre un circa lo stesso spettro ma traslato a energie minori, tuttavia come è chiaro dalla  \autoref{4ang} i fotoni che interagiscono sul muro subito dietro lo spettrometro o sul ferro (?CIPPALIPPA) del collimatore sovrastano in statistica il segnale che vorremmo osservare.\marginpar{Ci mettiamo una breve spiegazione della forma degli spettri o fottesega? (Bob)}

\begin{figure}[h]
\centering

\subfloat
{\includegraphics[width= 16 em]{0g}} \qquad
\subfloat
{\includegraphics[width= 16 em]{15g}} \\

\subfloat
{\includegraphics[width= 16 em]{45g}} \qquad
\subfloat
{\includegraphics[width= 16 em]{90g}}

\caption{Spettri raccolti (in modalità automatica) a diversi angoli rispetto al bersaglio.}
\label{4ang}
\end{figure}

\marginpar{La parte che segue è abbastanza inutile se non aggiungiamo la spiegazione che dicevo prima (Bob)}
Abbiamo poi confrontato lo spettro a \SI{45}{\degree} con o senza scintillatore plastico, ma non abbiamo notato nessuna differenza. Abbiamo anche verificato che, come atteso, il rate di eventi diminuisce all'aumentare della distanza.
\marginpar{forse alcune di queste considerazioni sono stupide, ma è più facile togliere che aggiungere}

L'ultima acquisizione di questa serie è diversa rispetto alle altre: lo scintillatore cristallino è stato schermato lateralmente e superiormente per ridurre il backscattering. L'istogramma di \autoref{casetta} mostra come questo fenomeno sia evidentemente diventato più raro.

\begin{figure}
\centering
\includegraphics[width=25 em]{45gs}
\caption{Spettro energetico del PMT2 con schermaggio superiore e laterale.}
\label{casetta}
\end{figure}

\paragraph{Trigger esterno} Prima di esporre i risultati ottenuti illustriamo il funzionamento del circuito adoperato.\\
Vogliamo acquisire i segnali ogni volta che l'uscita negativa di breve durata (qualche \si{ns}) del PMT2 supera una certa soglia. L'ampiezza tipica di questo segnale vale \SI{20}{mV}, perciò deve essere amplificata per poter essere letta dal discriminatore a nostra disposizione che ha una soglia minima di \SI{35}{mV}. Colleghiamo allora questa uscita ad una amplificatore 10x 
\marginpar{spero che gradiscano la notazione ``10x'' \\  ci vuole anche il disegno}
e mandiamo poi il segnale al solito discriminatore con soglia al minimo.

Per registrare il segnale dobbiamo provvedere alla costruzione di un trigger per l'ADC che, come specificato dalla documentazione, deve partire almeno \SI{200}{ns} prima del picco e deve terminare almeno \SI{200}{ns} dopo. Usando l'oscilloscopio, ritardiamo il segnale del discriminatore e lo allunghiamo attraverso l'uso di un modulo \texttt{gate \& delay} non retriggerabile. Scegliamo di farlo durare \SI{550}{ns} prima e dopo il picco del formatore.

Colleghiamo il cavo coassiale che trasmette l'uscita del formatore all'ingresso ``gate'' dell'ADC facendo un parallelo con un tappino da \SI{50}{\Omega} perché questo ingresso ha una impedenza alta e il nostro accorgimento evita la deformazione del segnale.
Modifichiamo il file di configurazione in modo che l'ADC acquisisca soltanto quando il trigger è attivo. Notiamo che la soglia appena utilizzata elimina tutti i segnali inferiori ai 1000 digit.
\marginpar{Quale soglia?\\
\emph{Petrillo}}


\subsection{Caratteristiche geometriche del fascio}

\subsubsection{Accorgimenti sperimentali}

Ci interessa misurare la divergenza angolare del fascio.
Per farlo posizioniamo il PMT2 di fronte alla sorgente a vari angoli e triggeriamo sul PMT2 stesso. Lo scopo è disegnare un grafico dei rate in funzione dell'angolo.

Per conoscere il tempo di acquisizione montiamo un circuito che, usando un timer,
fa partire e fermare simultaneamente il trigger e il contatore.

Per misurare meglio gli angoli abbiamo stampato un goniometro che abbiamo posizionato sotto il supporto trasparente della guida su cui scorre il PMT2. Infine abbiamo tracciato un segmento sulla parte trasparente che ci permette di capire a quale angolo posizioniamo il PMT2.
Lo zero degli angoli è riferito al goniometro e otterremo il centro del fascio dalle misure.

\subsubsection{Analisi dei dati e risultati}

Scegliamo di misurare il rate solo in un certo intervallo di energia che contiene i due fotopicchi,
per evitare di includere eventuali fondi a bassa energia che non provengono direttamente dalla sorgente.
Facciamo anche una misura di <<fondo costante>> con il PMT2 posizionato in un punto non visibile dalla sorgente,
ma con la sorgente aperta.

\begin{table}
	\centering

	\begin{tabular}{cc|cc|cc}
		\toprule
		angolo [$\si{\degree}$] & rate [$\si{s^{-1}}$] & angolo [$\si{\degree}$] & rate [$\si{s^{-1}}$] & angolo [$\si{\degree}$] & rate [$\si{s^{-1}}$] \\
		\midrule
			$-20$ & $102.8 \pm 1.1$ & $-4$ & $1310 \pm 6$  & $5$ & $544 \pm 4$       \\
			$-15$ & $149.6 \pm 1.4$ & $-3$ & $2092 \pm 8$  & $6$ & $205.5 \pm 2.1$   \\
			$-10$ & $191.4 \pm 1.7$ & $-2$ & $2847 \pm 9$  & $7$ & $173.2 \pm 1.9$   \\
			$-9$ & $208.0 \pm 1.8$  & $-1$ & $3521 \pm 11$ & $8$ & $173.7 \pm 1.6$   \\
			$-8$ & $215.2 \pm 1.8$  & $0$ & $3792 \pm 11$  & $9$ & $174.7 \pm 1.6$   \\
			$-7$ & $228.3 \pm 2.2$  & $1$ & $3426 \pm 10$  & $10$ & $170.2 \pm 1.6$  \\
			$-6$ & $298 \pm 3$      & $2$ & $2695 \pm 9$   & $15$ & $118.5 \pm 1.2$  \\
			$-5$ & $680 \pm 5$      & $3$ & $1930 \pm 8$   & $20$ & $78.4 \pm 0.9$   \\
			     &                  & $4$ & $1160 \pm 6$   &      &\\
		 \bottomrule
	\end{tabular}

	\caption{Rate senza coincidenza nei canali dell'ADC 5500--8040
	al variare dell'angolo del PMT2.
	L'errore sugli angoli è 0.1$^{\circ}$.
	Il fondo costante misurato è \SI{4.43\pm0.2}{s^{-1}}.}
	\label{tabfo}
\end{table}

\begin{figure}
	\centering
	\includegraphics[width=32em]{forma}
	\caption{Dati e fit della forma del fascio.
	Nella legenda, <<fondo p.>> sta per <<fondo piccato>> e <<fondo c.>> per <<fondo costante>>.}
	\label{forma}
\end{figure}

Le misure sono riportate in \autoref{tabfo}.
La misura di fondo costante risulta praticamente trascurabile.
In \autoref{forma} sono riportati i dati con sottratto il fondo costante.
Si nota che la forma sembra essere la somma di due gaussiane (che sono parabole in scala logaritmica).
Chiamiamo la gaussiana più larga <<fondo piccato>>
e la interpretiamo come raggi che non provengono direttamente dalla sorgente
perché si nota un'asimmetria rispetto al centro, che non c'è nella gaussiana stretta (il <<segnale>>)
e che non corrisponde alla geometria cilindrica del collimatore.

Se fittiamo una gaussiana per i dati degli angoli da \SI{-5}\degree{} a \SI5\degree{} otteniamo già un buon accordo.
Tuttavia, vista la nostra interpretazione, riteniamo più sensato sottrarre il fondo piccato.
A causa dell'asimmetria del fondo, se fittiamo le due gaussiane non otteniamo un buon accordo
e la gaussiana di segnale viene fittata visibilmente peggio.
In sostanza: sappiamo modellare bene il segnale ma non il fondo.
Allora estraiamo il fondo dal fit delle due gaussiane,
lo sottraiamo e fittiamo la sola gaussiana di segnale da \SI{-4}\degree{} a \SI4\degree.

La deviazione standard del segnale risulta \SI{2.52 \pm 0.03}{\degree},
il centro \SI{-0.09 \pm 0.04}{\degree}.
Dal segnale dobbiamo deconvolvere la forma dello scintillatore NaI,
la cui deviazione standard è \SI{1.82 \pm 0.07}\degree, ottenuta sapendo che è un cilindro di raggio \SI{2}{''}
e che dista \SI{40\pm 1.5}{cm} (l'incertezza sulla distanza è data dalla profondità dello scintillatore e dalla larghezza del collimatore).
La deviazione standard della forma del fascio si ottiene semplicemente dalla sottrazione in quadratura
e risulta \SI{1.73 \pm 0.09}{\degree}.


\subsection{Calibrazione}

\begin{figure}
	\centering
	\includegraphics[width=\textwidth]{calibration}
	\caption{\label{fig:calibration}
	Dati per la calibrazione.
	La curva di fit è riportata solo per 22feb.
	Nel fit della scala, il punto a sinistra escluso dal fit è l'americio.}
\end{figure}

\begin{table}
	\centering
	\begin{tabular}{c|cc|c}
		Calibrazione & $m$ [\si{digit\,keV^{-1}}] & $q$ [\si{digit}] & $a$ [\si{digit\,keV^{-1}}] \\
		\hline
		20feb & \num{5.6(3) } & \num{2(4)e+2} &   \num{4.93(8) } \\
		22feb & \num{5.52(6)} & \num{1.9(7)e+2} & \num{4.39(6) } \\
		26feb & \num{5.39(3)} & \num{11(28)} &    \num{4.52(8) } \\
		27feb & \num{5.49(7)} & \num{2(8)e+1} &   \num{4.49(10)}
	\end{tabular}
	\caption{\label{tab:calibration}
	Risultato della calibrazione.
	$m$ e $q$ sono i parametri per calibrare la scala di energia,
	$a$ è il parametro per calcolare la risoluzione.}
\end{table}

Per calibrare la scala di energia dell'ADC usiamo le sorgenti ausiliarie.
Poniamo una sorgente alla volta davanti allo spettrometro (con la sorgente principale chiusa e non visibile)
e misuriamo lo spettro.
Fittiamo i fotopicchi con gaussiane più un fondo modellato empiricamente con una retta tagliata alle ordinate negative
(in modo che sia forzatamente positivo).
Da ogni fotopicco otteniamo il centro e la deviazione standard.

\paragraph{Scala dell'energia}

Fittiamo i centri in funzione dell'energia dei fotoni delle sorgenti con una retta $y=mx+q$,
non includendo nel fit l'americio (quello a energia minore)
perché non è nell'intervallo di energie in cui useremo la calibrazione
e il modello lineare non è in accordo statistico con i dati;
per lo stesso motivo moltiplichiamo le incertezze su $m$ e $q$ per il fattore di scala $\sqrt{\chi^2/\mathrm{ndof}}$,
dove il $\chi^2$ è del fit complessivo di tutte le calibrazioni.
Teniamo conto della correlazione dei centri del cobalto.

\paragraph{Risoluzione}

Come modello per la risoluzione in funzione dell'energia usiamo la formula
FONTE CIPPALIPPA
\begin{equation}
	\sigma_E(E) = a \cdot E \cdot \frac{2.27 + 7.28 \cdot E ^ {-0.29} - 2.41 \cdot E ^ {0.21}} {235}
\end{equation}
con $E$ in \si{MeV}, dove fittiamo solo l'ampiezza $a$.
Anche in questo caso moltiplichiamo l'incertezza per il fattore di scala,
ma il fit è separato per ogni calibrazione.
I risultati sono riportati in \autoref{tab:calibration},
i dati in \autoref{fig:calibration}.


\subsection{Misura finale}\label{par:fondo2}

\subsubsection{Prese dati}

\begin{table}
	\centering
	\begin{tabular}{llll}
		\toprule
		Data & Durata & Calibrazioni & Angolo \\
		\midrule
		19--20 feb    & 20 ore & dopo, limitata &           45.0\,\si{\degree} \\
		20--22 feb    & 40 ore & dopo           &   61$\sfrac34$\,\si{\degree} \\
		26--27 feb    & 22 ore & prima e dopo   &           15.0\,\si{\degree} \\
		27 feb--1 mar & 38 ore & prima          & \phantom{1}7.0\,\si{\degree} \\
		\bottomrule
	\end{tabular}
	\caption{\label{tab:misure}
	Prese dati per ricavare la massa dell'elettrone.}
\end{table}

Per ricavare la massa dell'elettrone usiamo quattro misure di spettro.
Tutte le misure durano circa 24 ore per avere sufficiente statistica (si è poi rivelata eccessiva).
Era nelle nostre intenzioni avere una calibrazione prima e dopo ogni misura,
ma la rottura dell'alimentatore del crate ha ostacolato i nostri piani.
La misura dopo la quale il crate si è rotto ha la calibrazione solo prima;
due misure che abbiamo recuperato e che inizialmente erano solo misure di prova hanno la calibrazione solo dopo,
una delle quali non era intesa come misura di calibrazione e quindi ha solo il cobalto
(calibrazione 20feb in \autoref{tab:calibration}).
L'elenco delle misure è riportato in \autoref{tab:misure}.


\section{Conclusioni}

Abbiamo incontrato un rumore anomalo che non siamo riusciti a comprendere,
e nemmeno a caratterizzare abbastanza da essere sicuri su come fare le misure.
Ci siamo anche scontrati con difficoltà tecniche nel fare il fit.
Comunque abbiamo ottenuto un risultato in accordo con le aspettative, anche se poco preciso.
Esprimendo il risultato dato in \autoref{sec:risultato} come flusso verticale,
si ottiene $I_v=\SI{112(10)}{m^{-2}\,s^{-1},sr^{-1}}$ dalla formula
\begin{equation*}
	I_v = \Phi \frac{\alpha+1}{2\pi}
\end{equation*}


\begin{thebibliography}{99} % Bibliography

\bibitem[1] {1}
S. Cecchini, M. Spurio, Atmospheric muons: experimental aspects,
{\em arXiv}:1208.1171v1 [astro-ph.EP].

\bibitem[2] {2}
J.Crookes, B.Rastin, An investigation of the absolute intensity of muons at sea-level, 
{\em Nucl. Phys. B} 39 (1972) 493.

\bibitem[3] {3}
K.Greisen, The intensities of the hard and soft components of cosmic rays as functions of altitude and
zenith angle,
{\em Phys. Rev.} 61 (1942) 212.

\bibitem[4] {4}
B. Rossi, Interpretation of cosmic-ray phenomena,
{\em Rev. Mod. Phys.} 20 (1948) 537.

\bibitem[5] {5}
R.J.R.Judge, W.F.Nash, Measurements on the Muon Flux at Various Zenith Angles,
{\em Nuovo Cimento} 35 (1965) 999.

\bibitem[6] {6}
S.Fukui, T.Kitamura, Y.Murata, On the range spectrum of $\mu$-mesons at sea-level at geomagnetic latitude $\SI{24}{\degree N}$,
{\em J. Phys. Soc. Jpn.} 12 (1957) 854.

\bibitem[7] {7}
Peter K.F. Greider, Cosmic rays at Earth: researchers reference manual and data book.

\bibitem[8] {8}
N.Karmakar, A.Paul, N.Chaudhuri, Measurements of absolute intensities of cosmic-ray muons
in the vertical and greatly inclined directions at geomagnetic latitudes $\SI{16}{\degree N}$, 
{\em Nuovo Cim. B} 17 (1973) 173.

\bibitem[9] {9}
M.S. Sinha, N. Basu,
{\em Indian J. Phys.} 33 (1959) 335.

\bibitem[10] {10}
S.Pal, B.Acharya, G.Majumder, N.Mondal, D.Samuel, B.Satyanarayana, Measurement of integrated flux of cosmic ray muons at sea level using the INO-ICAL prototype detector,
{\em J. Cosmol. Astropart. Phys.} 07 (2012) 033.

\bibitem[11] {11}
O.C.Allkofer, R.D.Andresen, W.D.Dau, The muon spectra near the geomagnetic equator,
{\em Can. J. Phys.} 46 (1968) S301.

\bibitem[12] {12}
S. Pethuraja,b, V.M. Datarb, G. Majumderb, N.K. Mondalc, K.C. Ravindranb and B. Satyanarayanab, 
Measurement of cosmic muon angular distribution and vertical integrated flux by $\SI{2}{m} \times \SI{2}{m}$ RPC stack at IICHEP-Madurai, 
{\em J. Cosmol. Astropart. Phys.} 09 (2017) 021. 

\end{thebibliography}


\end{document}