\subsection{Misure e osservazioni preliminari}

\subsubsection{Punto di lavoro del PMT2}

Vogliamo trovare un buon punto di lavoro per il PMT2 che è alimentato con tensioni positive fino a $\sim \SI{800}{V}$. 
Mandiamo il segnale preamplificato del PMT2 all'amplificatore e la sua uscita all'ADC in modalità automatica.
Scegliamo di alimentare il PMT2 alla tensione consigliata di \SI{650}V. A questa tensione (e con guadagno dell'amplificatore impostato a $\times 1$) il fotopicco generato dai fotoni più energetici tra le sorgenti a disposizione (ovvero il picco a \SI{1.33}{MeV} del \co) si trova a $2/3$ della scala. Questo permette, agendo sul fattore di amplificazione del formatore/amplificatore di portare il fotopicco a fondo scala  e massimizzare la risoluzione dello spettrometro, mantenendo allo stesso tempo flessibilità nel caso la risposta del rivelatore non si rivelasse stabile\footnote{Cosa che abbiamo verificato essere vera: la posizione dei fotopicchi può variare nel tempo fino al $\sim 10\%$.}.

\subsubsection{Stabilità della risposta del PMT2}

Notiamo che la risposta dello spettrometro è molto sensibile alle variazioni della tensione di alimentazione:
variando da \SI{600}V a \SI{575}V, la posizione dei fotopicchi si dimezza.
Il manuale dell'alimentatore riporta una stabilità dello \SI{0.5}{\permil} su \SI{24}h e dell'\SI1{\permil} su un mese. Volendo fare una stima otteniamo variazioni del $2/25 \times 600/1000 = \SI5\%$ in un mese, la metà in \SI{24}h. Sembra quindi che la stabilità della risposta del nostro spettrometro sia un limite importante nella nostra misura per cui nel seguito analizzeremo approfonditamente questo parametro.
%%%%%%%%%%%%%%%%%%%%%%%%%%%%%%%%%%%%%%%%%%%%%%%%%%%%%%%%%%%

\subsubsection{Interazione con il bersaglio}

Poniamo lo scintillatore plastico (PMT1) davanti alla sorgente in modo che i raggi~$\gamma$ possano interagire e subire scattering Compton. Osserviamo l'istogramma dei campionamenti del nostro spettrometro a vari angoli rispetto al bersaglio (ne riportiamo due in \autoref{4ang}) mentre l'ADC campiona in modalità automatica.
Il range dei fotoni del \co\; nel bersaglio è $\sim\SI{10}{cm}$ e il suo spessore $\sim\SI{1}{cm}$, quindi solo il $10\%$ dei fotoni interagisce. 
Se lo spettrometro si trova a piccolo angolo rispetto al bersaglio viene attraversato dai fotoni che non hanno interagito e lo spettro è sostanzialmente uguale a quello acquisito senza bersaglio. A grande angolo vorremmo vedere i fotoni che fanno Compton sul bersaglio e arrivano con un certo angolo di scattering sul rivelatore (producendo quindi uno spettro simile al precedente ma traslato a energie minori) ma i fotoni che non interagiscono proseguono sino al muro subito dietro lo spettrometro, una parte di questi fa scattering Compton o Rayleigh e può arrivare nello spettrometro, sovrastando in statistica il segnale che vorremmo osservare.

\begin{figure}[h]
	\centering
	\newcommand*\mywidth{17.5em}

	\subfloat
	{\includegraphics[width=\mywidth]{0g}}
	\hfill
	\subfloat
	{\includegraphics[width=\mywidth]{90g}}

	\caption{Spettri raccolti senza coincidenza a diversi angoli rispetto al bersaglio.}
	\label{4ang}
\end{figure}

\marginpar{La parte che segue è abbastanza inutile se non aggiungiamo la spiegazione che dicevo prima (Bob)}
Abbiamo poi confrontato lo spettro a \SI{45}{\degree} con o senza scintillatore plastico, ma non abbiamo notato nessuna differenza. Abbiamo anche verificato che, come atteso, il rate di eventi diminuisce all'aumentare della distanza.
\marginpar{forse alcune di queste considerazioni sono stupide, ma è più facile togliere che aggiungere}

Proviamo a schermare lateralmente lo scintillatore cristallino per ridurre il fondo di backscattering e lo spettro in \autoref{casetta} mostra come effettivamente questo si riduca ma non a sufficienza da poter osservare il segnale. Si rende necessario l'uso di un trigger esterno per campionare sulla coincidenza di segnale tra PMT1 e PMT2.

\begin{figure}
\centering
\includegraphics[width=25 em]{45gs}
\caption{Spettro energetico del PMT2 con schermatura per ridurre il backscattering.}
\label{casetta}
\end{figure}

\subsubsection{Trigger esterno}
Per ottenere il segnale di trigger usiamo il circuito in \autoref{fig:trigger}:\marginpar{aggiungere figura?(Bob)}
\begin{itemize}
	\item il segnale "veloce" del PMT2 (quello non pre-amplificato con ampiezza di picco $\sim\SI{20}{mV}$) è prima amplificata di un fattore 10 attraverso un modulo di amplificazione lineare, quindi inviato ad un discriminatore con soglia a $\SI{35}{mV}$ (il minimo) in modo da ottenere una soglia efficace più bassa;
	\item il segnale del PMT1 è inviato direttamente ad un discriminatore con soglia a $\SI{35}{mV}$ (il minimo);
	\item entrambi gli output dei discriminatori vanno ad un sistema di anti-rettriger che impone un tempo morto di $\sim\SI{1}{\micro s}$ e produce segnali di durata $\sim\SI{30}{ns}$;
	\item il segnale discriminato di uno dei PMT è ritardato di $\sim\SI{10}{ns}$ per compensare un ritardo asimmetrico dovuto al modulo di anti-retrigger;
	\item facciamo la coincidenza dei due segnali così formati e mandiamo l'output al un modulo \texttt{gate \& delay}.
	\item l'ADC richiede un segnale di trigger che parta almeno \SI{200}{ns} prima del picco e termini almeno \SI{200}{ns} dopo: usando l'oscilloscopio, regoliamo il modulo \texttt{gate \& delay} in modo da avere un segnale della durata di $\sim\SI{1}{\micro s}$ centrato sul picco del formatore.
\end{itemize}
Acquisiamo degli spettri di prova e verifichiamo che con le soglie impostate i primi $\sim1000$ canali dell'ADC non mostrano campionamenti.

\subsubsection{Punto di lavoro del PMT1}
Il PMT1 può essere alimentato fino a \SI{1800}{V} e nel range \SI{1500}{V}-\SI{1700}{V} mostra circa lo stesso rate. Decidiamo di alimentarlo a \SI{1700}{V} per avere alta efficienza e mantenendo il rate ragionevole.
