\subsection{Circuito per le misure in coincidenza}

Per misurare soltanto l'energia dei fotoni che hanno subito diffusione Compton, realizziamo una coincidenza temporale tra i due PMT. Sappiamo che il PMT1 deve avere un'alimentazione inferiore ai \SI{1800}V; facendo varie prove notiamo che il rate di coincidenze tra \SI{1500}V e \SI{1800}V rimane circa invariato. Impostiamo allora la tensione di alimentazione del PMT1 a \SI{1700}V in modo da avere un'alta efficienza ed essere lontani da zone in cui il dispositivo è malfunzionante.
\marginpar{Ho letto dal logbook che questo lo abbiamo fatto dopo, ma penso che mentire a favore del filo logico non ci nuocia in questo caso.\\
Forse dobbiamo parlare dei pocci con l'attenuatore  e dei retrigger.}

Il circuito di misura è stato poi ottimizzato e segue lo schema di \autoref{sc_conc}. Il segnale positivo del PMT2 è rimasto collegato al formatore come nelle misure precedenti (\autoref{???}) e la sua uscita negativa, invece di essere usata come trigger dell'ADC, viene inviata, dopo le opportune precauzioni di antiretrigger, ad un modulo di coincidenze%
\footnote{Stiamo sottintendendo che i conteggi dei PMT ed il numero di coincidenze siano mostrati dal contatore. Anche loro vengono da cavi che hanno superato il circuito di antiretrigger.}.

Il segnale del PMT1  viene ritardato di \SI{22}{ns} per essere messo in tempo con il discriminatore dell'altro PMT. Esso è poi collegato al discriminatore che poi va al \texttt{gate \& delay} che funge da antiretrigger.
\marginpar{in realtà ci sono stati lo stesso degli antiretrigger ma molto meno del poccio che fanno i discriminatori}

L'uscita del modulo di coincidenza è poi portata ad un altro ingresso del modulo \texttt{gate \& delay} che la ritarda e la allunga per avere un segnale di trigger con le stesse caratteristiche della sezione precedente.

%%%%%%%%%%%%%%%%%%%%%%%%%%%%%%%%%%%%%%%%%%%%%%%%%%%%%%%%%%%%%%%

\subsection{Caratteristiche geometriche del fascio}

\subsubsection{Accorgimenti sperimentali}

Ci interessa misurare la divergenza angolare del fascio.
Per farlo posizioniamo il PMT2 di fronte alla sorgente a vari angoli e triggeriamo sul PMT2 stesso. Lo scopo è disegnare un grafico dei rate in funzione dell'angolo.

Per conoscere il tempo di acquisizione montiamo un circuito che, usando un timer,
fa partire e fermare simultaneamente il trigger e il contatore.

Per misurare meglio gli angoli abbiamo stampato un goniometro che abbiamo posizionato sotto il supporto trasparente della guida su cui scorre il PMT2. Infine abbiamo tracciato un segmento sulla parte trasparente che ci permette di capire a quale angolo posizioniamo il PMT2.
Lo zero degli angoli è riferito al goniometro e otterremo il centro del fascio dalle misure.

\subsubsection{Analisi dei dati e risultati}

Scegliamo di misurare il rate solo in un intervallo di energia che contiene i due fotopicchi,
per evitare di includere eventuali fondi di raggi $\gamma$ che non provengono direttamente dalla sorgente.
L'intervallo è abbastanza largo che il piccolo spostamento dei fotopicchi al variare dell'angolo del PMT2 non fa intersecare i fotopicchi con i bordi.
Facciamo anche una misura di <<fondo costante>> con il PMT2 posizionato in un punto non visibile dalla sorgente,
ma con la sorgente aperta.

Le misure sono riportate in \autoref{tabfo}.
La misura di fondo costante risulta praticamente trascurabile.
In \autoref{forma} sono riportati i dati con sottratto il fondo costante.
Si nota che la forma sembra essere la somma di due gaussiane (che sono parabole in scala logaritmica).
Chiamiamo la gaussiana più larga <<fondo piccato>>
e la interpretiamo come raggi che non provengono direttamente dalla sorgente
perché si nota un'asimmetria rispetto al centro, che non c'è nella gaussiana stretta (il <<segnale>>)
e che non corrisponde alla geometria cilindrica del collimatore.

Se fittiamo una gaussiana per i dati dagli angoli \SI{-5}\degree{} a \SI5\degree{} otteniamo già un buon accordo.
Tuttavia, vista la nostra interpretazione, riteniamo più sensato sottrarre il fondo piccato.
A causa dell'asimmetria del fondo, se fittiamo le due gaussiane non otteniamo un buon accordo
e la gaussiana di segnale viene fittata visibilmente peggio.
In sostanza: sappiamo modellare bene il segnale ma non il fondo.
Allora estraiamo il fondo dal fit delle due gaussiane,
lo sottraiamo e fittiamo la sola gaussiana di segnale da \SI{-4}\degree{} a \SI4\degree.

La deviazione standard del segnale risulta \SI{2.52 \pm 0.03}{\degree},
il centro \SI{-0.09 \pm 0.04}{\degree}.
Dal segnale dobbiamo deconvolvere la forma dello scintillatore NaI,
la cui deviazione standard è \SI{1.82 \pm 0.07}\degree, ottenuta sapendo che è un cilindro di raggio \SI{2}{''}
e che dista \SI{40\pm 1.5}{cm} (l'incertezza sulla distanza è data dalla profondità dello scintillatore e dalla larghezza del collimatore).
La deviazione standard della forma del fascio si ottiene semplicemente dalla sottrazione in quadratura
e risulta \SI{1.73 \pm 0.09}{\degree}.

\begin{table}
	\centering

	\begin{tabular}{cc|cc|cc}
		angolo [\si{\degree}] & rate [\si{s^{-1}}] & angolo [\si{\degree}] & rate [\si{s^{-1}}] & angolo [\si{\degree}] & rate [\si{s^{-1}}] \\
		\hline
			$-20$ & $102.8\,\pm\,1.1$ & $-4$ & $1310\,\pm\,6$  & $5$ & $544\,\pm\,4$       \\
			$-15$ & $149.6\,\pm\,1.4$ & $-3$ & $2092\,\pm\,8$  & $6$ & $205.5\,\pm\,2.1$   \\
			$-10$ & $191.4\,\pm\,1.7$ & $-2$ & $2847\,\pm\,9$  & $7$ & $173.2\,\pm\,1.9$   \\
			$-9$ & $208.0\,\pm\,1.8$  & $-1$ & $3521\,\pm\,11$ & $8$ & $173.7\,\pm\,1.6$   \\
			$-8$ & $215.2\,\pm\,1.8$  & $0$ & $3792\,\pm\,11$  & $9$ & $174.7\,\pm\,1.6$   \\
			$-7$ & $228.3\,\pm\,2.2$  & $1$ & $3426\,\pm\,10$  & $10$ & $170.2\,\pm\,1.6$  \\
			$-6$ & $298\,\pm\,3$      & $2$ & $2695\,\pm\,9$   & $15$ & $118.5\,\pm\,1.2$  \\
			$-5$ & $680\,\pm\,5$      & $3$ & $1930\,\pm\,8$   & $20$ & $78.4\,\pm\,0.9$   \\
			     &                  & $4$ & $1160\,\pm\,6$   &      &
	\end{tabular}

	\caption{Rate senza coincidenza nei canali dell'ADC 5500--8040
	al variare dell'angolo del PMT2.
	L'errore sugli angoli è 0.1$^{\circ}$.
	Il fondo costante misurato è \SI{4.43\pm0.2}{s^{-1}}.}
	\label{tabfo}
\end{table}

\begin{figure}
	\centering
	\includegraphics[width=32em]{forma}
	\caption{Dati e fit della forma del fascio.
	Nella legenda, <<fondo p.>> sta per <<fondo piccato>> e <<fondo c.>> per <<fondo costante>>.}
	\label{forma}
\end{figure}



