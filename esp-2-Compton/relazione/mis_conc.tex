\subsection{Misure in coincidenza}

Per misurare soltanto l'energia dei fotoni che hanno subito diffusione Compton, realizziamo una coincidenza temporale tra i due PMT. Sappiamo che il PMT1 deve avere un'alimentazione inferiore ai \SI{1800}V; facendo varie prove notiamo che il rate di coincidenze tra \SI{1500}V e \SI{1800}V rimane circa invariato. Impostiamo allora la tensione di alimentazione del PMT1 a \SI{1700}V in modo da avere un'alta efficienza ed essere lontani da zone in cui il dispositivo è malfunzionante.
\marginpar{Ho letto dal logbook che questo lo abbiamo fatto dopo, ma penso che mentire a favore del filo logico non ci nuocia in questo caso.\\
Forse dobbiamo parlare dei pocci con l'attenuatore  e dei retrigger.}

Il circuito di misura è stato poi ottimizzato e segue lo schema di \autoref{sc_conc}. Il segnale positivo del PMT2 è rimasto collegato al formatore come nelle misure precedenti \autoref{???} 