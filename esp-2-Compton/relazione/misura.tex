\subsection{Misura finale}

\subsubsection{Prese dati}

\begin{table}
	\centering
	\begin{tabular}{llll}
		\toprule
		Data & Durata & Calibrazioni & Angolo \\
		\midrule
		19--20 feb    & 20 ore & dopo, limitata &           45.0\,\si{\degree} \\
		20--22 feb    & 40 ore & dopo           &   61$\sfrac34$\,\si{\degree} \\
		26--27 feb    & 22 ore & prima e dopo   &           15.0\,\si{\degree} \\
		27 feb--1 mar & 38 ore & prima          & \phantom{1}7.0\,\si{\degree} \\
		\bottomrule
	\end{tabular}
	\caption{\label{tab:misure}
	Prese dati per ricavare la massa dell'elettrone.}
\end{table}

Per ricavare la massa dell'elettrone usiamo quattro misure di spettro.
Tutte le misure durano circa 24 ore per avere sufficiente statistica (si è poi rivelata eccessiva).
Era nelle nostre intenzioni avere una calibrazione prima e dopo ogni misura,
ma la rottura dell'alimentatore del crate ha ostacolato i nostri piani.
La misura dopo la quale il crate si è rotto ha la calibrazione solo prima;
due misure che abbiamo recuperato e che inizialmente erano solo misure di prova hanno la calibrazione solo dopo,
una delle quali non era intesa come misura di calibrazione e quindi ha solo il cobalto
(calibrazione 20feb in \autoref{tab:calibration}).
L'elenco delle misure è riportato in \autoref{tab:misure}.
