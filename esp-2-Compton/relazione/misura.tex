\subsection{Misura finale}\label{par:fondo2}

\subsubsection{Prese dati}

\begin{table}
	\centering
	\begin{tabular}{lllll}
		\toprule
		Data & Durata & Calibrazioni & Dati & Angolo \\
		\midrule
		19--20 feb    & 20 ore & dopo, solo \co & istogramma &           45.0\,\si{\degree} \\
		20--22 feb    & 40 ore & dopo           & istogramma &   61$\sfrac34$\,\si{\degree} \\
		26--27 feb    & 22 ore & prima e dopo   & campioni   &           15.0\,\si{\degree} \\
		27 feb \!--\! 1 mar & 38 ore & prima          & campioni   & \phantom{1}7.0\,\si{\degree} \\
		\bottomrule
	\end{tabular}
	\caption{\label{tab:misure}
	Prese dati per ricavare la massa dell'elettrone.}
\end{table}

Per ricavare la massa dell'elettrone usiamo quattro misure di spettro in coincidenza.
Tutte le misure durano circa 24 ore per avere sufficiente statistica (si è poi rivelata eccessiva).
Era nelle nostre intenzioni avere una calibrazione prima e dopo ogni misura
e avere l'elenco dei campioni anziché solo l'istogramma,
ma la rottura dell'alimentatore del crate ha ostacolato i nostri piani.
La misura dopo la quale il crate si è rotto ha la calibrazione solo prima;
due misure che abbiamo recuperato e che inizialmente erano solo misure di prova hanno la calibrazione solo dopo,
una delle quali non era intesa come misura di calibrazione e quindi ha solo il cobalto
(calibrazione 20feb in \autoref{tab:calibration}).
L'elenco delle misure è riportato in \autoref{tab:misure}.

\subsubsection{Funzione empirica spalla Compton}
\label{sec:empirical}

\begin{figure}
	\centering
	\includegraphics[width=33em]{empirical-secondary}
	\caption{\label{fig:empirical-secondary}
	Funzione empirica per descrivere la spalla Compton,
	ricavata con un fit della somma di una gaussiana, una log-gaussiana e la funzione di Fermi.
	Il Monte Carlo mostrato è ad angolo \SI{15}{\degree} e fotone della sorgente di \SI{1.33}{MeV}.}
\end{figure}

\begin{figure}
	\centering
	\includegraphics[width=28em]{empirical-test}
	\caption{\label{fig:empirical-test}
	Simulazione della spalla Compton a \SI{45}{\degree} e fotone della sorgente a \SI{1.33}{MeV}
	con diversi valori della massa dell'elettrone,
	fittata con la funzione empirica con solo un parametro di scala libero
	ed escludendo la parte sinistra della curva.}
\end{figure}

Per fare il fit della spalla Compton,
la simuliamo e fittiamo sulla simulazione una funzione empirica a 10 parametri
(vedi \autoref{fig:empirical-secondary});
per usarla sui dati blocchiamo tutti i parametri tranne un parametro di scala.
In \autoref{fig:empirical-test} verifichiamo che,
per valori della massa dell'elettrone di \SI{0.4}{MeV} e \SI{0.6}{MeV},
la funzione (ottenuta con massa \SI{0.511}{MeV}) descrive ancora sufficientemente bene la curva simulata.\footnotemark
\footnotetext{Una descrizione così accurata della spalla Compton
sarebbe servita per il punto facoltativo della scheda,
ovvero ricavare la massa dal fit della spalla.
Alla fine non abbiamo fatto il punto facoltativo.}

\subsubsection{Fit degli spettri}

\begin{figure}
	\hspace{-8em}\includegraphics[height=0.9\textwidth]{fit}
	\caption{\label{fig:fit}
	Spettri del rilascio di energia nel NaI
	fittati con la somma di due gaussiane, due spalle Compton e un'esponenziale,
	oppure solo con due gaussiane e un'esponenziale (<<fondo semplificato>>).
	La presa dati a \SI{61}{\degree} è fittata solo con il fondo semplificato
	perché il fit completo è instabile.
	Nelle legende è eventualmente riportata
	la frazione temporale usata della presa dati.}
\end{figure}

Fittiamo ogni spettro con la somma di due gaussiane,
due spalle Compton espresse come in \autoref{sec:empirical}
e un'esponenziale.
Per verificare la dipendenza dal modello dei fondi
fittiamo anche senza spalle Compton, con solo l'esponenziale.
Quando i fotopicchi sono sovrapposti
fissiamo la normalizzazione relativa delle gaussiane a quella ottenuta dalla simulazione.
Imponiamo che il centro della gaussiana corrispondente al fotopicco a energia maggiore
sia maggiore del centro dell'altra.
Gli spettri e i fit sono riportati in \autoref{fig:fit}.

\subsubsection{Misura della massa}

\begin{figure}
	\hspace{-5em}\includegraphics[width=46em]{me}
	\caption{\label{fig:me}
	Misure della massa dell'elettrone.
	Nel grafico a sinistra quelle ottenute da tutti i fit di spettri,
	anche nelle finestre temporali spostate usate per stimare l'incertezza di stabilità.
	Nel grafico a destra solo quelle usate per la media,
	con inclusa l'incertezza di stabilità nelle barre di errore.}
\end{figure}

Convertiamo i centri dei fotopicchi ottenuti dai fit degli spettri
da digit a MeV usando la calibrazione più vicina a ogni presa dati.
Per ogni fotopicco otteniamo la massa dell'elettrone invertendo la \eqref{energia_compton}:
\begin{equation}
	\label{eq:me}
	m_e = \frac{1-\cos\theta}{\frac1{E'} - \frac1E}.
\end{equation}

\paragraph{Bias}

Per ogni fit, simuliamo i soli fotopicchi e li fittiamo con due gaussiane
per stimare il bias del fit. Calcoliamo direttamente il bias sulla massa dell'elettrone.
I~bias risultano circa \SI{-0.01}{MeV} per \SI{15}{\degree} e \SI{7}{\degree},
e dell'ordine di \SI{\pm0.001}{MeV} per \SI{45}{\degree} e \SI{61}{\degree}
(in cui la normalizzazione relativa delle gaussiane è imposta).
Sottraiamo i bias dalle masse calcolate.
Le masse sono riportate in \autoref{fig:me}.

\paragraph{Stabilità}

Per le prese dati di cui abbiamo i campioni anziché solo l'istogramma,
fittiamo lo spettro solo in una finestra temporale vicina alla calibrazione,
e ripetiamo il fit spostando la finestra.
Come incertezza sulla stabilità prendiamo la differenza
tra il risultato sulla finestra principale e sulla prima vicina,
prendendo la massima tra quella ottenuta con il fotopicco corrispondente a \SI{1.33}{MeV}
e quella del fotopicco \SI{1.17}{MeV}.

Quando abbiamo solo l'istogramma,
stimiamo l'incertezza di stabilità in questo modo:
dalla \eqref{eq:scalibur} e dalla \eqref{eq:me}
si ottiene che la variazione relativa della massa misurata dell'elettrone
al variare della calibrazione è
\begin{equation}
	\label{eq:scal}
	r \is \frac{\Delta m_e}{m_e}
	= \frac{1-\cos\theta}{\left(1-\frac{E'}E\right)^2}
	\frac{\frac1{E'} - \frac1E}{1-\cos\theta}
	\left(\frac{\Delta E'}{E'}\right) E'
	= \frac1{1-\frac{E'}E} \left(\frac{\Delta E'}{E'}\right).
\end{equation}
Il rapporto tra le variazioni relative in due misure diverse $a$ e $b$ è
\begin{equation*}
	\frac{r_b}{r_a}
	= \frac{1-\frac{E'_a}E}{1-\frac{E'_b}E}
	\frac{\left(\frac{\Delta E'}{E'}\right)_b}{\left(\frac{\Delta E'}{E'}\right)_a},
\end{equation*}
dove $E'_a$ e $E'_b$ sono espressi dalla \eqref{energia_compton}.
Se il tempo di acquisizione è diverso,
stimiamo la variazione della scala di $b$ come un random walk:
\begin{equation*}
	\left(\frac{\Delta E'}{E'}\right)_b = \left(\frac{\Delta E'}{E'}\right)_a \sqrt{\frac{t_b}{t_a}};
\end{equation*}
questa stima è conservativa perché l'effetto è integrato sul tempo.
Infine:
\begin{equation*}
	r_b = r_a
	\frac{1-\frac{E'_a}E}{1-\frac{E'_b}E}
	\sqrt{\frac{t_b}{t_a}}.
\end{equation*}
Nel nostro caso per $a$ prendiamo la presa dati a \SI{15}{\degree}
perché è quella con la maggiore incertezza di stabilità.

\paragraph{Misura a \SI{7}{\degree}}

Non usiamo la massa ottenuta dalla misura a \SI{7}{\degree}
perché risulta fortemente incompatibile con le altre misure.
Lo riteniamo plausibile perché, a questo angolo,
dalla \eqref{eq:scal} si ottiene che una variazione della scala dell'\SI1\%
provoca una variazione della massa misurata di \SI{0.27}{MeV},
quindi diventa molto significativo il tempo che intercorre tra la calibrazione
e l'inizio della presa dati.

\paragraph{Fondo semplificato}

Dalla \autoref{fig:me} si nota che i fit con fondo semplificato
non danno risultati significativamente diversi da quelli completi.
Inoltre, come si vede in \autoref{fig:fit},
i fit semplificati sono intenzionalmente poco sensati.
Fa eccezione la misura a \SI{61}{\degree}
in cui il fondo è effettivamente descritto dal modello semplificato.
Quindi per le misure useremo il fit completo,
tranne per quella a \SI{61}{\degree} in cui usiamo quello semplificato.

\paragraph{Media delle misure}

\begin{table}
	\centering
	\begin{tabular}{cccc|cccc}
		\toprule
		\multicolumn{4}{c|}{$E=\SI{1.33}{MeV}$} & \multicolumn{4}{c}{$E=\SI{1.17}{MeV}$} \\
		$\SI{15}{\degree}_\text{inizio}$ & $\SI{15}{\degree}_\text{fine}$ & \SI{61}{\degree} & \SI{45}{\degree} & $\SI{15}{\degree}_\text{inizio}$ & $\SI{15}{\degree}_\text{fine}$ & \SI{61}{\degree} & \SI{45}{\degree} \\
		\midrule
  0.46(6) &     0.4\,\% &    0.0\,\% &    0.0\,\%  &  98.4\,\% &     0.3\,\% &    0.0\,\%  &   0.0\,\% \\
          & 0.463(23) &    0.1\,\% &    0.1\,\%  &   0.4\,\% &    91.5\,\% &    0.1\,\%  &   0.1\,\% \\
          &           &  0.52(7) &    0.0\,\%  &   0.0\,\% &     0.1\,\% &   96.4\,\%  &   0.0\,\% \\
          &           &          &  0.49(8)  &   0.0\,\% &     0.1\,\% &    0.0\,\%  &  98.9\,\% \\
          &           &          &           & 0.46(6) &     0.3\,\% &    0.0\,\%  &   0.0\,\% \\
          &           &          &           &         & 0.477(27) &    0.1\,\%  &   0.1\,\% \\
          &           &          &           &         &           &  0.56(7)  &   0.0\,\% \\
          &           &          &           &         &           &           & 0.50(9) \\
		\bottomrule
	\end{tabular}
	\caption{\label{tab:cov}
	Misure di massa (in \si{MeV}) usate per la media finale con la matrice di correlazione.}
\end{table}

Calcoliamo la media pesata con la matrice di covarianza delle varie misure di massa.
Calcoliamo anche separatamente le medie per le masse corrispondenti a $E=\SI{1.33}{MeV}$ e $E=\SI{1.17}{MeV}$.
Le masse usate per la media sono riportate in \autoref{tab:cov} e graficamente in \autoref{fig:me}.
Otteniamo (in \si{MeV}):
\begin{align*}
	m_e^{(1.33)}                &= \num{0.468 \pm 0.020}  ,    \\
	m_e^{(1.17)}                &= \num{0.483 \pm 0.023}  ,    \\
	m_e^{(1.17)} - m_e^{(1.33)} &= \num{0.014 \pm 0.008}  \quad(1.7\,\sigma),   \\
	m_e                         &= \num{0.466 \pm 0.020} = \\
	                            &=      0.466 \pm 0.020^{(\text{sist})} \pm 0.004^{(\text{stat})} = \\
										 &=      0.466 \pm 0.016^{(\text{stab})} \pm 0.010^{(\text{cal})} \pm 0.005^{(\text{ang})} \pm 0.004^{(\text{stat})}.
\end{align*}
Il $\chi^2$ della media pesata è 6.5, per 7 gradi di libertà.
La nostra misura dista $2.3\,\sigma$ dal valore noto \SI{0.511}{MeV}.
% Se imponiamo che la differenza tra $m_e^{(1.33)}$ e $m_e^{(1.17)}$ sia $1\,\sigma$
% aggiungendo due incertezze uguali $\sigma_\text{comp}=0.012$ alle due masse, otteniamo
% \begin{align*}
% 	m_e^{(\text{comp})}                &= \num{0.485 \pm 0.020} = \\
% 	                            &=      0.485 \pm 0.014^{(\text{stab})} \pm 0.010^{(\text{cal})} \pm 0.009^{(\text{comp})} \pm 0.004^{(\text{ang})} \pm 0.002^{(\text{stat})};
% \end{align*}
% $m_e^{(\text{comp})}$ dista $1.3\,\sigma$ da 0.511.
