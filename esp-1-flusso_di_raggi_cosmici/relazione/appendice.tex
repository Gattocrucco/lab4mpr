\appendix
\section{Varianza del rapporto di conteggi\protect\footnote{Questa parte è presa con pochi cambiamenti dalla nostra relazione preliminare.}}
\label{sec:vareff}

Consideriamo i conteggi di due coincidenze $A$ e $B$,
in cui la coincidenza $B$ contiene tutte le lastre della $A$ più un'altra diversa.
Chiamiamo $k_A$ e $k_B$ i rispettivi conteggi.
Assegnamo una distribuzione poissoniana a $k_A$ e binomiale a $k_B$ dato $k_A$:
\begin{align*}
	P(k_A;\mu)
	&= \frac{\mu^{k_A}}{k_A!}e^{-\mu} \\
	P(k_B|k_A;\epsilon)
	&= \binom{k_A}{k_B} \epsilon^{k_B} (1-\epsilon)^{k_A-k_B},
\end{align*}
ovvero $\epsilon$ è la probabilità che un raggio che fa scattare $A$
faccia scattare anche la lastra in più in $B$.
Definiamo
\begin{equation*}
	\hat\epsilon := \frac{k_B}{k_A}
\end{equation*}
e studiamone le proprietà come stimatore di $\epsilon$.
Calcoliamo la distribuzione congiunta dei conteggi:
\begin{align*}
	P(k_A,k_B;\mu,\epsilon)
	&= P(k_A;\mu) P(k_B|k_A;\epsilon) = \\
	&= \frac{e^{-\mu}}{k_B!(k_A-k_B)!} \big(\mu(1-\epsilon)\big)^{k_A} \left(\frac\epsilon{1-\epsilon}\right)^{k_B},
	\quad k_B \le k_A.
\end{align*}
Notiamo che $\hat\epsilon$ è definito per $k_A\neq 0$.
Restringiamo il dominio a $k_A\neq 0$,
quindi ricalcoliamo la normalizzazione:
\begin{align*}
	P(k_A=0,k_B=0)
	&= e^{-\mu} \implies \\
	\implies P(k_A\neq 0,k_B)
	&= \frac{P(k_A,k_B)}{1-e^{-\mu}}.
\end{align*}
Calcoliamo il valore atteso di $\hat\epsilon$:
\begin{align*}
	E[\hat\epsilon]
	&= E \left[ \frac{k_B}{k_A} \right] = \\
	&= \sum_{k_A\ge k_B} \frac{k_B}{k_A} P(k_A) P(k_B|k_A) = \\
	&= \sum_{k_A=1}^\infty \frac{P(k_A)}{k_A}
	\sum_{k_B=0}^{k_A} P(k_B|k_A) k_B = \\
	\intertext{riconosciamo che la seconda somma è la media della binomiale}
	&= \sum_{k_A=1}^\infty \frac{P(k_A)}{k_A} k_A \epsilon = \\
	&= \epsilon \sum_{k_A=1}^\infty P(k_A)
	= \epsilon,
\end{align*}
quindi $\hat\epsilon$ ha bias nullo.
Calcoliamo la varianza:
\begin{align*}
	\operatorname{Var}[\hat\epsilon]
	&= E[\hat\epsilon^2] - E[\epsilon]^2 \\
	E[\hat\epsilon^2]
	&= \sum_{k_A=1}^\infty \frac{P(k_A)}{k_A^2}
	\sum_{k_B=0}^{k_A} P(k_B|k_A) k_B^2 = \\
	\intertext{la seconda somma è $E[k_B^2|k_A]$}
	&= \sum_{k_A=1}^\infty \frac{P(k_A)}{k_A^2}
	\big (k_A\epsilon(1-\epsilon) + k_A^2\epsilon^2 \big) = \\
	&= \epsilon(1-\epsilon) \sum_{k_A=1}^\infty \frac{\mu^{k_A}}{k_Ak_A!}\frac{e^{-\mu}}{1-e^{-\mu}}
	+ \epsilon^2 \sum_{k_A=1}^\infty P(k_A) = \\
	&= \frac{\epsilon(1-\epsilon)}{e^{\mu}-1} \sum_{k_A=1}^\infty \frac{\mu^{k_A}}{k_Ak_A!} + \epsilon^2.
\end{align*}
Si può dimostrare che\footnote{L'abbiamo calcolato con WolframAlpha.}
\begin{align*}
	\sum_{k_A=1}^\infty \frac{\mu^{k_A}}{k_Ak_A!}
	&= \Ei(\mu) - \log\mu - \gamma,
\end{align*}
dove $\Ei$ è la funzione integrale esponenziale che è già implementata nelle librerie standard
e $\gamma$ è la costante di Eulero-Mascheroni $\approx 0.6$.
Quindi infine
\begin{align*}
	\operatorname{Var}[\hat\epsilon]
	&= \epsilon(1-\epsilon)\frac{\Ei(\mu) - \log\mu - \gamma}{e^\mu - 1}.
\end{align*}
Vediamo l'andamento per $\mu$ grande.
Vale $\Ei(\mu) \approx e^{\mu}/\mu$, dunque
\begin{align*}
	\operatorname{Var}[\hat\epsilon]
	&\approx \frac{\epsilon(1-\epsilon)}{\mu}.
\end{align*}

\paragraph{Sviluppo asintotico in $\mu$}

Dimostriamo lo sviluppo asintotico della funzione Ei usato precedentemente
e calcoliamo anche i termini superiori al primo.
L'integrale esponenziale è definito come
\begin{equation*}
	\operatorname{Ei}(x)
	= - \fint_{-x}^\infty \de t \frac{e^t}t
	= \fint_{-\infty}^x \de t \frac{e^t}t,
\end{equation*}
quindi la derivata è
\begin{equation*}
	\Ei'(x) = \frac{e^x}x.
\end{equation*}

