\subsection{Luce ambientale}

Prima di effettuare le misure abbiamo controllato quanto la luce della stanza potesse influenzare i conteggi dei fotomoltiplicatori.
Puntiamo la torcia del cellulare su vari punti delle lastre, delle guide di luce e sull'attacco tra esse ed i PMT. Avendo già sistemato i PM2, PM3 e PM4 nell'esperienza preliminare (abbiamo utilizzato lo stesso apparato), elenchiamo le criticità maggiormente riscontrate negli scintillatori restanti. Solitamente le infiltrazioni luminose avvengono sulla giunzione tra le guide di luce ed i PMT, abbiamo trovato infiltrazioni negli angoli ed anche un buco nel nastro adesivo nero che ricopriva uno scintillatore. Attraverso di esso potevamo intravedere la carta argentata. Per quanto riguarda il miniscint, molta luce entrava dalla parte anteriore della sua lastrina scintillante.
Abbiamo coperto tutti i buchi con dello scotch nero.

\marginpar{collegare le 2 parti}
Due settimane dopo, analizzando il rumore anomalo, ci siamo accorti che posizionando un telo nero sul nostro apparato i conteggi dimezzavano. Questo accorgimento sembrava aver migliorato la situazione, ma misure seguenti senza telo hanno smentito questa impressione. Parleremo più in dettaglio di questa situazione nella \autoref{rumore}.