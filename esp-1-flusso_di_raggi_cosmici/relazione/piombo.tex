\subsection{Effetto del materiale sulla misura}

Abbiamo testato quanto la presenza di lastre di metallo potesse influire sulla rivelazione dei raggi cosmici,
per indagare quanto possa essere rilevante il materiale circostante (soffitto, pareti).
Abbiamo a disposizione:
\begin{itemize}
	\item 3 lastre di piombo rivestite di alluminio, ciascuna di spessore totale \SI{4}{mm},
	\item 1 lastra di alluminio di spessore \SI{4}{mm},
	\item 10 lastre di piombo spesse \SI{2}{mm}.
\end{itemize}
Non abbiamo misurato lo spessore della parte di piombo di quelle con piombo e alluminio.
Le lastre di metallo ricoprono circa l'\SI{80}\% delle lastre di scintillatore. 

\subsubsection{Conteggi}

Abbiamo posizionato le lastre di metallo tra il PM3 ed il PM4
per vedere se esse riuscivano a fermare parte dei muoni.
Per fare questa misura abbiamo confrontato il rapporto tra le coincidenze 5\&4\&3 e 5\&4
al variare del numero di lastre di metallo.
Abbiamo confrontato i rapporti per non dover tener conto dell'accettanza%
\footnote{Sia l'efficienza che la geometria.}.

\begin{table}[h]
\centering
\begin{tabular}{| c | r @{$\pm$} l | r @{$\pm$} l | r @{$\pm$} l |}
\hline
tempo totale [\si{s}] & \multicolumn{2}{c|}{R($C2$) [1/\si{s}]} & \multicolumn{2}{c|}{R($C2\cdot \delta$) [1/\si{s}]} & \multicolumn{2}{c|}{R($C3$) [1/\si{s}]} \\
\hline
1030 & 19.6&0.1 & 12.92&0.09 & 13.1&0.1 \\
1030 & 19.8&0.1 & 13.00&0.09 & 12.9&0.1 \\
1030 & 19.8&0.1 & 13.00&0.09 & 13.1&0.1 \\
1000 & 20.1&0.1 & 13.19&0.09 & 12.7&0.1 \\
\hline
\end{tabular}
\caption{MODIFICARE:
colonna 1: lastre usate;
colonna 2: conteggio c2;
colonna 3: conteggio c3;
colonna 4: c3/c2 con l'errore giusto}
\label{dati cfr}
\end{table}


\begin{figure}[h]
\centering
\includegraphics[width=8 cm]{confronto}
\caption{Rapporto tra coincidenze a 3 ($C3$) e coincidenze a 2 ($C2$) in funzione delle lastre di materiale inserite.}
\label{cfr}
\end{figure}

\subsubsection{Energia}

Infine abbiamo eseguito due acquisizioni di lunga durata.
Nella prima abbiamo messo tutte le lastre a nostra disposizione sul PM1
e abbiamo acquisito i loro rilasci di energia per tutta la notte.
Il giorno seguente le abbiamo tolte ed abbiamo preso dati per \SI{4}{ore}.
In entrambi i casi il trigger dell'ADC è dato dalle coincidenze PM2 \& PM1.
\marginpar{Aggiungere test di Kolmogorov-Smirnov.}

\begin{figure}[h]
\centering
\includegraphics[width=8 cm]{gemelli}
\caption{MODIFICARE:
subplot 211: istogrammi;
subplot 212: differenza degli istogrammi.}
\label{gemini}    
\end{figure}
