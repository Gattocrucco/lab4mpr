
\subsection{Effetto del materiale sulla misura}

Abbiamo testato quanto la presenza di lastre di piombo potesse influire sulla rivelazione dei raggi cosmici. Abbiamo a disposizione il seguente materiale:
\begin{itemize}

\item 3 lastre rigide di piombo\\
spessore=\SI{4\pm1}{mm}\\
L1=\SI{39.9\pm0.1}{ cm}\\
L2=\SI{40.0\pm0.1}{cm}

\item 1 lastra di alluminio con le stesse dimensioni delle 3 precedenti

\item 10 lastre flessibili di piombo\\
spessore=\SI{2\pm1}{ mm}\\
L1=\SI{47.5\pm0.1}{ cm}\\
L2=\SI{45.0\pm0.1}{ cm}

\end{itemize}

\subsubsection{Conteggi}

Abbiamo posizionato le lastre rigide tra il PM3 ed il PM4 per vedere se esse riuscivano a fermare parte dei muoni. Per fare questa misura abbiamo confrontato le coincidenze  PM5 \& PM4 e PM5 \& PM4 \& PM3, però non possiamo confrontarle così come sono a causa dell'accettanza geometrica: anche in assenza delle lastre le coincidenze a tre sono diverse da quelle a due. Allora correggiamo le coincidenze a due con un Monte Carlo che tiene in considerazione le efficienze dei tre rivelatori: $\epsilon(\text{PM5})=92.6\pm0.3\%$, $\epsilon(\text{PM4})=97.2\pm0.1\%$, $\epsilon(\text{PM3})=90.1\pm0.2\%$. Il fattore di correzione ottenuto è il rapporto tra le accettanze delle due configurazioni  $\delta=mc(\text{PM5 \& PM4 \&PM3})/mc(\text{PM5 \& PM4})=65.71\pm0.02\%$, dove $mc$ indica il risultato del Monte Carlo per la configurazione in questione. Siano $C2$ il numero di coincidenze a due e $C3$ il numero di quelle a tre, per capire quanto sia significativo l'effetto delle lastre sulle nostre misure confrontiamo le quantità $C2'=C2 \cdot \delta$ e $C3$ in funzione del numero di lastre inserite%
\footnote{Nel grafico di \autoref{cfr} non è presente il caso di 3 lastre perché, avendone una di alluminio, abbiamo deciso di metterla insieme alla terza lastra di piombo.}. %
Dal confronto, presente in \autoref{cfr}, non si evince nessuna differenza tra i conteggi a meno di una separazione di $3.2\sigma$ nell'ultimo caso che, come descritto nella \autoref{sez}, rappresenta solo una fluttuazione.

\begin{table}[h]
\centering
\begin{tabular}{| c | r @{$\pm$} l | r @{$\pm$} l | r @{$\pm$} l |}
\hline
tempo totale [\si{s}] & \multicolumn{2}{c|}{R($C2$) [1/\si{s}]} & \multicolumn{2}{c|}{R($C2\cdot \delta$) [1/\si{s}]} & \multicolumn{2}{c|}{R($C3$) [1/\si{s}]} \\
\hline
1030 & 19.6&0.1 & 12.92&0.09 & 13.1&0.1 \\
1030 & 19.8&0.1 & 13.00&0.09 & 12.9&0.1 \\
1030 & 19.8&0.1 & 13.00&0.09 & 13.1&0.1 \\
1000 & 20.1&0.1 & 13.19&0.09 & 12.7&0.1 \\
\hline
\end{tabular}
\caption{Dati presenti nel grafico di \autoref{cfr}. R indica il rapporto tra il numero di eventi di più acquisizioni ed il tempo totale.}
\label{dati cfr}
\end{table}


\begin{figure}[h]
\centering
\includegraphics[width=8 cm]{confronto}
\caption{Confronto tra le coincidenze a due corrette e quelle a tre in funzione del numero di lastre inserite sul PM3.}
\label{cfr}
\end{figure}

\subsubsection{Energia}

\begin{huge}
QUI CI SAR\`A DA SCRIVERE IL VERGOGNOSO FALLIMENTO DELL'ADC
\end{huge}
\label{sez}
Infine abbiamo eseguito due acquisizioni di lunga durata. Nella prima abbiamo messo tutte le lastre a nostra disposizione sul PM1 e abbiamo acquisito i loro rilasci di energia per tutta la notte. Il giorno seguente le abbiamo tolte ed abbiamo preso dati per \SI{4}{ore}. In entrambi i casi il trigger dell'ADC è dato dalle coincidenze PM2 \& PM1.  Abbiamo normalizzato (dividendo per l'area)
\marginpar{dire che Python divide per l'area o si capisce?\\
\emph{Bisogna scriverlo anche nella caption, anche la label dell'asse $y$ è da cambiare.}}
entrambi gli spettri per poterli confrontare ed evincere che non mostrano nessuna differenza, come si può vedere in \autoref{gemini}. Questo fatto ci autorizza a pensare che la deviazione di $3.2\sigma$ vista precedentemente sia solo una fluttuazione, data la grande differenza tra i tempi di acquisizione (10$^3$\! s e tutta la notte). 
 \marginpar{non ci sono le barre d'errore sulle barre dell'istogramma\\
\emph{Non si vedrebbero credo. In questo caso un confronto quantitativo andrebbe fatto con una qualche statistica.}}

\begin{figure}[h]
\centering
\includegraphics[width=8 cm]{gemelli}
\caption{Confronto tra gli spettri normalizzati (dividendo per l'area) con e senza piombo. Il picchetto a destra è dato dagli eventi talmente energetici da saturare l'uscita dell'amplificatore.}
\label{gemini}    
\end{figure}

