
\subsection{Configurazione utilizzo ADC}  % troppo giornalistico?  (mi riferisco alla mancanza di articoli)

La necessità di misurare quantitativamente il rilascio di energia nel rivelatore ci obbliga ad aggiungere nel nostro crate 3 moduli che permettono di digitalizzare l'uscita del PMT: il preamplificatore, l'amplificatore e l'ADC. Il preamplificatore ha il compito di allungare la durata dell'output del PMT dai nanosecondi ai microsecondi, in modo che l'amplificatore e l'ADC abbiano il tempo di leggerla. Esso è collegato direttamente al PMT che vogliamo usare per evitare riflessioni e non è possibile operare un adattamento di impedenza, in quanto il preamplificatore necessita di un'elevata impedenza d'ingresso per allungare la durata del segnale. 
Lo abbiamo impostato in modo che esso restituisca un segnale con una salita praticamente istantanea ed una discesa esponenziale \marginpar{se credete che stia sbagliando, ricordatevi le convenzioni di nomenclatura} con un $\tau\approx\SI{15}{\mu s}$.

Il compito dell'amplificatore è intensificare il segnale al suo ingresso (in questo caso quello del preamplificatore) e restituire un'uscita analogica proporzionale all'energia del segnale in ingresso%
\footnote{Il manuale dice che la forma dell'uscita è data dalla funzione $e^{-3t}\sin^4{t}$, in cui $t$ è il tempo. Aggiunge poi che questa forma ``simile ad una Gaussiana'' è ``proporzionale all'energia''.}.
Ha un'uscita massima di \SI{11}{V}.

Lo scopo dell'ADC è trasformare un segnale analogico in uno digitale (solitamente un numero intero) in modo che questo possa essere salvato su un computer.  
\marginpar{so che non è corretto dire segnale digitale=numero intero ma non so in che altro modo dirlo!}
La conversione del segnale parte dopo \SI{1}{\mu s} dall'arrivo di un segnale di \emph{trigger} e dura all'incirca tale intervallo di tempo. L'ADC può leggere segnali analogici che vanno da qualche \si{mV} a \SI{2.5}{V} e può digitalizzarne al massimo 40 al secondo, il trigger legge un segnale TTL di ampiezza massima \SI{3.3}{V}. 

Per poter sfruttare l'apparato di digitalizzazione dobbiamo trasformare l'uscita del PMT adoperando il preamplificatore, smorzare il segnale dell'amplificatore in modo che non possa danneggiare l'ADC e costruire un trigger che faccia partire l'acquisizione \SI{1}{\mu s} prima che il segnale amplificato raggiunga il massimo.
Agiamo come segue: \marginpar{inserire uno schema. lo schema che ho descritto non si trova disegnato nel logbook. è più generale}
colleghiamo un certo PMT al preamplificatore e la sua uscita all'amplificatore. Prendiamo il segnale del preamplificatore%
\footnote{Un'uscita dell'amplificatore ci permette di farlo.}
che vogliamo usare nelle coincidenze e nella fabbricazione di un trigger. Portiamo questo segnale ad un discriminatore a cui sono collegati anche gli altri PMT. Le uscite di quest'ultimo sono mandate ad un modulo \emph{gate \& delay}
\marginpar{cercare di italianizzarlo}
non retriggerabile%
\footnote{Abbiamo riscontrato numerose ripartenze prima di utilizzare questo modulo.}
collegato poi ad un secondo discriminatore che abbiamo usato come fan-out (l'altro fan-out a disposizione non ci bastava) per mandare i segnali a dei moduli di coincidenze, al contatore, ecc. In caso di coincidenze, un segnale tra quelli che ci interessano viene ritardato in modo da poter raggiungere l'ingresso del trigger  \SI{1}{\mu s} prima che il segnale dell'amplificatore raggiunga il suo massimo. Questa operazione viene effettuata dapprima ritardando il segnale con un \emph{timer} per poi farlo passare attraverso un convertitore NIM-TTL in modo che sia leggibile dal modulo dell'ADC. Nel frattempo il segnale amplificato viene smorzato da un attenuatore da \SI{14}{dB} che riduce l'ampiezza dei segnali maggiori da \SI{11}{V} a \SI{2.5}{V} in modo che sia impossibile danneggiare l'ADC. Il fatto che il convertitore TTL-NIM abbia un'uscita massima di \SI{3}{V} evita che il trigger possa causare danni.
