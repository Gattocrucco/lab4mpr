\subsection{Coincidenze casuali e tempo morto}

Per tenere conto con simulazioni o calcoli delle coincidenze casuali e dei tempi morti,
trattiamo i segnali digitali come forme perfettamente rettangolari.
Assegniamo la distribuzione esponenziale al tempo tra un \emph{evento} e il successivo.
Un evento genera un fronte di salita se il tempo trascorso dall'ultimo fronte di salita
è minore del tempo morto.
Il segnale rimane alto per una certa \emph{durata del segnale} fissata.
Il tempo morto è almeno la durata del segnale.
I segnali fanno coincidenza quando sono entrambi alti per almeno un certo \emph{tempo di coincidenza}.
