\subsection{Calcolo delle coincidenze casuali}

Per tenere conto con simulazioni o calcoli delle coincidenze casuali e dei tempi morti,
trattiamo i segnali digitali come forme perfettamente rettangolari.
Assegniamo la distribuzione esponenziale al tempo tra un \emph{evento} e il successivo.
Un evento genera un fronte di salita%
\footnote{Noi usiamo moduli NIM che hanno logica negativa, quindi un fronte di salita corrisponde in tensione a un fronte di discesa.}
se il tempo trascorso dall'ultimo fronte di salita
è minore del tempo morto.
Il segnale rimane alto per una certa \emph{durata del segnale} fissata.
Il tempo morto è almeno la durata del segnale.
I segnali fanno coincidenza quando sono entrambi alti per almeno un certo \emph{tempo di coincidenza}.

\paragraph{Simulazione Monte Carlo}

Prendiamo un insieme di segnali, indicizzati da $j$,
con rate $r_j$, durate $c_j$ e tempi morti $d_j \ge c_j$.
Per ogni segnale siano $t_{j,i}$ i tempi dei vari eventi.

L'idea di base per simulare le coincidenze è semplice.
Sia $t_{j,i}$ l'ultimo evento che ha generato un fronte di salita sul segnale $j$,
estraiamo $t_{j,i+1}$ dalla distribuzione esponenziale di scala $r_j^{-1}$;
continuiamo a estrarre nuovi eventi finché non superano il tempo morto cioè $t_{j,i+k}>t_{j,i} + d_j$,
allora $t_{j,i+k}$ genera il nuovo fronte di salita.
Generiamo un nuovo fronte sul segnale 0.
Generiamo nuovi fronti sul segnale~1 finché non si verifica che:
o il segnale 1 fa coincidenza con il segnale 0,
o l'ultimo fronte~1 supera l'ultimo fronte 0 abbastanza da non poter fare coincidenze.
Se c'è coincidenza si prosegue generando fronti sul segnale 2;
se alla fine tutti i segnali fanno coincidenza si incrementa il conteggio delle coincidenze.
Si prosegue generando un nuovo fronte sul segnale 0,
finché $t_{0,i}$ non supera il tempo totale.

Questo algoritmo ha il bug che un fronte sul segnale 0 non può generare più di una coincidenza.
La soluzione è la seguente:
si introduce un \emph{segnale leader} $J$ che inizialmente è $J=0$
e che ha il ruolo del segnale 0 nella precedente descrizione.
Quando si verifica una coincidenza,
il nuovo leader è il segnale che ha il fronte di discesa $t_{j,i}+c_j$ maggiore.
Viene aggiunto un nuovo evento al leader solo quando non si verifica una coincidenza.
Il motivo è che qualunque nuova coincidenza
che avvenga senza aggiungere un nuovo evento su un certo segnale,
deve necessariamente avvenire senza aggiungere un evento sul segnale che termina più tardi.
