\documentclass[a4paper]{article}

\usepackage[utf8]{inputenc}
\usepackage[T1]{fontenc}
\usepackage[italian]{babel}

\usepackage{siunitx}
\usepackage{amsmath}
\usepackage{amssymb}
\usepackage[hidelinks]{hyperref}
\usepackage{graphicx}
\usepackage[font={sf}]{caption}

\setlength{\marginparwidth}{95pt}

\sisetup{%
separate-uncertainty=true,
multi-part-units=single,
exponent-product=\cdot}

\frenchspacing

\begin{document}
  
\subsection*{Elettroni vs muoni}  % titolo provvisorio

I raggi cosmici carichi presenti al livello del suolo\footnote{Informazioni prese dalla scheda dell'esperienza} sono costituiti principalmente da muoni e da elettroni (positroni). Alcune di queste particelle potrebbero non essere rivelate dal nostro telescopio di raggi cosmici a causa dell'energia persa nell'attraversare il tetto dell'edificio. Calcoliamo allora l'entità di questo effetto.
\subsubsection*{Muoni} 
Per i calcoli sui muoni ci avvaliamo del grafico%
\footnote{\emph{Particle Physics Booklet 2014}, pag. 255}
che mostra il rapporto tra percorso residuo e massa in funzione dell'impulso del muone.
I muoni a livello del suolo hanno un'energia media $E_\mu=\SI4{GeV}$. Essendo $m_\mu \ll E_\mu$, $p_\mu \simeq E_\mu$. Il valore di $R/M$ equivalente a questo impulso nel piombo è circa \SI{3e4}{g cm^{-2} GeV^{-1}}. Per trovare il percorso residuo bisogna moltiplicare questa quantità per la massa della particella e dividere per la densità del materiale.
\footnote{Ho inserito tutti questi passaggi in modo che almeno voi due non dobbiate scervellarvi per capire che cosa ho fatto}
\begin{equation}
R(\si{cm})=\frac{R}{M} (\si{g cm^{-2} GeV^{-1}}) \cdot \frac{M(\si{GeV})}{\rho_{\text{Pb}} (\si{g/cm^3})}=3\cdot10^4 \cdot  \frac{10^{-1} }{11.35}=\SI{300}{cm}    
\end{equation}
Quindi per fermare un muone da \SI{4}{GeV} sono necessari \SI{3}{m} di piombo, ma il tetto dell'edificio ha uno spessore di circa \SI{1}{m}
\footnote{spero di non sbagliarmi. Comunque se fosse spesso 2 m non dovrebbe cambiare la situazione.} 
 ed è costituito da un materiale meno denso del piombo. I muoni penetrati all'interno del laboratorio hanno quindi perso meno di un terzo della loro energia iniziale e sono ancora ultrarelativistici: le loro perdite di energia in aria e nello scintillatore sono trascurabili. \footnote{non ho scritto la formula del minimo di ionizzazione e relativi calcoli perché questi dovrebbero trovarsi più su a relazione finita.}

\subsubsection*{Elettroni} 
Quando un elettrone di energia $E_0$ attraversa uno strato di materiale di lunghezza $L$, l'energia persa può essere scritta come 
\begin{equation}
\Delta E=E_0(1-e^{-L/X_0})  \label{Eres}
\end{equation}
dove $X_0$ è la lunghezza di attenuazione del materiale.
Facciamo una stima supponendo la densità del tetto pari a quella delle tegole ($X_0=\SI{8.53}{cm}$)
ed $L=\SI{1}{m}$. Otteniamo $(E_0-\Delta E)/E_0 \approx e^{-12} \simeq 8\cdot10^{-6}$. \footnote{basta fare $L/X_0$} 
\footnote{Nessun elettrone riesce a superare un metro di piombo, ma di cosa è fatto sto cazzo di tetto? Ho calcolato la $X_0$ dei mattoni. Ma 1 m è troppo. In realtà il tetto è fatto da materiali diversi ed ha una parte vuota in cui scorrazzano le faine.}



  
\end{document}