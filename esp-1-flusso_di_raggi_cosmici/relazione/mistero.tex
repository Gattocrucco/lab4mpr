\subsection{Rumore anomalo}
\label{rumore}

Quando abbiamo analizzato il file signal\_2000.dat ci siamo accorti del picco descritto nella \autoref{cenno}. Descriviamo le operazioni eseguite per capirne l'origine o cercare di delinearne la dipendenza dalle grandezze in gioco nel caso in cui non riuscissimo a scoprirne la causa.

Iniziamo accorgendoci che le coincidenze a tre sono molto diverse da quelle a 2 e cambiamo le soglie per vedere se hanno una qualche influenza sui dati, ma non riscontriamo nulla. Pensiamo che la causa del fenomeno siano i retrigger, ma non notiamo nessun miglioramento dopo aver inserito un circuito di antiretrigger. Allora attacchiamo il preamplificatore al PMT4 per vedere se il problema era del PMT5: non notiamo differenze, però sospettiamo che a volte il trigger dell'amplificatore scatti quando quello che abbiamo costruito per l'ADC non lo fa.
\marginpar{aggiungere foto?}
Pensiamo che questi rumori facciano scattare le coincidenze a 2 ma non quelle a 3. Allora facciamo 4\&2\&$\neq$3 per vedere quali segnali fanno scattare una coincidenza a 2 ma non a 3. Scopriamo che sono segnali che non superano i \SI{200}{mV}, ovvero sono in quella zona che a \SI{2000}V è dominata dal rumore. Questo ci fa pensare che siano coincidenze casuali nonostante siano molto più frequenti di quanto calcolato con la formula al primo ordine $R_1 R_2 \Delta t$. 
Modifichiamo il circuito di misura in modo da contare la differenza tra coincidenze a 2 e coincidenze a 3.
A questo punto ripetiamo le misure precedenti ma senza ADC e non notiamo differenze.
Poi ci accorgiamo che abbiamo fatto alcune misure tenendo collegato l'oscilloscopio, quindi le terminazioni inadatte hanno rovinato la lettura di alcuni segnali. Dopo averlo scollegato la situazione non è cambiata per niente. A questo punto ci viene in mente che un rumore correlato può venire soltanto dall'alimentatore ad alta tensione o direttamente dalla rete elettrica. Quindi facciamo le stesse misure di prima ma con un  \texttt{alimentatore HV NIM} e poi con entrambi gli alimentatori a nostra disposizione, ma senza risultati. Come ``ultima spiaggia'' ritardiamo molto il segnale di un PMT in modo che non possa mai essere in coincidenza a 2 con un altro a nostra scelta. I conteggi che otteniamo da questa ``coincidenza'' sono compatibili con le coincidenze casuali attese. 
Infine copriamo l'apparato con un telo nero: i conteggi di alcuni PMT si dimezzano, ma la differenza tra coincidenze a 2 e a 3 rimane la stessa.

Concludiamo che il rumore anomalo dipende solo dai conteggi dei singoli PMT.
La \autoref{orrore} riporta i risultati di una serie di misure semplici da comprende con una breve occhiata e riassume in modo quantitativo la complicata situazione sopra descritta. 

\begin{table}
\centering
\begin{tabular}{c|c|c|c|c|c|c}
\hline
alimentazione [V] & PM5 & PM3 & PM2 & 5\&2 & 5rit\&2 & 5\&3\&2 \\
\hline
2000 & 86781 & 3289968 & 137650 & 2092 & 50 & 1457 \\
1950 & 49883 & 3154704 & 105920 & 1824 & 25 & 1467 \\
1900 & 17214 & 3277759 & 46638 & 1426 & 1 & 1306 \\
1850 & 10583 & 3064378 & 15191 & 1325 & 0 & 1272 \\
1800 & 6069 & 3161410 & 8943 & 1087 & 0 & 1068 \\
\hline
\end{tabular}
\caption{Misure di rumore per varie tensioni di alimentazione del PMT5 e PMT2. L'alimentazione del PMT3 è 2000\,V. Ogni misura è stata eseguita in 100\,s e l'errore su tutti i conteggi è la loro radice quadrata.}
\label{orrore}
\end{table}