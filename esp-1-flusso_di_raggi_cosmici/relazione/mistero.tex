\subsection{Rumore anomalo}
\label{rumore}

Come accennato in \autoref{cenno} analizzando gli spettri energetici di segnale e rumore ci siamo accorti che ad alte tensioni (la cosa è evidente a $\SI{2000}{V}$ in \autoref{quattro}) lo spettro del segnale mostra un eccesso nella zona a basse energie, quella popolata dal rumore, forte indizio che si tratti in realtà di falsi positivi. I conteggi in questa zona sono però più del 10\% del totale, quasi 2 ordini di grandezza più di quanto atteso dalla previsione teorica al prim'ordine delle coincidenze casuali: $R_1 R_2 \Delta t$, dove $R_1$ e $R_2$ sono i rate e $\Delta t$ è la somma delle durate dei segnali.

Inizialmente si è pensato ad un problema del trigger dell'ADC ma l'ipotesi è stata scartata perchè il problema si presenta anche eliminando l'ADC dal circuito. Le misure riportate in \autoref{orrore} sono ben rappresentative del problema: si è fatta una normale misura di efficienza sul PMT3, usando i PMT2 e 5\footnote{Si sono cioè misurate le coincidenze 2\&5 e 2\&3\&5.}. La tensione di alimentazione del PMT3 era massima ($\SI{2000}{V}$) e la misura è stata ripetuta al variare dell'alimentazione dei PMT2 e 5, spaziando da $\SI{1800}{V}$ a $\SI{2000}{V}$. Per valutare sperimentalmente il rumore dovuto alle coincidenze casuali si sono misurate le coincidenze tra PMT2 e il segnale del PMT5 ritardato abbastanza da poter eliminare qualsiasi correlazione temporale residua (decine di $\si{\micro s}$).
Come è evidente dalle misure il rapporto tra coincidenze a 3 e coincidenze a 2 che dovrebbe essere l'efficienza (a meno della correzione per le coincidenze casuali) è $\sim 100\%$ per basse tensioni dei PMT esterni (e quindi bassi rate) e scende fino al $\sim 70\%$ a $\SI{2000}{V}$ (ovvero rate elevati) e questa differenza non è in nessun modo giustificata dalle coincidenze casuali previste dalla teoria che ben si accordano con quelle misurate con le coincidenze ritardate. Sembra quindi che ci siano coincidenze di impulsi che non sono segnale ma che sono correlate temporalmente.

Abbiamo verificato sperimentalmente che questo tipo di rumore anomalo viene ridotto (ma non eliminato) aumentando il numero di rivelatori in coincidenza e che sembra dipende solo dal rate e non dal particolare valore di alimentazione e soglia.

Abbiamo escluso che la luce ambientale possa avere un ruolo in questo rumore, coprendo completamente l'apparato e spegnendo la luce, senza riscontrare cambiamenti.

Sospettiamo si possa trattare di rumore sulla linea elettrica, non soppresso dall'alimentatore, che correla temporalmente tra di loro segnali di rumore su canali diversi.

\begin{table}
\centering
\begin{tabular}{c|c|c|c|c|c|c}
\hline
alim. PM2 e 5[V] & PM5 & PM3 & PM2 & 5\&2 & 5rit\&2 & 5\&3\&2 \\
\hline
2000 & 86781 & 3289968 & 137650 & 2092 & 50 & 1457 \\
1950 & 49883 & 3154704 & 105920 & 1824 & 25 & 1467 \\
1900 & 17214 & 3277759 & 46638 & 1426 & 1 & 1306 \\
1850 & 10583 & 3064378 & 15191 & 1325 & 0 & 1272 \\
1800 & 6069 & 3161410 & 8943 & 1087 & 0 & 1068 \\
\hline
\end{tabular}
\caption{Misure di "efficienza" al variare della tensioni di alimentazione dei PMT esterni con misura delle coincidenze casuali tra questi ultimi attraverso coincidenze ritardate. L'alimentazione del PMT3 è 2000\,V. Tempo di acquisizione 100\,s..}
\label{orrore}
\end{table}