\subsection{Rumore anomalo}
\label{rumore}

Quando abbiamo analizzato il file signal\_2000.dat ci siamo accorti del picco descritto nella \autoref{cenno}. Descriviamo le operazioni eseguite per capirne l'origine o cercare di delinearne la dipendenza dalle grandezze in gioco nel caso in cui non riuscissimo a scoprirne la causa.


\begin{table}
\centering
\begin{tabular}{c|c|c|c|c|c|c}
\hline
alimentazione [V] & PM5 & PM3 & PM2 & 5\&2 & 5rit\&2 & 5\&3\&2 \\
\hline
2000 & 86781 & 3289968 & 137650 & 2092 & 50 & 1457 \\
1950 & 49883 & 3154704 & 105920 & 1824 & 25 & 1467 \\
1900 & 17214 & 3277759 & 46638 & 1426 & 1 & 1306 \\
1850 & 10583 & 3064378 & 15191 & 1325 & 0 & 1272 \\
1800 & 6069 & 3161410 & 8943 & 1087 & 0 & 1068 \\
\hline
\end{tabular}
\caption{Misure di rumore per varie tensioni di alimentazione del PMT5 e PMT2. L'alimentazione del PMT3 è 2000\,V. Ogni misura è stata eseguita in 100\,s e l'errore su tutti i conteggi è la loro radice quadrata.}
\label{orrore}
\end{table}