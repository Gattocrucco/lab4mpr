\subsubsection{Decadimento di muoni nei rivelatori}

Si potrebbe discutere della possibilità che un muone che è stato rivelato da uno scintillatore decada prima di raggiungere il successivo.  I muoni hanno una vita media $\tau=\SI{2.2}{\micro s}$ e percorrono in tale intervallo di tempo una distanza media $\lambda=\gamma\beta c\tau$. Siccome la probabilità di decadere non dipende dall'istante di tempo in cui si inizia una misura, il nostro modo di procedere non viene influenzato dal decadimento dei muoni se la distanza tra due scintillatori in coincidenza è molto minore di $\lambda$.
Sapendo che i muoni cosmici al livello del mare hanno un'energia di almeno circa \SI{1}{GeV}, possiamo porre $\gamma=E/m\approx10$ e $\beta=1$. Ne consegue $\lambda\approx\SI{7}{km}$, una distanza molto maggiore degli \SI{80}{cm} che separano gli scintillatori più lontani.

\subsubsection{Muoni dal sottosuolo}
Si potrebbe indagare la presenza di raggi cosmici provenienti dal sottosuolo: il modo più semplice mettere in coincidenza il segnale ritardato del PMT più basso con il PMT più alto. Nel nostro caso le lastre più lontane si trovano a $\sim\SI{80}{cm}$ di distanza ed una particella ultrarelativistica le attraversa in \SI{2.7}{ns}: un tempo troppo breve per i discriminatori a disposizione con una durata minima di \SI{10}{ns}, l'attraversamento delle due lastre è di fatto simultaneo in qualsiasi direzione.