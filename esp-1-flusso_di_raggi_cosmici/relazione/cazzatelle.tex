\subsection{Riflessioni} 
\marginpar{bozza da completare a fine relazione o inglobare i pezzi in punti diversi della relazione stessa}

\subsubsection{Muoni che decadono}

Il nostro esperimento consiste nel rivelare muoni che attraversano le nostre lastre facendo uso di coincidenze. Questo metodo è inutile se una particella che ha rilasciato fotoni nel primo scintillatore decade prima di raggiungere il successivo.  I muoni hanno una vita media $\tau=\SI{2.2}{\micro s}$ e percorrono in tale intervallo di tempo una distanza media $\lambda=\gamma\beta c\tau$. Siccome la probabilità di decadere non dipende dall'istante di tempo in cui si inizia una misura, il nostro modo di procedere non viene influenzato dal decadimento dei muoni se la distanza tra due scintillatori in coincidenza è molto minore di $\lambda$.
Sapendo che i muoni cosmici al livello del mare hanno un'energia di almeno circa \SI{1}{GeV}, possiamo porre $\gamma=E/m\approx10$ e $\beta=1$. Ne consegue $\lambda\approx\SI{7}{km}$, una distanza molto maggiore degli \SI{80}{cm} che separano gli scintillatori più lontani.

\subsubsection{Muoni da sotto}

Nella descrizione dell'esperienza abbiamo sempre sottinteso che i muoni possono arrivare sul nostro telescopio soltanto dall'alto, ma sappiamo che una parte dei muoni presenti sulla superficie terrestre proviene dal sottosuolo.
Il modo più semplice per capire se una qualsiasi particella arriva dal sottosuolo consiste nel ritardare il segnale del PMT più basso per metterlo in tempo con quello del PMT più alto. Nel nostro caso le lastre più lontane si trovano a \SI{80}{cm} di distanza ed una particella ultrarelativistica le attraversa in \SI{2.7}{ns}: un tempo troppo breve se si hanno dei discriminatori la cui uscita ha una durata minima di \SI{10}{ns}. Da questo punto di vista l'attraversamento delle due lastre è simultaneo in qualsiasi direzione.

