\documentclass[a4paper]{article}

\usepackage[utf8]{inputenc}
\usepackage[T1]{fontenc}
\usepackage[italian]{babel}

\usepackage{siunitx}
\usepackage{amsmath}
\usepackage{amssymb}
\usepackage[hidelinks]{hyperref}
\usepackage{graphicx}
\usepackage[font={sf}]{caption}

\setlength{\marginparwidth}{95pt}

\sisetup{%
separate-uncertainty=true,
multi-part-units=single}

\frenchspacing

\newcommand*\de{\mathrm{d}}
\newcommand*\pdv[2]{\frac{\partial #1}{\partial #2}}

\title{Relazione laboratorio particelle\\Esperienza 0: efficienza di un contatore}
\author{Andrea Marasciulli\\
Giacomo Petrillo\\
Roberto Ribatti}
\date{Dal 13 al 19 novembre 2017\footnote{Compilazione \LaTeX{} di questo documento: \today.}}

\begin{document}
	
\maketitle

Scegliamo di usare, in ordine dal basso verso l'alto, i PMT 2, 3, 4.
Accendiamo il PMT~2 a \SI{1500}V.
Schizzo dei segnali del PMT visti sull'oscilloscopio:
\begin{center}
	\includegraphics[width=6cm]{fig3a}
\end{center}
Con il trigger a \SI{-27}{mV}, contando a mano su \SI1{min}, otteniamo $\sim\SI1{Hz}$.

\subsection*{Amplificazione}

Stimiamo l'energia rilasciata da una particella nello scintillatore.
Supponiamo MIP, quindi
$\de E/\rho\de\ell=\SI{1.5}{MeV\,g^{-1}\,cm^2}$.
Lo spessore dello scintillatore è $\sim\SI1\cm$ e la densità $\sim\SI1{g\,cm^{-3}}$,
quindi rilascia $\SI{1.5}{MeV} = \SI{2.4e-13}{J}$.

Stimiamo l'energia del segnale in uscita. È
$\int\de t\, I\cdot V \approx \Delta t\cdot V^2/R$
dove $R=\SI{50}\ohm$, quindi otteniamo
$\num{10e-9}\cdot(\num{100e-3})^2/50=\SI{2e-12}{J}$.

L'amplificazione complessiva è quindi $\sim10$.

\subsection*{Dipendenza rate dall'alimentazione}

Aumentiamo la tensione di alimentazione a \SI{1600}V.
La frequenza di segnali aumenta notevolmente,
per contarli a mano alziamo\footnote{Trattando la logica negativa, useremo alto-basso riferito al modulo e spesso tralasceremo il segno $-$,  la strumentazione suggerisce questo approccio perché la vite della soglia del discriminatore aumenta (abbassa) la soglia in verso orario.} il trigger a \SI{150}{mV},
otteniamo $85/\SI1{min}$.

Con \SI{1800}V otteniamo $\sim\SI{100}{Hz}$ e con \SI{2000}V $\sim\SI{1}{kHz}$
(frequenze misurate dall'oscilloscopio).

\subsection*{Documentazione discriminatore}

Non troviamo la documentazione del nostro discriminatore tra quella fornita;
poiché è a 4 canali e l'unica documentazione per uno a quattro canali è il CAEN~N84,
e inoltre a parte colore e etichette appare identico,
leggiamo la documentazione per quello e chiameremo il nostro <<N84>>.

Documentazione:
ritardo \SI{14}{ns},
durata 6--400\,\si{ns},
soglia massima \SI{400}{mV}.

\subsection*{Verifica discriminatore}

Montiamo questo circuito:
\begin{center}
	\includegraphics[width=9cm]{fig4a}
\end{center}
Girando la vite della durata vediamo che l'intervallo impostabile è circa 40--300\si{\,ns}.
È di tipo ``restarting'' perché per soglia abbastanza bassa si vedono durate doppie in uscita,
con soglia a opportuni valori intermedi si vedono alternativamente durate doppie oppure due singole, ravvicinate ma non più di tanto.
L'ampiezza di uscita è \SI{750}{mV} come atteso dallo standard NIM.
Verifichiamo la durata impostabile anche per il canale~2, è la stessa.

Modifichiamo il circuito in modo da visualizzare sia l'ingresso che l'uscita del discriminatore:
\begin{center}
	\includegraphics[width=5cm]{fig4b}
\end{center}
Misuriamo la soglia del discriminatore con l'oscilloscopio in questo modo:
triggeriamo sull'ingresso del discriminatore,
se la soglia del discriminatore è minore di quella del trigger
si vede sempre l'uscita del discriminatore,
viceversa a volte scompare.
Trovando due soglie del trigger più vicini possibili in modo che con una a volte scompare l'uscita (verificato su un tempo abbastanza lungo rispetto alla differenza delle soglie) e con l'altra no, si ottiene una misura della soglia del discriminatore.
Il problema di questa procedura è che il trigger dell'oscilloscopio funziona in modo diverso dal discriminatore, quindi non è detto che a parità di soglia vengano selezionati gli stessi segnali.
Con la soglia del discriminatore al massimo otteniamo le soglie del trigger 368 e \SI{376}{mV},
mentre il TP del discriminatore segna \SI{0.411}V, quindi lo scarto è circa \SI8\percent.

Il ritardo è dato da quello dei cavi (\SI3{ns}) più quello del discriminatore.
Puntando i cursori dell'oscilloscopio otteniamo un ritardo di \SI{20}{ns}
quindi il ritardo del discriminatore è circa \SI{17}{ns}.

Ci è difficile misurare il jitter del discriminatore perché il trigger dell'oscilloscopio funziona in modo diverso dal discriminatore, perciò facciamo una stima superiore in questo modo:
mettiamo la soglia del trigger ai \SI{376}{mV} trovati prima che facevano circa coincidere i segnali selezionati, triggerando il segnale del PMT, e guardiamo a occhio sull'oscilloscopio la variazione temporale del fronte finale dell'uscita del discriminatore; otteniamo \SI{\pm1}{ns}.

\subsection*{Documentazione contatore}

Non troviamo la documentazione, leggiamo quella del 130~PCZ che ci somiglia.
Frequenza massima \SI{70}{MHz} (sul nostro c'è scritto \SI{100}{MHz}),
conta fino a $10^8-1$,
l'input deve durare almeno \SI7{ns},
l'output ``clock'' è un'onda quadra (duty \SI{50}\%) a \SI1{kHz}.

\subsection*{Clock del contatore}

Si può collegare il clock all'ingresso~8 per misurare il tempo.
Si può impostare in modo che dopo $10^n$ $(n=2,\dots,7)$
conteggi si fermi.
Se avviamo il conteggio con il pulsante <<start>> o con l'ingresso corrispondente
e lo facciamo fermare in automatico con il clock,
introduciamo un'incertezza sul tempo totale dovuta al non allineamento dello start con il clock\footnote{Potremmo darlo per ovvio, però notiamo che quasi sicuramente i comandi manuali/ingresso non vengono allineati dallo strumento con il fronte successivo del clock perché gli ingressi possono definire intervalli di tempo con molta più granularità di \SI1{ms}.}.
L'incertezza è dell'ordine del periodo del clock.

Visualizziamo il clock sull'oscilloscopio, che calcola una frequenza di \SI{0.99996}{kHz},
cioè la discrepanza è di \SI3s al giorno.

\subsection*{Primo test di conteggio}

Montiamo questo circuito:
\begin{center}
	\includegraphics[width=9cm]{fig5a}
\end{center}
Il trigger dell'oscilloscopio è impostato a \SI{376}{mV}.
Mettiamo lo stop a 1000 conteggi di clock (\SI1s) e facciamo un po' di conteggi.
Il valore dei conteggi gira intorno a 20;
la frequenza calcolata dall'oscilloscopio oscilla ampiamente perché probabilmente la calcola su intervalli di tempo piccoli rispetto al tasso di conteggi,
quando si stabilizza per qualche secondo gira intorno a \SI{30}{Hz},
quindi non troviamo incongruenze.

\subsection*{Distribuzione dei conteggi}

Ci aspettiamo che, per un tasso costante, i conteggi siano poissoniani con
$\text{media} = \text{tasso}\times \text{tempo}$.
Lo stimatore di massima verosimiglianza per la media è il conteggio ottenuto (la media aritmetica per più estrazioni) e ha bias nullo e varianza minima.
La varianza dello stimatore è la media, quindi,
con la convenzione standard di calcolare la varianza nella stima,
per $k$ conteggi il risultato per la media $\mu$ sarà scritto come
\[ \text{<<$\mu = k \pm \sqrt k$>>.} \]

\paragraph{Errore di clock}

Calcoliamo approssimativamente il rapporto tra l'incertezza su $\mu$ introdotta dall'usare il clock per misurare il tempo
e quella ``normale'' $\sqrt\mu$.
Fissiamo l'unità di misura del tempo a un periodo del clock.
L'incertezza sul tempo di misura $T$ è quindi $\Delta T\sim 1$.
Sia $R$ il tasso.
Allora $\mu=RT$, quindi
\[ \frac{\Delta\mu_\text{clock}}{\Delta\mu_\text{normale}}
= \frac{\mu\Delta T/T}{\sqrt{\mu}}
= \frac{\sqrt\mu}T
= \sqrt{\frac RT}.\]
Osserviamo che $T$ è il numero di cicli di clock usati,
e che il rapporto va a zero con la radice del tempo di misura.
Per le nostre misure vedremo che l'errore di clock è trascurabile.

\subsection*{Secondo test di conteggio}

Facciamo una serie di conteggi trascrivendo i risultati.
Facciamo 45 conteggi con clock 1000 e 15 con clock \num{10000}.
Verifichiamo che siano poissoniane con un test del $\chi^2$ senza allargare i bin rispetto ai valori discreti;
poiché il numero di conteggi è basso calcoliamo la distribuzione della statistica con monte carlo.
Riportiamo il risultato in \autoref{fig:testc}.
Vediamo che la poissonianità della serie da \SI{10}s è in dubbio.

\begin{figure}
	\centering
	\includegraphics[width=\textwidth]{fig6a} \\
	\includegraphics[width=\textwidth]{fig6b}
	\caption{\label{fig:testc}
	Test di conteggio sul PMT2.
	Il primo istogramma riporta 45 conteggi da 1000 cicli,
	il secondo 15 conteggi da 10\,000 cicli.}
\end{figure}

\subsection*{Tempo dall'accensione}

Può darsi che il PMT, dopo l'accensione, ci metta un po' ad arrivare in una condizione stazionaria di funzionamento.
I PMT 3 e 4 li abbiamo lasciati spenti, quindi li usiamo per fare un test.
Accendiamo il PMT3 e subito prendiamo una serie di 15 conteggi con clock~1000.
Poi accendiamo il PMT4 e prendiamo una serie, poi proseguiamo alternando PMT~3 e 4,
per un totale di 5 serie. Grafichiamo le medie con la varianza nell'ipotesi di poissoniane:
\begin{center}
	\includegraphics[width=8cm]{fig6c}
\end{center}
Chiaramente il tasso varia nel tempo, quindi in futuro aspetteremo sempre qualche minuto dopo che abbiamo acceso i PMT e almeno 1~minuto quando cambiamo tensione di alimentazione.

\subsection*{Luce ambientale}

Verifichiamo la sensibilità alla luce esterna passando una torcia vicino allo scintillatore e al PMT e verificando se il numero di conteggi cambia sensibilmente.
Testiamo i PMT 2, 3, 4.
\begin{center}
\begin{tabular}{c|p{50ex}}
	PMT & Risultato \\
	\hline
	4 &
	Troviamo un ``buco ottico'' sull'attacco della guida ottica al PMT,
	puntando la torcia i conteggi passano da $\approx 100$ a $\approx 200$. \\
	3 &
	Anche qui l'attacco del PMT alla guida ottica è problematico;
	la luce entra riflettendosi sulla base del PMT.
	I conteggi variano da $\approx 100$ a $\approx 280$.\\
	2 &
	Tutto ok.
\end{tabular}
\end{center}
Sistemiamo tutto con il nastro isolante e ricontrolliamo.
I conteggi sono minori di prima del rattoppo e rimangono stabili puntando la torcia.

\subsection*{Alimentazione}

Variamo l'alimentazione dei vari PMT per vedere come cambia il numero di conteggi. I risultati di queste misure sono  riportati in \autoref{tensio}. 

\begin{figure}[h]
\centering
\includegraphics[width=8 cm]{tensio_pmt}
\caption{Numero di conteggi in funzione della tensione di alimentazione.}
\label{tensio}
\end{figure}

Il grafico mostra chiaramente l'assenza di qualsiasi \emph{plateau} tranne nei punti a conteggio nullo.
La pendenza è crescente, quindi scegliamo la tensione più bassa che ci dia un tasso abbastanza elevato
per fare le misure in tempi ragionevoli.
Scegliamo \SI{1800}V.

\subsection*{Conteggio atteso}

Dal \emph{Particle Physics Booklet 2016} sappiamo che il flusso di muoni%
\footnote{Gli elettroni hanno un $1/E\, \de E/\de x \gtrsim \SI{0.15}{g^{-1}cm^{2}}$, quindi supponiamo che l'edificio schermi il nostro apparato dagli elettroni.}
 su una superficie orizzontale al livello del mare è circa \SI{130}{Hz/m^2}.
Essendo il nostro rivelatore di area $\SI{48}{cm}\cdot\SI{40}{cm}=\SI{0.19}{m^2}$,
ci aspettiamo il passaggio di \SI{25}{particelle/s}.

\subsection*{Modulo di coincidenze}

Per effettuare il conteggio delle coincidenze colleghiamo le uscite dei discriminatori a un modulo di coincidenze e la sua uscita al  contatore.
 Sono collegati al contatore anche il PMT4 e al PMT2 attraverso il discriminatore.
Le soglie di entrambi sono massime, ovvero \SI{-0.410}V misurate dal testpoint.
Con questi settaggi i nostri rivelatori misurano circa 40 eventi al secondo,
quindi facciamo un calcolo approssimativo del rate di coincidenze casuali usando la formula
$R_C=R_1R_2\Delta t$.
Siccome il modulo di coincidenze scatta quando c'è una sovrapposizione di due segnali di livello logico 1 per più di \SI{1.5}{ns},
possiamo dire che il $\Delta t$ vale circa il doppio della durata dell'uscita del discriminatore, ovvero \SI{80}{ns}.
Abbiamo allora $R_C\approx 128\cdot10^{-6}\approx1\cdot10^{-4}$,
ovvero una coincidenza casuale ogni 3 ore.

\subsection*{Tempi di propagazione dei segnali}

Usiamo l'oscilloscopio per visualizzare le coincidenze, ovvero colleghiamo i due discriminatori utilizzati al nostro strumento ed osserviamo degli impulsi NIM quasi simultanei, in particolare vediamo che il segnale che arriva più tardi si discosta solo di qualche nanosecondo da quello precedente, ma questa asincronia non ci causa nessun problema perché vale al massimo il 10\% della lunghezza dei nostri impulsi.
Ciò non dovrebbe mascherare l'arrivo di una seconda coincidenza in quell'intervallo di tempo perché il rate atteso è di decine di hertz, che equivale in media al passaggio di un raggio cosmico ogni decimo di secondo in caso di accettanza geometrica unitaria.

Per trovare il miglior punto di lavoro inseriamo nel crate due \emph{delay unit} che colleghiamo in serie ai discriminatori di cui sopra. Prima di effettuare i conteggi testiamo una delay unit mandandole in ingresso l'onda quadra data dal clock del contatore. Confermiamo che il ritardo minimo introdotto da questo modulo è \SI{2.5\pm0.2} {ns} e che tutti gli altri inseribili sono compatibili con quanto dichiarato con un'incertezza integrale di meno del \SI{10}\% e una differenziale di meno di \SI{0.2}{ns}. La \autoref{curv} mostra la curva di cavo che, come atteso, mostra un massimo di conteggi quando il ritardo relativo è inferiore alla durata dell'impulso del discriminatore ed è quasi nulla altrove. La variabile utilizzata è $\Delta t=D2-D4$ dove questi ultimi sono i ritardi selezionati per i rispettivi PMT. Nel $\Delta t$ non compaiono termini dipendenti dai cavi coassiali utilizzati perché i due segnali sono stati collegati ai vari moduli NIM con cavi di uguale lunghezza.

\begin{figure}[h]
\centering
\includegraphics[width=8 cm]{curva_cavo}
\caption{Numero di coincidenze in funzione della differenza dei ritardi impostati sulle delay unit.}
\label{curv}
\end{figure}

% aggiungere il disegno di Giacomuzzo?


\subsection*{Rapporto segnale/fondo}

Dopo aver tolto le delay unit abbiamo variato la tensione di un PMT  e la soglia del discriminatore a cui era collegato mentre l'altro era lasciato ad un punto di lavoro costante in modo che non influenzasse le misure fatte sul primo.
Abbiamo scelto il punto di lavoro costante in modo che sicuramente le coincidenze casuali fossero minori di \SI{0.1}{s^{-1}}.
Abbiamo analizzato dapprima il PMT2 lasciando fisso il 4 ad una tensione nominale $V_4=\SI{1800} V$ ed una soglia $V_{thr_4}=\SI{-353\pm2}{mV}$. 
Il rapporto tra coincidenze e conteggi meno coincidenze stima il rapporto segnale/rumore.
La stessa cosa è stata fatta per il PMT4 lasciando il PMT2 al punto di lavoro V$_2=1820$\! V e V$_{thr_2}=-298\pm2$\! mV.
Risultati in \autoref{test2}.

\begin{figure}[h]
\centering
\includegraphics[width=8 cm]{calib_pmt2}
\includegraphics[width=8 cm]{calib_pmt4}
\caption{Rapporti S/N (a meno di una costante moltiplicativa) per il PMT2 e il PMT4 in diversi punti di lavoro.}
\label{test2}
\end{figure}

\subsection*{Efficienza}

\paragraph{Analisi statistica}

Chiamiamo $k_2$, $k_3$ le coincidenze a due e a tre rispettivamente.
Assumiamo che se una particella passa per i due scintillatori esterni allora passa anche per quello interno.
Assegnamo una distribuzione poissoniana a $k_2$ e binomiale a $k_3$ dato $k_2$:
\begin{align*}
	P(k_2;\mu)
	&= \frac{\mu^{k_2}}{k_2!}e^{-\mu} \\
	P(k_3|k_2;\epsilon)
	&= \binom{k_2}{k_3} \epsilon^{k_3} (1-\epsilon)^{k_2-k_3}
\end{align*}
dove $\epsilon$ è l'efficienza del rivelatore interno.
Calcoliamo la distribuzione congiunta:
\begin{align*}
	P(k_2,k_3;\mu,\epsilon)
	&= P(k_2;\mu) P(k_3|k_2;\epsilon) = \\
	&= \frac{e^{-\mu}}{k_3!(k_2-k_3)!} \big(\mu(1-\epsilon)\big)^{k_2} \left(\frac\epsilon{1-\epsilon}\right)^{k_3},
	\quad k_3 \le k_2.
\end{align*}
Calcoliamo lo stimatore di massima verosimiglianza:
\begin{align*}
	L := \log P(k_2,k_3;\mu,\epsilon)
	&= k_2 \big( \log\mu + \log(1-\epsilon) \big) + {}\\
	&\phantom{{}={}}+ k_3 \big( \log\epsilon - \log(1-\epsilon) \big)
	- \mu + f(k_2,k_3) \\
	\pdv{L}{\epsilon}
	&= \frac{k_3-k_2}{1-\epsilon} + \frac{k_3}\epsilon \implies \\
	\implies \hat\epsilon
	&= \frac{k_3}{k_2} \\
	\pdv{L}{\mu}
	&= \frac{k_2}{\mu} - 1 \implies \\
	\implies \hat\mu
	&= k_2
\end{align*}
Il risultato è quello intuitivo.
Notiamo che $\hat\epsilon$ è definito per $k_2\neq 0$,
infatti per $k_2=0$, ricordando che $k_3\le k_2$, si ha
$L = -\mu + f(k_2,k_3)$ che non dipende da $\epsilon$.
Restringiamo il dominio a $k_2\neq 0$,
quindi ricalcoliamo la normalizzazione:
\begin{align*}
	P(k_2=0,k_3=0)
	&= e^{-\mu} \implies \\
	\implies P(k_2\neq 0,k_3)
	&= \frac{P(k_2,k_3)}{1-e^{-\mu}}.
\end{align*}
Le proprietà di $\hat\mu$ sono immediate,
studiamo $\hat\epsilon$.
Calcoliamo il valore atteso:
\begin{align*}
	E[\hat\epsilon]
	&= E \left[ \frac{k_3}{k_2} \right] = \\
	&= \sum_{k_2\ge k_3} \frac{k_3}{k_2} P(k_2) P(k_3|k_2) = \\
	&= \sum_{k_2=1}^\infty \frac{P(k_2)}{k_2}
	\sum_{k_3=0}^{k_2} P(k_3|k_2) k_3 = \\
	\intertext{riconosciamo che la seconda somma è la media della binomiale}
	&= \sum_{k_2=1}^\infty \frac{P(k_2)}{k_2} k_2 \epsilon = \\
	&= \epsilon \sum_{k_2=1}^\infty P(k_2)
	= \epsilon,
\end{align*}
quindi $\hat\epsilon$ ha bias nullo.
Calcoliamo la varianza:
\begin{align*}
	\operatorname{Var}[\hat\epsilon]
	&= E[\hat\epsilon^2] - E[\epsilon]^2 \\
	E[\hat\epsilon^2]
	&= \sum_{k_2=1}^\infty \frac{P(k_2)}{k_2^2}
	\sum_{k_3=0}^{k_2} P(k_3|k_2) k_3^2 = \\
	\intertext{la seconda somma è $E[k_3^2|k_2]$}
	&= \sum_{k_2=1}^\infty \frac{P(k_2)}{k_2^2}
	\big (k_2\epsilon(1-\epsilon) + k_2^2\epsilon^2 \big) = \\
	&= \epsilon(1-\epsilon) \sum_{k_2=1}^\infty \frac{\mu^{k_2}}{k_2k_2!}\frac{e^{-\mu}}{1-e^{-\mu}}
	+ \epsilon^2 \sum_{k_2=1}^\infty P(k_2) = \\
	&= \frac{\epsilon(1-\epsilon)}{e^{\mu}-1} \sum_{k_2=1}^\infty \frac{\mu^{k_2}}{k_2k_2!} + \epsilon^2.
\end{align*}
Si può dimostrare che\footnote{L'abbiamo calcolato con WolframAlpha.}
\begin{align*}
	\sum_{k_2=1}^\infty \frac{\mu^{k_2}}{k_2k_2!}
	&= \operatorname{Ei}(\mu) - \log\mu - \gamma,
\end{align*}
dove Ei è la funzione integrale esponenziale che è già implementata nelle librerie standard
e $\gamma$ è la costante di Eulero-Mascheroni $\approx 0.6$.
Quindi infine
\begin{align*}
	\operatorname{Var}[\hat\epsilon]
	&= \epsilon(1-\epsilon)\frac{\operatorname{Ei}(\mu) - \log\mu - \gamma}{e^\mu - 1}.
\end{align*}
Vediamo l'andamento per $\mu$ grande.
Vale $\operatorname{Ei}(\mu) \approx e^{\mu}/\mu$, dunque
\begin{align*}
	\operatorname{Var}[\hat\epsilon]
	&\approx \frac{\epsilon(1-\epsilon)}{\mu}
\end{align*}
com'è intuitivo, la varianza scende come $1/\mu$ cioè come il tempo di misurazione.

Quindi come errore indicheremo $\sqrt{\operatorname{Var}[\hat\epsilon]\big|_{\epsilon=\hat\epsilon,\mu=\hat\mu}}$.
Notiamo che questa notazione è poco utile quando $\epsilon$ è vicina a 0 o 1 oppure $\mu$ piccolo.

\paragraph{Misura}

Per misurare l'efficienza del PMT3 è necessario eseguire i rapporti tra le coincidenze di tutti i tre rivelatori e quelle dei due più esterni. Per fare ciò abbiamo usato un secondo modulo di coincidenze per visualizzare quelle di tutti e tre. Non avendo ancora analizzato i dati riguardanti la sezione precedente, abbiamo stimato il punto di lavoro $V_\text{alim}=\SI{1800}{V}$ e $V_{\text{thr}}=\SI{-200}{mV}$ essere adeguato per i fotomoltiplicatori esterni. Quindi abbiamo variato soglia e alimentazione di quello centrale e disegnato il grafico di \autoref{eff}.
I dati misurati sono riportati in \autoref{tab:dati}.

\paragraph{Tasso}

Per stimare il tasso, abbiamo bisogno dell'efficienza dei PMT 2 e 4.
Possiamo stimarla con l'efficienza del PMT3 nello stesso punto di lavoro.
Le coincidenze 2\&4 vengono contate sempre nello stesso punto di lavoro,
quindi possiamo sommare tutte le prese dati.
Il tasso allora è dato da
\begin{equation*}
	R = \frac{\sum c_{24}}{\epsilon_0^2 \sum t}.
\end{equation*}
dove le somme sono sulle prese dati,
$t$ sono i tempi di acquisizione,
$\epsilon_0$ è l'efficienza del PMT3 alimentato a \SI{1800}V con soglia \SI{-200.3}{mV}
e $c_{24}$ sono i conteggi delle coincidenze a due.
Risulta $R=\SI{22.0 \pm 1.4}{s^{-1}}$,
dove l'errore è calcolato con propagazione al primo ordine.
Questo errore non include quello sistematico sull'aver usato l'efficienza di un altro PMT,
che non sappiamo stimare.

Notiamo che questo tasso è riferito a una certa accettanza geometrica.
L'accettanza dipende dalla forma dei rivelatori,
dalla distribuzione dei raggi rivelabili,
e dalla disuniformità spaziale e angolare dell'efficienza.
Possiamo dare una stima dell'accettanza considerando l'angolo solido
descritto da una piramide con vertice al centro di uno dei piani esterni e base l'altro piano esterno,
integrandoci la distribuzione
\[\frac{4\cos^2\theta\,\de\theta}{\pi}\frac{\de\varphi}{2\pi}\]
dove $\theta$ è l'angolo dalla verticale e $\varphi$ l'azimut.
Risulta \SI{87}\%.
Supponendo l'efficienza uniforme,
questa stima è in eccesso.
Abbiamo infine
$R \ge 25$ con errore $\sigma_R>1.6$.

\paragraph{Segnale/rumore}

Analogamente alla misura del tasso,
occorrono le efficienze dei PMT 2 e 4,
che stimiamo uguali a quella del PMT3.
Il segnale è quindi dato da
$S=c_{234}/\epsilon_0^2$,
dove $c_{234}$ sono le coincidenze a tre.
Il rumore è definito come conteggi del PMT3 meno conteggio di segnale,
quindi lo stimiamo con $N=c_3-S$.
Notiamo che questa stima non è costruita in modo che $N$ sia necessariamente positivo.
Riportiamo i risultati in \autoref{fig:sn3}.

\begin{figure}
	\centering
	\includegraphics[width=8 cm]{efficienza}
	\caption{\label{eff}%
	Efficienza del PMT3.
	Ogni curva corrisponde a una certa tensione di alimentazione del PMT3.}
\end{figure}

\begin{figure}
	\centering
	\includegraphics[width=10cm]{SN3}
	\caption{%
	\label{fig:sn3}%
	Rapporto segnale/rumore per il PMT3,
	dove segnale ($S$) e rumore ($N$) sono stimati come
	$S=c_{234}/\epsilon_0^2$,
	$N=c_3 - S$,
	dove $c_3$ è il conteggio del PMT3,
	$c_{234}$ il conteggio della coincidenza a tre,
	$\epsilon_0$ l'efficienza calcolata per il PMT3 nel punto di lavoro dei PMT 2 e 4.
	Le barre d'errore sono calcolate con propagazione al primo ordine.}
\end{figure}

\begin{table}
	\centering
	\begin{tabular}{ll|rrrrr}
		Alim. [$\si V_\text{nom}$] & Soglia [$-\si{mV}$] & PMT4 & PMT3 & PMT2 & 2\&4 & 2\&3\&4 \\ \hline
		1700 & 408 & 1253 & 6 & 750 & 140 & 2               \\
		     & 304.4 & 1200 & 18 & 743 & 143 & 7               \\
		     & 200.3 & 1190 & 63 & 719 & 127 & 21              \\
		     & 101.8 & 1239 & 367 & 828 & 148 & 79           \\ \hline
		1800 & 407 & 1221 & 193 & 791 & 138 & 43        \\
		     & 306.1 & 1301 & 341 & 785 & 141 & 76       \\
		     & 200.3 & 1262 & 657 & 785 & 138 & 110       \\
		     & 101.8 & 1273 & 1415 & 799 & 133 & 131     \\ \hline
		1900 & 407 & 1322 & 1006 & 802 & 148 & 134          \\
		 & 294.3 & 1291 & 1468 & 759 & 138 & 135          \\
		 & 193.8 & 1226 & 3068 & 809 & 134 & 133          \\
		 & 107.1 & 1239 & 137541 & 783 & 144 & 141       \\ \hline
		2000 & 408 & 1235 & 8615 & 769 & 142 & 140     \\
		     & 305.3 & 1212 & 77732 & 799 & 143 & 140     \\
		     & 208.2 & 1221 & 319848 & 787 & 134 & 134       \\
		     & 107.1 & 1239 & 444108 & 791 & 143 & 143    \\
	\end{tabular}
	\caption{\label{tab:dati}
	Dati per la misura dell'efficienza del PMT3.
	Le colonne a sinistra riportano la tensione di alimentazione del PMT3 e la soglia del discriminatore collegato al PMT3,
	misurata dal test point del discriminatore.
	Tensione di alimentazione e soglia dei PMT 2 e 5 sono rispettivamente
	\SI{1800}V, \SI{-200}{mV}.
	Le due colonne più a destra riportano le coincidenze.
	Il conteggio è su un tempo di \num{10000} cicli di clock,
	cioè \SI{10}s a meno dell'errore di clock.}
\end{table}

\end{document}
